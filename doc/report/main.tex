%\title{University of Bristol Thesis Template}
\RequirePackage[l2tabu]{nag}		% Warns for incorrect (obsolete) LaTeX usage
%
%
% File: memoirthesis.tex
% Author: Victor Brena
% Description: Contains the thesis template using memoir class,
% which is mainly based on book class but permits better control of 
% chapter styles for example. This template is an adaptation and 
% modification of Oscar's.
% 
% Memoir is a flexible class for typesetting poetry, fiction, 
% non-fiction and mathematical works as books, reports, articles or
% manuscripts. CTAN repository is found at:
% http://www.ctan.org/tex-archive/macros/latex/contrib/memoir/
%
%
% UoB guidelines for thesis presentation were found at:
% http://www.bris.ac.uk/esu/pg/pgrcop11-12topic.pdf#page=49
%
% UoB guidelines:
%
% The dissertation must be printed on A4 white paper. Paper up to A3 may be used
% for maps, plans, diagrams and illustrative material. Pages (apart from the
% preliminary pages) should normally be double-sided.
%
% Memoir class loads useful packages by default (see manual).
\documentclass[a4paper,11pt,reqno,twoside]{memoir} %add 'draft' to turn draft option on (see below)

%
%
% Adding metadata:
\usepackage{datetime}
\usepackage{ifpdf}
\ifpdf
\pdfinfo{
   /Author (Jan Stocker and Noah H\"utter)
   /Title (Bachelor Thesis)
   /Keywords (FPGA; UDP; Image Processing; Wallis filter)
   /CreationDate (D:\pdfdate)
}
\fi
% When draft option is on. 
\ifdraftdoc 
	\usepackage{draftwatermark}				%Sets watermarks up.
	\SetWatermarkScale{0.3}
	\SetWatermarkText{\bf Draft: \today}
\fi
%
% Declare figure/table as a subfloat.
% \newsubfloat{figure}
% \newsubfloat{table}
% Better page layout for A4 paper, see memoir manual.
\settrimmedsize{297mm}{210mm}{*}
\setlength{\trimtop}{0pt} 
\setlength{\trimedge}{\stockwidth} 
\addtolength{\trimedge}{-\paperwidth} 
\settypeblocksize{634pt}{448.13pt}{*} 
\setulmargins{4cm}{*}{*} 
\setlrmargins{*}{*}{1.5} 
\setmarginnotes{17pt}{51pt}{\onelineskip} 
\setheadfoot{\onelineskip}{2\onelineskip} 
\setheaderspaces{*}{2\onelineskip}{*} 
\checkandfixthelayout
\setlength{\parindent}{0ex} % Set paragraph indentation
%
\frenchspacing
% Font with math support: New Century Schoolbook
\usepackage{fouriernc}
\usepackage[T1]{fontenc}
%
% UoB guidelines:
%
% Text should be in double or 1.5 line spacing, and font size should be
% chosen to ensure clarity and legibility for the main text and for any
% quotations and footnotes. Margins should allow for eventual hard binding.
%
% Note: This is automatically set by memoir class. Nevertheless \OnehalfSpacing 
% enables double spacing but leaves single spaced for captions for instance. 
\OnehalfSpacing 
%
% Sets numbering division level
\setsecnumdepth{subsection} 
\maxsecnumdepth{subsubsection}
%
% Chapter style (taken and slightly modified from Lars Madsen Memoir Chapter 
% Styles document
\usepackage{calc,soul,fourier}
\makeatletter 
\newlength\dlf@normtxtw 
\setlength\dlf@normtxtw{\textwidth} 
\newsavebox{\feline@chapter} 
\newcommand\feline@chapter@marker[1][4cm]{%
	\sbox\feline@chapter{% 
		\resizebox{!}{#1}{\fboxsep=1pt%
			\colorbox{gray}{\color{white}\thechapter}% 
		}}%
		\rotatebox{90}{% 
			\resizebox{%
				\heightof{\usebox{\feline@chapter}}+\depthof{\usebox{\feline@chapter}}}% 
			{!}{\scshape\so\@chapapp}}\quad%
		\raisebox{\depthof{\usebox{\feline@chapter}}}{\usebox{\feline@chapter}}%
} 
\newcommand\feline@chm[1][4cm]{%
	\sbox\feline@chapter{\feline@chapter@marker[#1]}% 
	\makebox[0pt][c]{% aka \rlap
		\makebox[1cm][r]{\usebox\feline@chapter}%
	}}
\makechapterstyle{daleifmodif}{
	\renewcommand\chapnamefont{\normalfont\Large\scshape\raggedleft\so} 
	\renewcommand\chaptitlefont{\normalfont\Large\bfseries\scshape} 
	\renewcommand\chapternamenum{} \renewcommand\printchaptername{} 
	\renewcommand\printchapternum{\null\hfill\feline@chm[2.5cm]\par} 
	\renewcommand\afterchapternum{\par\vskip\midchapskip} 
	\renewcommand\printchaptertitle[1]{\color{gray}\chaptitlefont\raggedleft ##1\par}
} 
% This sets equal page margins albeit two sided document
\setlength\@tempdima       {\paperwidth}
\addtolength\@tempdima     {-\textwidth}
\setlength\oddsidemargin   {.5\@tempdima}
\addtolength\oddsidemargin {-1in}
\setlength\marginparwidth  {.5\@tempdima}
\addtolength\marginparwidth{-\marginparsep}
\addtolength\marginparwidth{-0.8in} % don't know why this isn't .4
\setlength\evensidemargin\oddsidemargin
\makeatother 
\chapterstyle{daleifmodif}


% Page header and footer style
\makepagestyle{myruled}
\makeheadrule {myruled}{\textwidth}{\normalrulethickness}
\makefootrule {myruled}{\textwidth}{\normalrulethickness}{\footruleskip}
\makeevenhead {myruled}{\small\leftmark}{} {}
\makeoddhead {myruled}{}{}{\small\rightmark}
\makeevenfoot {myruled}{}{\thepage} {}
\makeoddfoot {myruled}{}{\thepage} {}
\makeatletter % because of \@chapapp
\makepsmarks {myruled}{
	\nouppercaseheads
	\createmark {chapter} {both} {shownumber}{\@chapapp\ }{. \ }
	% \createmark {section} {both}{nonumber}{} {. \ }
	% \createmark {subsection} {both}{nonumber}{} {. \ }
	% \createmark {subsubsection}{both}{nonumber}{} {. \ }
	\createplainmark {toc} {both} {\contentsname}
	\createplainmark {lof} {both} {\listfigurename}
	\createplainmark {lot} {both} {\listtablename}
	\createplainmark {bib} {both} {\bibname}
	\createplainmark {index} {both} {\indexname}
	\createplainmark {glossary} {both} {\glossaryname}
}
\makeatother
\setsecnumdepth{subsubsection}
\pagestyle{myruled}


%
% Oscar's command (it works):
% Fills blank pages until next odd-numbered page. Used to emulate single-sided
% frontmatter. This will work for title, abstract and declaration. Though the
% contents sections will each start on an odd-numbered page they will
% spill over onto the even-numbered pages if extending beyond one page
% (hopefully, this is ok).
\newcommand{\clearemptydoublepage}{\newpage{\thispagestyle{empty}\cleardoublepage}}
%
%
% Creates indexes for Table of Contents, List of Figures, List of Tables and Index
\makeindex
% \printglossaries below creates a list of abbreviations. \gls and related
% commands are then used throughout the text, so that latex can automatically
% keep track of which abbreviations have already been defined in the text.
%
% The import command enables each chapter tex file to use relative paths when
% accessing supplementary files. For example, to include
% chapters/brewing/images/figure1.png from chapters/brewing/brewing.tex we can
% use
% \includegraphics{images/figure1}
% instead of
% \includegraphics{chapters/brewing/images/figure1}
\usepackage{import}

% Add other packages needed for chapters here. For example:
\usepackage{lipsum}					%Needed to create dummy text
\usepackage{amsfonts} 					%Calls Amer. Math. Soc. (AMS) fonts
\usepackage[centertags]{amsmath}			%Writes maths centred down
\usepackage{stmaryrd}					%New AMS symbols
\usepackage{amssymb}					%Calls AMS symbols
\usepackage{amsthm}					%Calls AMS theorem environment
\usepackage{newlfont}					%Helpful package for fonts and symbols
\usepackage{layouts}					%Layout diagrams
\usepackage{graphicx}					%Calls figure environment
\usepackage{longtable,rotating}			%Long tab environments including rotation. 
\usepackage[applemac]{inputenc}			%Needed to encode non-english characters 
									%directly for mac
\usepackage{colortbl}					%Makes coloured tables
\usepackage{wasysym}					%More math symbols
\usepackage{mathrsfs}					%Even more math symbols
\usepackage{float}						%Helps to place figures, tables, etc. 
\usepackage{verbatim}					%Permits pre-formated text insertion
\usepackage{upgreek }					%Calls other kind of greek alphabet
\usepackage{latexsym}					%Extra symbols
% \usepackage[square,numbers,
% 		     sort&compress]{natbib}		%Calls bibliography commands 
\usepackage[style=ieee,urldate=comp,backend=biber]{biblatex} % fhnw report
\addbibresource{bibtex.bib}
\usepackage{url}						%Supports url commands
\usepackage{etex}						%eTeXÕs extended support for counters
\usepackage{fixltx2e}					%Eliminates some in felicities of the 
									%original LaTeX kernel
\usepackage[spanish,english]{babel}		%For languages characters and hyphenation
\usepackage{color}                    				%Creates coloured text and background
\usepackage[colorlinks=true,
		     allcolors=black]{hyperref}              %Creates hyperlinks in cross references
\usepackage{memhfixc}					%Must be used on memoir document 
									%class after hyperref
\usepackage{enumerate}					%For enumeration counter
\usepackage{footnote}					%For footnotes
\usepackage{microtype}					%Makes pdf look better.
\usepackage{rotfloat}					%For rotating and float environments as tables, 
									%figures, etc. 
\usepackage{alltt}						%LaTeX commands are not disabled in 
\usepackage{setspace}
									%verbatim-like environment
\usepackage[version=0.96]{pgf}			%PGF/TikZ is a tandem of languages for producing vector graphics from a 
\usepackage{tikz,pgfplots}						%geometric/algebraic description.
\usepackage{xcolor}
\usepackage{tabularx}
\usepackage{fmtcount}
\usetikzlibrary{arrows,shapes,snakes,
		       automata,backgrounds,
		       petri,topaths,
		       positioning, spy,
		       fit, external ,calc,
		       arrows.meta, patterns}				%To use diverse features from tikz
\pgfdeclarelayer{background}
\pgfdeclarelayer{foreground}
\pgfsetlayers{background,foreground,main}
\tikzset{>=latex}

\usepackage{xintbinhex}
\usepackage{xintexpr}
\usepackage{listings}
\usepackage{color}
\usepackage[toc,nopostdot,style=altlist,nonumberlist]{glossaries}
% \usepackage[toc,style=longheader,nonumberlist]{glossaries}
\makeglossaries
% ==============================================================================
%
%                             Glossary
%
% ==============================================================================
\newglossaryentry{vivadohlx}
{
    name=Vivado HLx,
    description={Xilinx tool used for synthesis, implementation and debugging}
}
\newglossaryentry{vivadohls}
{
    name=Vivado HLS,
    description={Xilinx tool used for high level synthesis. Hardware is
    described in C/C++ language that is synthesized by the tool to hardware
    description language.}
}
\newglossaryentry{ila}
{
    name=integrated logic analyzer (ILA),
    description={Can be configureed to record FPGA internal signals and send to PC}
}
\newglossaryentry{diip}
{
    name=Distributed Image Processing (diip),
    description={Project nickname of the bachelor thesis}
}
\newglossaryentry{rmse}
{
    name=root mean square error (RMSE),
    description={Used as a metric to measure the difference in two images}
}
\newglossaryentry{bramtile}
{
    name=BRAM tile,
    description={A single block ram instance on the FPGA. In this case one tile
    can be configured as either one 36Kb RAM or two inpdependant 18Kb RAM}
}
\newglossaryentry{ghdl}
{
    name=ghdl,
    description={Open-source simulator used for simulating VHDL language}
}

\usepackage{subfig}
\usepackage{cclicenses}
 
\definecolor{mGreen}{rgb}{0,0.6,0}
\definecolor{mGray}{rgb}{0.5,0.5,0.5}
\definecolor{mPurple}{rgb}{0.58,0,0.82}
\definecolor{backgroundColour}{rgb}{0.95,0.95,0.92}

\lstdefinestyle{CStyle}{
    backgroundcolor=\color{backgroundColour},   
    commentstyle=\color{mGreen},
    keywordstyle=\color{magenta},
    numberstyle=\tiny\color{mGray},
    stringstyle=\color{mPurple},
    basicstyle=\footnotesize,
    breakatwhitespace=false,         
    breaklines=true,                 
    captionpos=b,                    
    keepspaces=true,                 
    numbers=left,                    
    numbersep=5pt,                  
    showspaces=false,                
    showstringspaces=false,
    showtabs=false,                  
    tabsize=2,
    language=C
}

\lstdefinestyle{TextStyle}{
    backgroundcolor=\color{backgroundColour},   
    basicstyle=\footnotesize,
    breakatwhitespace=false,         
    breaklines=false,                 
    captionpos=b,                    
    keepspaces=true,                 
    numbers=left,                    
    % numbersep=5pt,                  
    showspaces=false,                
    showstringspaces=false,
    showtabs=false,                  
    tabsize=2
}

\lstdefinestyle{VHDLStyle}{
    backgroundcolor=\color{backgroundColour},   
    commentstyle=\color{mGreen},
    keywordstyle=\color{magenta},
    numberstyle=\tiny\color{mGray},
    stringstyle=\color{mPurple},
    basicstyle=\footnotesize,
    breakatwhitespace=false,         
    breaklines=true,                 
    captionpos=b,                    
    keepspaces=true,                 
    numbers=left,                    
    numbersep=5pt,                  
    showspaces=false,                
    showstringspaces=false,
    showtabs=false,                  
    tabsize=2,
    language=VHDL
}

\lstdefinestyle{ShellStyle}{
    backgroundcolor=\color{backgroundColour},   
    commentstyle=\color{mGreen},
    keywordstyle=\color{magenta},
    numberstyle=\tiny\color{mGray},
    stringstyle=\color{mPurple},
    basicstyle=\footnotesize,
    breakatwhitespace=false,         
    breaklines=true,                 
    captionpos=b,                    
    keepspaces=true,                 
    numbers=left,                    
    numbersep=5pt,                  
    showspaces=false,                
    showstringspaces=false,
    showtabs=false,                  
    tabsize=2,
    language=bash
}
\lstdefinestyle{TextStyle}{
    backgroundcolor=\color{backgroundColour},   
    commentstyle=\color{mGreen},
    keywordstyle=\color{magenta},
    numberstyle=\tiny\color{mGray},
    stringstyle=\color{mPurple},
    basicstyle=\footnotesize,
    breakatwhitespace=false,         
    breaklines=true,                 
    captionpos=b,                    
    keepspaces=true,                 
    numbers=left,                    
    numbersep=5pt,                  
    showspaces=false,                
    showstringspaces=false,
    showtabs=false,                  
    tabsize=2
}

\usepackage{mathtools}
\DeclarePairedDelimiter\ceil{\lceil}{\rceil}
\DeclarePairedDelimiter\floor{\lfloor}{\rfloor}

% Reduce line skip in itemize env
\let\tempone\itemize
\let\temptwo\enditemize
\renewenvironment{itemize}{\tempone\addtolength{\itemsep}{-0.5\baselineskip}}{\temptwo}

% Reduce line skip in enumerate env
\let\tempone\enumerate
\let\temptwo\endenumerate
\renewenvironment{enumerate}{\tempone\addtolength{\itemsep}{-0.5\baselineskip}}{\temptwo}


\usepackage[colorinlistoftodos,prependcaption]{todonotes}
\usepackage[section]{placeins} 		% This prevents placing floats (therefore also figures) before a section.
\usepackage{tabularx}
\usepackage{adjustbox}
\usepackage[bottom]{footmisc} 	 	% Allows footnotes to be placed below bottom floats
\usepackage{pdfpages}
\usepackage{cellspace, multirow, booktabs}


%							
%Reduce widows  (the last line of a paragraph at the start of a page) and orphans 
% (the first line of paragraph at the end of a page)
\widowpenalty=1000
\clubpenalty=1000
%
% New command definitions for my thesis
%
\newcommand{\keywords}[1]{\par\noindent{\small{\bf Keywords:} #1}} %Defines keywords small section
\newcommand{\parcial}[2]{\frac{\partial#1}{\partial#2}}                             %Defines a partial operator
\newcommand{\vectorr}[1]{\mathbf{#1}}                                                        %Defines a bold vector
\newcommand{\vecol}[2]{\left(                                                                         %Defines a column vector
	\begin{array}{c} 
		\displaystyle#1 \\
		\displaystyle#2
	\end{array}\right)}
\newcommand{\mados}[4]{\left(                                                                       %Defines a 2x2 matrix
	\begin{array}{cc}
		\displaystyle#1 &\displaystyle #2 \\
		\displaystyle#3 & \displaystyle#4
	\end{array}\right)}
\newcommand{\pgftextcircled}[1]{                                                                    %Defines encircled text
    \setbox0=\hbox{#1}%
    \dimen0\wd0%
    \divide\dimen0 by 2%
    \begin{tikzpicture}[baseline=(a.base)]%
        \useasboundingbox (-\the\dimen0,0pt) rectangle (\the\dimen0,1pt);
        \node[circle,draw,outer sep=0pt,inner sep=0.1ex] (a) {#1};
    \end{tikzpicture}
}
\newcommand{\range}[1]{\textnormal{range }#1}                                             %Defines range operator
\newcommand{\innerp}[2]{\left\langle#1,#2\right\rangle}                                 %Defines inner product
\newcommand{\prom}[1]{\left\langle#1\right\rangle}                                         %Defines average operator
\newcommand{\tra}[1]{\textnormal{tra} \: #1}                                                       %Defines trace operator
\newcommand{\sign}[1]{\textnormal{sign\,}#1}                                                   %Defines sign operator
\newcommand{\sech}[1]{\textnormal{sech} #1}                                                  %Defines sech
\newcommand{\diag}[1]{\textnormal{diag} #1}                                                    %Defines diag operator
\newcommand{\arcsech}[1]{\textnormal{arcsech} #1}                                       %Defines arcsech
\newcommand{\arctanh}[1]{\textnormal{arctanh} #1}                                         %Defines arctanh
%Change tombstone symbol
\newcommand{\blackged}{\hfill$\blacksquare$}
\newcommand{\whiteged}{\hfill$\square$}
\newcounter{proofcount}
\renewenvironment{proof}[1][\proofname.]{\par
 \ifnum \theproofcount>0 \pushQED{\whiteged} \else \pushQED{\blackged} \fi%
 \refstepcounter{proofcount}
 \normalfont 
 \trivlist
 \item[\hskip\labelsep
       \itshape
   {\bf\em #1}]\ignorespaces
}{%
 \addtocounter{proofcount}{-1}
 \popQED\endtrivlist
}
%
%
% New definition of square root:
% it renames \sqrt as \oldsqrt
\let\oldsqrt\sqrt
% it defines the new \sqrt in terms of the old one
\def\sqrt{\mathpalette\DHLhksqrt}
\def\DHLhksqrt#1#2{%
\setbox0=\hbox{$#1\oldsqrt{#2\,}$}\dimen0=\ht0
\advance\dimen0-0.2\ht0
\setbox2=\hbox{\vrule height\ht0 depth -\dimen0}%
{\box0\lower0.4pt\box2}}
%
% My caption style
\newcommand{\mycaption}[2][\@empty]{
	\captionnamefont{\scshape} 
	\changecaptionwidth
	\captionwidth{0.9\linewidth}
	\captiondelim{.\:} 
	\indentcaption{0.75cm}
	\captionstyle[\centering]{}
	\setlength{\belowcaptionskip}{10pt}
	\ifx \@empty#1 \caption{#2}\else \caption[#1]{#2}
}
%
% My subcaption style
% \newcommand{\mysubcaption}[2][\@empty]{
% 	\subcaptionsize{\small}
% 	\hangsubcaption
% 	\subcaptionlabelfont{\rmfamily}
% 	\sidecapstyle{\raggedright}
% 	\setlength{\belowcaptionskip}{10pt}
% 	\ifx \@empty#1 \subcaption{#2}\else \subcaption[#1]{#2}
% }
%
%An initial of the very first character of the content
\usepackage{lettrine}
\newcommand{\initial}[1]{%
	\lettrine[lines=3,lhang=0.33,nindent=0em]{
		\color{gray}
     		{\textsc{#1}}}{}}
% To print roman numbers
\makeatletter
\newcommand*{\rom}[1]{\expandafter\@slowromancap\romannumeral #1@}
\makeatother
%
% Theorem styles used in my thesis
%
\theoremstyle{plain}
\newtheorem{theo}{Theorem}[chapter]
\theoremstyle{plain}
\newtheorem{prop}{Proposition}[chapter]
\theoremstyle{plain}
\theoremstyle{definition}
\newtheorem{dfn}{Definition}[chapter]
\theoremstyle{plain}
\newtheorem{lema}{Lemma}[chapter]
\theoremstyle{plain}
\newtheorem{cor}{Corollary}[chapter]
\theoremstyle{plain}
\newtheorem{resu}{Result}[chapter]
%
% Hyphenation for some words
%
\hyphenation{res-pec-tively}
\hyphenation{mono-ti-ca-lly}
\hyphenation{hypo-the-sis}
\hyphenation{para-me-ters}
\hyphenation{sol-va-bi-li-ty}
%
%

\begin{document}



% ==============================================================================
%
%                            F R O N T M A T T E R
%
% ==============================================================================
\frontmatter
\pagenumbering{roman}

% <<< -------------------------------------------------------------- TITLEPAGE %
\begin{titlingpage}
\begin{SingleSpace}
\calccentering{\unitlength} 
\begin{adjustwidth*}{\unitlength}{-\unitlength}
\vspace*{0mm}
\begin{center}
\rule[0.5ex]{\linewidth}{2pt}\vspace*{-\baselineskip}\vspace*{3.2pt}
\rule[0.5ex]{\linewidth}{1pt}\\[\baselineskip]

{\HUGE Distributed FPGA for enhanced Image Processing }\\[4mm]
{\Large \textit{Bachelor Thesis}}\\

\rule[0.5ex]{\linewidth}{1pt}\vspace*{-\baselineskip}\vspace{3.2pt}
\rule[0.5ex]{\linewidth}{2pt}\\
\vspace{2mm}
{\large By}\\
\vspace{2mm}

{\large\textsc{Noah H\"utter and Jan Stocker}}\\

\vspace{11mm}

\includegraphics[width=0.88\textwidth]{images/titlepage/p6_titlepage.png}\\

%\vspace{6mm}
%{\large Department of Engineering Mathematics\\
%\textsc{University of Bristol}}\\
\vspace{2mm}

\begin{tabular}{>{\bfseries}rl}
        \large Degree Program   & Electrical and Information Technology \\[2mm]
        \large Coach            & Michael Pichler \\[2mm]
        \large Expert           & Dr. J\"urg M. Stettbacher\\[2mm]
        \large Team             & Noah H\"utter, Jan Stocker \\[2mm]
        \large Date             & \today \\[2mm]
\end{tabular}

%\vspace{9mm}
%{\large\textsc{January 2018}}
\vspace{12mm}
\end{center}
\end{adjustwidth*}
\end{SingleSpace}

\tikzexternaldisable
\begin{tikzpicture}[remember picture,overlay]
    \node[anchor=north east,yshift=-8mm,xshift=-8mm,] 
        at (current page.north east) {\includegraphics[width=10cm]{images/titlepage/logo-fhnw.pdf}};
\end{tikzpicture}
\tikzexternalenable

\end{titlingpage}

% ------------------------------------------------------------------------->>> %

% <<< ------------------------------------------------------------------- INFO %
% Source
% https://raw.githubusercontent.com/alpenwasser/pitaya/master/doc/report/chunks/info.tex


\vspace*{30mm}

\begin{small}
    \begin{tabular}{lll}
        Content \& Design & \copyright~2018 & Noah H\"utter \\
                          &                  & Jan Stocker  \\
    \end{tabular}


    \vspace{3em}
    \begin{tabular}{p{.9\textwidth}}
        This Bachelor Thesis was created in the spring semester of 2018 at the
        FHNW University of Applied  Sciences and Arts Northwestern Switzerland
        in the degree program \emph{Electrical and Information Technology}.\\
        \\
        \\
    \end{tabular}
    
    \vspace{3em}

    \begin{tabular}{p{.9\textwidth}}
        The code repository to this project is available at\\
        \url{https://gitlab.fhnw.ch/noah.huetter/diip}
        \\
        \\
    \end{tabular}
    \vspace{3em}

    \input{build/revision.tex}
    \begin{tabular}{lp{.9\textwidth}}
        Last compiled on & \compiledate \\
        Based on commit  & \revision \\
        On host          & \hostname \\
        \\
        \\
    \end{tabular}

    \vspace{3em}
    \begin{tabular}{>{\ttfamily}llp{88mm}}
        \textbf{Version History} \\[1ex]
        Version 1.0.0 & June 13, 2018 & Initial Version\\
        Version 1.0.1 & August 10, 2018 & Version for proofreading\\
    \end{tabular}
\end{small}
% ------------------------------------------------------------------------->>> %

% <<< --------------------------------------------------------------- ABSTRACT %
\renewcommand{\abstractname}{Abstract}
{\pagestyle{plain}{
	\setcounter{tocdepth}{2}
	\chapter*{}
\begin{abstract}
% \linespread{2}\selectfont

In the world of self-driving cars and virtual reality games it is becoming
increasingly important to represent digitally what we see.
Therefore, using high resolution
cameras, images of the environment have been recorded.
These large images are processed to be presented three dimensionally. 
This image processing task needs to be accelerated to have a fast work flow. 
A dedicated hardware approach using Field Programmable Gate Arrays (FPGA) was
implemented that is scalable onto multiple FPGAs.
High level synthesis was used to describe a Sobel filter operation in
C language that was then synthesized to hardware description language.
Simulations were used to ensure correct operation. The result is an image
processing core and a file transfer protocol stack on top of the User Datagram
Protocol (UDP) to transfer the images from a computer to the FPGA. The
processing core takes image data from FPGA block memory and processes data at
one
pixel per clock at 100MHz which corresponds to 100 megapixels per second. The
project
runs on
an Artix 7 evaluation board and can simply be extended due to a clean source code
structure. 
With some additions, workload distribution is possible. This study proves 
that a dedicated hardware approach for image processing is possible and will
speed up the process of creating virtual representations of
reality.

\vspace{3em}
\begin{tabular}{l p{.95\textwidth}}
    \textbf{Team members}  & Noah H\"utter, Jan Stocker \\
    \textbf{Keywords}      & FPGA, UDP, Image Processing, Sobel Filter \\
\end{tabular}

\end{abstract}

% % % % % % % % % % % % % % % % % % % % % % % % % % % % % % % % % % % % 
% Motivation:
% Why do we care about the problem and the results? If the problem isn't obviously "interesting" it might be better to put motivation first; but if your work is incremental progress on a problem that is widely recognized as important, then it is probably better to put the problem statement first to indicate which piece of the larger problem you are breaking off to work on. This section should include the importance of your work, the difficulty of the area, and the impact it might have if successful.
% 
% Why do we care about the problem? 
% What practical, scientific, theoretical or artistic gap is your research filling?
% 
% 
% % % % % % % % % % % % % % % % % % % % % % % % % % % % % % % % % % % % 
% Problem statement:
% What problem are you trying to solve? What is the scope of your work (a generalized approach, or for a specific situation)? Be careful not to use too much jargon. In some cases it is appropriate to put the problem statement before the motivation, but usually this only works if most readers already understand why the problem is important.
% 
% % % % % % % % % % % % % % % % % % % % % % % % % % % % % % % % % % % % 
% Approach:
% How did you go about solving or making progress on the problem? Did you use simulation, analytic models, prototype construction, or analysis of field data for an actual product? What was the extent of your work (did you look at one application program or a hundred programs in twenty different programming languages?) What important variables did you control, ignore, or measure?

% % % % % % % % % % % % % % % % % % % % % % % % % % % % % % % % % % % % 
% Results:
% What's the answer? Specifically, most good computer architecture papers conclude that something is so many percent faster, cheaper, smaller, or otherwise better than something else. Put the result there, in numbers. Avoid vague, hand-waving results such as "very", "small", or "significant." If you must be vague, you are only given license to do so when you can talk about orders-of-magnitude improvement. There is a tension here in that you should not provide numbers that can be easily misinterpreted, but on the other hand you don't have room for all the caveats.

% % % % % % % % % % % % % % % % % % % % % % % % % % % % % % % % % % % % 
% Conclusions:
% What are the implications of your answer? Is it going to change the world (unlikely), be a significant "win", be a nice hack, or simply serve as a road sign indicating that this path is a waste of time (all of the previous results are useful). Are your results general, potentially generalizable, or specific to a particular case?
% 
% https://users.ece.cmu.edu/~koopman/essays/abstract.html
}}
% \chapter*{}
\begin{abstract}
% \linespread{2}\selectfont

In the world of self-driving cars and virtual reality games it is becoming
increasingly important to represent digitally what we see.
Therefore, using high resolution
cameras, images of the environment have been recorded.
These large images are processed to be presented three dimensionally. 
This image processing task needs to be accelerated to have a fast work flow. 
A dedicated hardware approach using Field Programmable Gate Arrays (FPGA) was
implemented that is scalable onto multiple FPGAs.
High level synthesis was used to describe a Sobel filter operation in
C language that was then synthesized to hardware description language.
Simulations were used to ensure correct operation. The result is an image
processing core and a file transfer protocol stack on top of the User Datagram
Protocol (UDP) to transfer the images from a computer to the FPGA. The
processing core takes image data from FPGA block memory and processes data at
one
pixel per clock at 100MHz which corresponds to 100 megapixels per second. The
project
runs on
an Artix 7 evaluation board and can simply be extended due to a clean source code
structure. 
With some additions, workload distribution is possible. This study proves 
that a dedicated hardware approach for image processing is possible and will
speed up the process of creating virtual representations of
reality.

\vspace{3em}
\begin{tabular}{l p{.95\textwidth}}
    \textbf{Team members}  & Noah H\"utter, Jan Stocker \\
    \textbf{Keywords}      & FPGA, UDP, Image Processing, Sobel Filter \\
\end{tabular}

\end{abstract}

% % % % % % % % % % % % % % % % % % % % % % % % % % % % % % % % % % % % 
% Motivation:
% Why do we care about the problem and the results? If the problem isn't obviously "interesting" it might be better to put motivation first; but if your work is incremental progress on a problem that is widely recognized as important, then it is probably better to put the problem statement first to indicate which piece of the larger problem you are breaking off to work on. This section should include the importance of your work, the difficulty of the area, and the impact it might have if successful.
% 
% Why do we care about the problem? 
% What practical, scientific, theoretical or artistic gap is your research filling?
% 
% 
% % % % % % % % % % % % % % % % % % % % % % % % % % % % % % % % % % % % 
% Problem statement:
% What problem are you trying to solve? What is the scope of your work (a generalized approach, or for a specific situation)? Be careful not to use too much jargon. In some cases it is appropriate to put the problem statement before the motivation, but usually this only works if most readers already understand why the problem is important.
% 
% % % % % % % % % % % % % % % % % % % % % % % % % % % % % % % % % % % % 
% Approach:
% How did you go about solving or making progress on the problem? Did you use simulation, analytic models, prototype construction, or analysis of field data for an actual product? What was the extent of your work (did you look at one application program or a hundred programs in twenty different programming languages?) What important variables did you control, ignore, or measure?

% % % % % % % % % % % % % % % % % % % % % % % % % % % % % % % % % % % % 
% Results:
% What's the answer? Specifically, most good computer architecture papers conclude that something is so many percent faster, cheaper, smaller, or otherwise better than something else. Put the result there, in numbers. Avoid vague, hand-waving results such as "very", "small", or "significant." If you must be vague, you are only given license to do so when you can talk about orders-of-magnitude improvement. There is a tension here in that you should not provide numbers that can be easily misinterpreted, but on the other hand you don't have room for all the caveats.

% % % % % % % % % % % % % % % % % % % % % % % % % % % % % % % % % % % % 
% Conclusions:
% What are the implications of your answer? Is it going to change the world (unlikely), be a significant "win", be a nice hack, or simply serve as a road sign indicating that this path is a waste of time (all of the previous results are useful). Are your results general, potentially generalizable, or specific to a particular case?
% 
% https://users.ece.cmu.edu/~koopman/essays/abstract.html

\pagestyle{plain}
\renewcommand{\chaptermark}[1]{\markboth{#1}{}}


%

\chapter*{Dedication and acknowledgements}
\begin{SingleSpace}
\todo[inline]{Fill that stuff}
% We wish to express our sincere thanks to ..., Principal of the Faculty, for
% providing us with all the necessary facilities for the research.
% \\

% We place on record, our sincere thank you to ... Dean of the Faculty, for the
% continuos encouragement.
% \\

% We are also grateful to ..., lecturer, in the Department of .... We are extremly
% thankful and indebted to him for sharing expertise, sincere and valuable
% guidance and encouragement extended to us.
% \\

% We take this opportunity to express gratitude to all of the Department faculty
% members for their help and support. We also thank our parents for the unceasing
% encouragement, support and attention.


We wish to express our sincere thanks to those who supported us in completing
this Thesis. With technical support from the guys at Nomoko, in particular Hakki
and
Luc, the implementation of the Wallis filter became possible. We are
grateful to Michael Pichler, lecturer and Thesis coach, for providing us with
his profound knowledge in FPGA development. 
\\

Further, we would like to place on record, our sincere thanks to the people who
have made themselves available for proofreading our work. In terms of spelling
and reading flow, Tabea Berger and Anita Gertiser provided us with their
thoughts on the text. The feedback from Thomas H\"utter and Dino Zardet helped
us phrase the findings and our line of thought in a comprehensible manner.


% Gegenlesen
% - Anita Gertiser
% - Tabea Berger
% - Thomas H\"utter
% - Dino Zardet
%
% Technische Unterstützung
% - Michael Pichler
% - Luc, Hakki from Nomoko for image processing
%
%

 \end{SingleSpace} \clearpage
\clearemptydoublepage
%
\chapter*{Author's declaration}
\begin{SingleSpace}
\begin{quote}
We declare that the work in this thesis was carried out in accordance with
the requirements of the University's Regulations and Code of Practice for 
Degree Programmes and that it has not been submitted for any other
academic award.

Except where indicated by specific reference in the text, the
work is the candidate's own work. Work done in collaboration with, or with the
assistance of others, is indicated as such. Any views expressed in the
thesis are those of the authors.

\vspace{3cm}

\begin{table}[h!]
    \centering
    \begin{tabular}{p{5cm} p{3cm} p{5cm}}
         & & \\ \cline{1-1} \cline{3-3} 
        \vspace{1ex} & & \\
        \multicolumn{1}{c}{Noah H\"utter} &
        August 17, 2018
         & \multicolumn{1}{c}{Jan Stocker} \\
    \end{tabular}
\end{table}

% \vspace{2cm}
% \noindent
% \hspace{-0.75cm}
% \textsc{SIGNED: ....................................................
% DATE: ..........................................}
% Noah H\"utter


% \vspace{1.5cm}
% \noindent
% \hspace{-0.75cm}
% \textsc{SIGNED: ....................................................
% DATE: ..........................................}
% Jan Stocker

\end{quote}
\end{SingleSpace}
\clearpage
\clearemptydoublepage
%


\newpage
% ------------------------------------------------------------------------->>> %

% <<< ------------------------------------------------------ TABLE of CONTENTS %
{\pagestyle{plain}{
\setcounter{tocdepth}{2}
\begin{KeepFromToc} % remove TOC from TOC
  \tableofcontents
\end{KeepFromToc}
%}}

\pagestyle{plain}
%\pagenumbering{arabic}
\renewcommand{\chaptermark}[1]{\markboth{#1}{}}
% ------------------------------------------------------------------------->>> %




% ==============================================================================
%
%                             M A I N M A T T E R
%
% ==============================================================================
\mainmatter
\pagestyle{myruled}
% <<< ----------------------------------------------------------- INTRODUCTION %

% ==============================================================================
%
%                             Introduction
%
% ==============================================================================
\chapter{Introduction}
% Field/Context
Self driving cars are getting more popular and virtual reality video games
increasingly find their way into people's living rooms. One thing they share is
the need for a digital copy of the world. Self driving cars are trained in such
worlds to
accelerate the development of the algorithm and virtual reality games are
becoming more realistic. The Zurich based company Nomoko \footnote{
\url{https://www.nomoko.world/}} is developing a
technology that will enable the creation of digital copies of the world. They
built a giga pixel camera and a 3D pipeline to make this happen. This pipeline
consists among other things of high volume image processing. Modern computers
and graphical processing units (GPU) are fast in sequentially processing data
but are designed to serve many different tasks and not a specific one. A
dedicated hardware approach designed for a specific image processing task would
expedite the 3D pipeline of creating digital copies of the world.
\\

% Project goal/aims
The goal of this project is to implement such an image processing task on a
Field
Programmable Gate Array (FPGA). FPGAs
consist of thousands of logical elements that can be configured and connected
together to form a complex logical operation. Together with on chip
memory they offer high throughput by processing the data parallel in contrast to
sequential. The data is transfered to the FPGA over an Ethernet LAN connection
for
fast transfer rates. To accelerate the computing even further, the system needs
to be scalable to multiple FPGA boards to distribute the workload.
\\

% Primary objectives
Thus, two primary objectives can be defined:
\begin{enumerate}
    \item The image needs to be transfered from a computer to the FPGA board and
    back
    \item The image data on the FPGA must be processed using a common
    image processing task
\end{enumerate}
\begin{figure}[t!]
    \centering
    \input{images/introduction/dataflow.tikz}
    \caption{Data Flow}
    \label{fig:datafl}
\end{figure}

% What is used to realize...
To realize an Ethernet communication multiple protocols and standards must be
implemented. Ready to use solutions are available. These are compared and
selected. To keep it simple, User Datagram Protocol (UDP) is chosen for
transport layer. Due to UDP's lack of reliable data transfer a session protocol
is defined on top of UDP. This UDP file transfer protocol is implemented using
hardware description language (HDL).

When working with mathematically more complex problems like image processing,
writing the code in hardware description language is complex and
incomprehensible. Therefore a high level synthesis (HLS) approach is put to use.
The algorithm is described and thoroughly tested in C/C++ and then synthesized to
HDL by the Xilinx Vivado HLS toolchain. A Sobel filter operation is chosen as
image processing task to detect edges in the source image. This operation is
rather simple but serves the purpose of this project and can be used for
improved edge detection algorithms.

An Artix7 Evaluation Kit by Xilinx serves as a development and testing platform.
It is equipped with gigabit Ethernet LAN and an FPGA with sufficient logic
elements and memory for both the communication and image processing task.
\\

% What is the result of the project
The result is an image processing core that runs a Sobel filter operation on an
input
image to detect edges. It processes one eight-bit pixel per clock cycle at
100MHz
resulting in a 100 megapixels per second throughput. The
data is read from FPGA block memory and stored there as well. An Ethernet
communication stack written in VHDL handles layers two through five of the OSI
model and transfers data at 8.9 MB/s to a computer using a custom file
transfer protocol.
\\

% How the document is built up
This report is split into five main parts: Theoretical background, mission,
image processing, communication and verification. 
The theoretical background starting
on page \pageref{chapt:theoreticalback} explains the basics of image
processing, FPGA and Ethernet communication. Starting on page 
\pageref{chapt:mission} the chapter mission will cover the starting point and
presents the concept. Chapters \ref{chapt:image_processing} and \ref{chapt:dataflow} cover the actual implementation process of the image processing and communication parts
before verifying these components in chapter \ref{chapt:ver_bench} starting
on page \pageref{chapt:ver_bench}.


% ------------------------------------------------------------------------->>> %



%%%%%%%%%%%%%%%%%%%%%%%%%%%%%%%%%%%%%%%%%%%%%%%%%%%%%%%%%%%%%%%%%%%%%%%%%%%%%%%%
% 							P R O J E C T   R E P O R T  					   %
%%%%%%%%%%%%%%%%%%%%%%%%%%%%%%%%%%%%%%%%%%%%%%%%%%%%%%%%%%%%%%%%%%%%%%%%%%%%%%%%
\part*{Project Report}
\label{part:project_report}

% <<< ----------------------------------------------------------------- THEORY %
% ==============================================================================
%
%                             Theory
%
% ==============================================================================
% -----------------------------------------------------------------------------
\chapter{Theoretical Background} \label{chapt:theoreticalback}
% -----------------------------------------------------------------------------
This chapter provides the basics of the project. On the one hand, it is about image and image processing as well as communication via Ethernet. How an image is recorded and what is necessary for filters and sensors is explained in the chapter \ref{chapt:imag}. Image processing follows in the chapter \ref{chapt:theroy_imageprocessing} and the communication in the chapter \ref{chapt:ethernet}

% ==============================================================================
%
%                                   Image
%
% ==============================================================================
\section{Image} \label{chapt:imag}
This section explains how an image is recorded by the camera. Also, how an image filter is built for a color image.

\subsection{Image Sensor}
An image sensor is a light sensor array that detects light intensity.
In this process light is converted into an electrical signal and saved. The most commonly used image sensors in the visible range are charge-coupled devices (CCD) and active pixel sensors (CMOS). In order not to distort the light intensities in the visible range, an IR filter is often placed in front of the sensor (see figure \ref{fig:image_sensor}).
\\

\textbf{CCD:} With the CCD sensor, each pixel is shifted to the output. At the output the signal is amplified by an amplifier. Manufacturing is therefore cheaper than the CMOS sensor and is used for cheap cameras \cite{ccd_cmos}.
\\

\textbf{CMOS:} CMOS sensors have an amplifier for each pixel. Thus CMOS sensors are faster than CCD sensors and are used in most cameras \cite{ccd_cmos}.

\subsection{Bayer Filter}
A bayer filter is a color filter array for arranging RGB color filters on a
square grid of photosensors (see figure \ref{fig:image_sensor}). There are twice as many green as red and blue
pixels in order to emulate the higher sensitivity of the human eye to green
light. Each pixel only measures one color due to the filter. The two missing
colors are interpolated in post processing from the color values of the eight adjacent pixels.
This technique is called bayer demosaicing.
\begin{figure}[tb!]
    \centering
    \includegraphics[width=0.7\textwidth]{images/theory/image_sensor.png}
    \caption{An image sensor with a bayer filter \cite{image_sensor}}
    \label{fig:image_sensor}
\end{figure}

% ==============================================================================
%
%                            Image Processing
%
% ==============================================================================
\section{Image Processing} \label{chapt:theroy_imageprocessing}
In technical terms, image processing is the processing of image data. This
includes image processing, image analysis and the output of image files 
\cite{image_processing}. Procedures that generate a new image can be
distinguished into point operations, neighborhood operations and global operations based on their input data.

\subsection{Point Operations}
The point operations use the color or intensity information at a given pixel in the image as input, calculates a new intensity value as the result and stores it to the same point in the target image (figure \ref{fig:image_operation}a). Typical applications of point operations are, for example, the correction of contrast and brightness, color correction by rotating the color space or the application of different threshold value methods.

Example for a point operation with $u$ and $v$ being pixel coordinates:
\begin{equation}
    I'(u, v) = f(I(u, v))
    \label{eq:point_operation}
\end{equation}

\subsection{Neighborhood Operations}
Neighborhood operations use a certain amount of neighboring pixels as input (figure \ref{fig:image_operation}b).
They calculate the result and stored it at the reference point in the target
image. Neighborhood operations are often used in convolution filters. These
filters can be used, for instance, to implement smoothing filters such as the
Gauss filter. Convolution filters can also be used to detect edges in an image. This is possible, for example, with the Sobel filter.

Example to calculate the x-derivative and y-derivative with the Sobel matrix \cite{sobel_matrix}:

\noindent\begin{minipage}{.5\linewidth}
\begin{equation}
    G_{x} = I * \begin{bmatrix}
                -1 & 0 & 1 \\ 
                -2 & 0 & 2 \\ 
                -1 & 0 & 1
                \end{bmatrix}
    \label{eq:neighborhood_operation}
\end{equation}

\end{minipage}%
\begin{minipage}{.5\linewidth}

\begin{equation}
    G_{y} = I * \begin{bmatrix}
                -1 & -2 & -1 \\ 
                0 & 0 & 0 \\ 
                1 & 2 & 1
                \end{bmatrix}
    \label{eq:neighborhood_operation}
\end{equation} 

\end{minipage}

\subsubsection*{Border Handling}
With the application of filters, the so-called border handling problem occurs in every image (see figure \ref{fig:image_handling}). Because the filter protrudes beyond the original image.

There are several possible solutions to this issue:
\begin{itemize}
\item The border pixels are not considered. However, the output image is two
pixels smaller in its height and width than the original image
\item If the filter exceeds the original image, the filter coefficients
at the outside of the image are set to zero
\item The pixels required outside the image are extrapolated according to the closest pixels
\item The image is continued periodically
\end{itemize}
    
\begin{figure}[b!]
    \centering
    \includegraphics[width=0.6\textwidth]{images/theory/border_handling.png}
    \caption{Border problem with filter operations \cite{border_handling}}
    \label{fig:image_handling}
\end{figure}

\subsection{Global Operations}
Image analysis often employs global image operations that uses the entire image as input data (figure \ref{fig:image_operation}c). It is often about finding regions or recognizing geometrical objects. A typical representative of this is the Hough transform and the Fourier transform.
\begin{figure}[tb!]
    \centering
    \includegraphics[width=\textwidth]{images/theory/image_operations.jpg}
    \caption{(a) point, (b) neighborhood and (c) global image processing
    operations \cite{image_operation}}
    \label{fig:image_operation}
\end{figure}

% ==============================================================================
%
%                             Ethernet
%
% ==============================================================================
\section{Ethernet Communication} \label{chapt:ethernet}
Exchanging data between two devices can be done using different approaches. The
following section contains an overview of the communication standards used in
local area networks (LAN).

\subsection{Open Systems Interconnection (OSI) Model}
Such a telecommunication system can be characterized by the Open Systems 
Interconnection Model. The OSI model is a stack of seven abstraction layers 
grouped into two groups: The host layers and the media layers (see figure \ref{fig:osi}). 
Each layer serves the layer above it and is served with data from the layer
beneath it. In the following chapters the OSI reference model is used to
characterize the Ethernet standard.

\begin{figure}[tb!]
    \centering
    \includegraphics[width=0.5\textwidth]{images/theory/osi.png}
    \caption{OSI Model \cite{osi}}
    \label{fig:osi}
\end{figure}

\subsection{Physical} \label{chapt:physical}
The first layer of the OSI model is the physical layer. It defines the 
electrical and physical specification of the connection. In the case of local
area networks the connection medium is usually copper. The circuitry required to
implement
the physical layer is done by the PHY-Chip. This integrated circuit provides 
digital access through a media independent interface (MII) to the analog physical
data link.

\subsection{Data Link} 
The data served from the physical layer is then passed to the data link layer.
Its main purpose is to ensure a reliable transfer of data frames between two
nodes connected by a physical layer. It may also provide means to detect errors
that may occur in the physical layer. Ethernet is the protocol used in the data
layer of local area networks and the layer is split into two sublayers, the logical
link control (LLC) and the medium access control (MAC). The LLC provides means
to allow multiple network protocols (OSI layer three) to be multiplexed onto
the same medium. The MAC encapsulates higher level frames into frames 
appropriate to be transmitted by the physical layer.
\\

Figure \ref{fig:eth} shows an Ethernet frame. The first seven bytes consist of a
fixed preamble. It allows devices on the network to easily synchronize their 
clocks for bit-level synchronization. It is followed by the start frame delimiter
(SFD) that marks the beginning of a frame. Sender and receiver MAC addresses 
ensure that the packet is received by the corresponding host and that it can 
reply to the sender. The type field indicates the protocol used on the next layer
(network layer). After the data payload a frame check sequence in form of a CRC
(cyclic redundancy check) is sent to provide error detection. The maximum data
payload size is limited to 1500 bytes.
\\

\clearpage
\begin{figure}[tb!]
    \centering
    \begin{adjustbox}{max width=\textwidth}
        \begin{tikzpicture}[
    rounded corners=0mm,
]
    %nodes
    \node[draw, minimum height=1.0cm] (pre) {Preamble};
    \node[draw, minimum height=1.0cm, right = 0cm of pre] (sfd) {SFD};
    \node[draw, minimum height=1.0cm, right = 0cm of sfd] (dst) {Destination MAC Adr.};
    \node[draw, minimum height=1.0cm, right = 0cm of dst] (src) {Source MAC Adr.};
    \node[draw, align = center, text width=1cm, minimum height=1.0cm, right = 0cm of src] (tp) {Type\\Field};
    \node[draw, minimum height=1.0cm, right = 0cm of tp] (dat) {Data (46 - 1500 Bytes)};
    \node[draw, minimum height=1.0cm, right = 0cm of dat] (pad) {PAD};
    \node[draw, minimum height=1.0cm, right = 0cm of pad] (crc) {CRC};

    \path[draw,-] ($(dst.180) + (0,0)$) -- ++(0,1.2) ;
    \path[draw,-] ($(crc.0) + (0,0)$) -- ++(0,1.2) ;
    \path[draw,{Latex[length=2.5mm]}-{Latex[length=2.5mm]}] ($(dst.180) + (0,1.0)$) -- ($(crc.0) + (0,1.0)$) node [midway, above] () {Basic MAC Frame} ;
\end{tikzpicture}
    \end{adjustbox}
    \caption{Ethernet Frame}
    % \includegraphics[width=\textwidth]{images/theory/ethernet.png}
    % \caption{Ethernet Frame \cite{ethernet}}
    \label{fig:eth}
\end{figure}
The medium access controller is implemented in hardware to ensure that every bit
is received and stored. The transmit and receive data is commonly stored in
FiFo (first in first out) buffers. This way the next layer in the OSI model is
not required to have low latency capability.

\subsection{Network} 
The data link layer provides means to send frames across nodes in the same
network. As soon as the destination node is in another network, a network layer
is required. Using logical device addresses, network packets can be routed
across different networks and on different media. This allows data to be sent
over long distances.
\\

The most commonly used network layer is the Internet Protocol version 4 (IPv4).
It consists of a 20 byte sized header that contains destination and source IP
addresses, total length, checksum and other fields. The protocol field indicates
what layer four protocol is used. Figure \ref{fig:ip} shows a complete IPv4
header.

\clearpage
\begin{figure}[tb!]
    \centering
    \includegraphics[width=\textwidth]{images/theory/ip.png}
    \caption{IPv4 header \cite{ip}}
    \label{fig:ip}
\end{figure}
\begin{figure}[tb!]
    \centering
    \includegraphics[width=\textwidth]{images/theory/tcp.png}
    \caption{TCP Header \cite{tcpudp}}
    \label{fig:tcp}
\end{figure}
\begin{figure}[tb!]
    \centering
    \includegraphics[width=\textwidth]{images/theory/udp.png}
    \caption{UDP Header \cite{tcpudp}}
    \label{fig:udp}
\end{figure}
\clearpage

\subsection{Transport} 
The transport layer is the first layer in the OSI model that is not required by
the network. Its main purpose is to control the communication of different
applications on two hosts. Therefore a port number is required to distinguish
between the different applications utilizing the same network connection. The
transport layer may also provide segmentation of the data, guarantee of delivery
and flow control to avoid network jam.
\\

The two most used transport layer protocols are the Transmission Control
Protocol (TCP) and the User Datagram Protocol (UDP). Table \ref{tab:tcpudp}
shows the main differences between these two protocols. The most important
aspect is the protocol connection setup. While UDP is connection less (data is
sent without setup), TCP establishes a connection between host and client
prior to data transmission. This ensures a reliable data delivery hence all
messages are acknowledged. UDP has its benefits in lower overhead and for that
reason has slightly higher transmission speed but the protocol does not
guarantee
that the message has been received by the client.

\begin{table}[h]
    \centering
    \begin{tabular}{ l  c  c }
        \toprule
         & \textbf{TCP} & \textbf{UDP} \\
        \midrule
        Connection oriented & Yes & No \\
        Header size & 20 Byte & 8 Byte \\
        Reliable transmission & Yes & No \\
        Acknowledge & Yes & No \\
        Segmentation & Yes & No \\
        Best for & reliable transfer & fast transfer  \\
        \bottomrule
    \end{tabular}
    \caption{TCP vs. UDP}
    \label{tab:tcpudp}
\end{table}

% ==============================================================================
%
%                             FPGA
%
% ==============================================================================
% \section{FPGA}


% \subsection{Available Resources}

% \subsection{Criteria}
% Area, Speed, latency, throughput


% ==============================================================================
%
%                             FPGA Bus
%
% ==============================================================================
% \section{FPGA Bus}
% \subsection{AXI4}



% ------------------------------------------------------------------------->>> %

% <<< ---------------------------------------------------------------- MISSION %
% ==============================================================================
%
%                             Mission
%
% ==============================================================================
\chapter{Mission} \label{chapt:mission}
The following sections will
first cover the starting point based on appendix \ref{app:aufgabenstellung} and the available resources (appendix \ref{app:technicial_requirements}). Different possible
solutions will be covered in \ref{chapt:solutions} before presenting the concept
in \ref{chapt:mission:concept}.
\\

In a world of self driving cars and virtual reality, having a digital copy of
the real world yields several benefits. Cars can be trained in a virtual city
to increase the performance of their algorithms and video games could get more
realistic if the player could walk through the streets of a major city. Gathering
this data is one problem and processing the images is another. Pictures have
to be converted, analyzed and processed. This requires a great deal of
computational power for a task that is repeated several times.

% ==============================================================================
%
%                             Starting Point
%
% ==============================================================================
\section{Starting Point}

% ==============================================================================
%
%                             Possible Solutions
%
% ==============================================================================
\section{Possible Solutions} \label{chapt:solutions}


% ==============================================================================
%
%                             Concept
%
% ==============================================================================
\clearpage

% ------------------------------------------------------------------------->>> %

% <<< ------------------------------------------------------- IMAGE PROCESSING %
% ==============================================================================
%
%                             Image Processing
%
% ==============================================================================
\chapter{Image Processing}  \label{chapt:image_processing}

\section{Concept}
The figure ?????? shows the concept of the C code programmed in the Vivado HLS. First, the mean and the variance must be calculated so that the Wallis filter can be applied afterwards with the parameters. \\
The sequence of the code is explained in the figure ????. This consists of an initialization and a iteration. During the initialization, the complete neighborhood is read in and the mean pixel is calculated. To calculate the next pixels, only one new column is read in. This step is the so called iteration.\\
The code is row based. This means that each new row of an image have the initialization procedure.


\section{Implementation}

\subsection{Mean \& Variance}

\subsection{Fixed Point}

\subsection{Throughput Optimization}

% ------------------------------------------------------------------------->>> %

% <<< ------------------------------------------------ COMMUNICATION & CONTROL %
% ==============================================================================
%
%                             Dataflow
%
% ==============================================================================
\chapter{Dataflow} \label{chapt:dataflow}
With the Wallis filter implemented on the FPGA the problem occurs that the
image data has to be sent from the computer to the FPGA and back. The following
chapter explains the realization of said dataflow. It is split into two main
parts: the communcation and control parts. But before diving into them the
concept of the dataflow is briefly described in the following chapter.\\

During the work on the project it has been discovered that the initial approach
to
the problem will not lead to the optimum solution. This is why a second
implementation was made that differs in both communication and control parts. To
prevent confusion these two solutions are referred to as solution A (the first
approach) and solution B (the second, more performant approach).

% ==============================================================================
%                             Concept
% ==============================================================================
\section{Concept} \label{ch:concept}
As seen in the introduction the image data is sent to the FPGA over Ethernet.
This means that on the FPGA side an IP stack has to be implemented to handle
Ethernet
communication. The received data is sent to the Wallis filter for
processing and its results are sent back to the host PC. To distinguish
the
data transmission from the dataflow inside the FPGA, the dataflow was split into
two blocks: the communication and control block. The communication block handles
the Ethernet communication while the control block feeds the data in the right
order to the Wallis filter and does housekeeping work. Figure \ref{fig:dataflowa}
shows the dataflow for solution A.
\clearpage
Image data received from the UFT core is stored directly into FPGA block memory
through AXI4 memory mapped interfaces. The UFT core signals a complete file
transfer by asserting \texttt{rx\_done}. This is when the controller starts
reading the data from memory and sending it through AXI4-Stream to the Wallis
filter. The processed data is again stored in block memory. If all data has been
processed, the controller configures the UFT core with the amount of data to
send back and starts the transmission.

\begin{figure}[t!]
    \centering
    \begin{adjustbox}{max width=\textwidth}
        % \tikzsetnextfilename{system-overview}

%-----ABoxes
%-----#1 height, #2 width, #3 aspect, #4 name of the node, #5
%-----coordinate, #6 label
\def\memory[#1,#2,#3,#4,#5]#6{%
  \node[draw, cylinder, alias=cyl, shape border rotate=90, aspect=#3, %
  minimum height=#1, minimum width=#2, outer sep=-0.5\pgflinewidth, %
  color=white!40!black, left color=white!70, right color=white!80, middle
  color=white] (#4) at #5 {};%
  \node at #5 {#6};%
  \fill [white!30] let \p1 = ($(cyl.before top)!0.5!(cyl.after top)$), \p2 =
  (cyl.top), \p3 = (cyl.before top), \n1={veclen(\x3-\x1,\y3-\y1)},
  \n2={veclen(\x2-\x1,\y2-\y1)} in (\p1) ellipse (\n1 and \n2); }

\begin{tikzpicture}[
    rounded corners=0mm,
]
    %coordinates
    \coordinate (ccom)       at (0,0);
    \coordinate (cbram)       at (6,0);
    \coordinate (cip)       at (3,-2);


    %nodes

    \begin{pgfonlayer}{main}

        % Blocks
        \memory[45,40,1.6,bram,(cbram)] {BRAM};

        \node[draw, fill=white, minimum width=3.5cm, minimum height=1cm, anchor=west, text width=2.8cm, align=center] (com) at (ccom) {UFT core};

        \node[draw, fill=white, minimum width=3.5cm, minimum height=1cm, anchor=west, text width=2.8cm, align=center, below = 1cm of bram] (control)  {Controller};

        \node[draw, fill=white, minimum width=3cm, minimum height=1cm, anchor=west, text width=2.8cm, align=center, right = 2cm of control] (ip)  {Image\\Processing};
        
        % Paths
        % UFT to BRAM
        \path[draw,{Latex[length=2.5mm]}-{Latex[length=2.5mm]}] 
            ($(ccom.0) + (3.5,0)$) -- ($(cbram.180) + (-0.7,0.0)$) 
            node [midway, above] () {AXI} ;
        % control to BRAM
        \path[draw,{Latex[length=2.5mm]}-{Latex[length=2.5mm]}] 
            ($(control.0) + (-1.75,0.5)$) -- ($(cbram.180) + (0,-0.7)$) 
            node [midway, right] () {AXI} ;
        % control to ip
        \path[draw,-{Latex[length=2.5mm]}] 
            ($(control.0) + (0,0.3)$) -- ($(ip.180) + (0,0.3)$) 
            node [midway, above] () {Stream} ;
        \path[draw,-{Latex[length=2.5mm]}] 
            ($(ip.180) + (0,-0.3)$) -- ($(control.0) + (0,-0.3)$)
            node [midway, above] () {Stream} ;
        % Control paths
        \path[draw,-{Latex[length=2.5mm]}] 
            ($(control.0) + (-3.5,0)$) -| ($(ccom.0) + (1.75,-0.5)$) 
            node [near start, above] () {control} ;
        \path[draw,-{Latex[length=2.5mm]}] 
            ($(control.0) + (-1.75,-0.5)$) |- ($(control.0) + (0,-1)$) node [near end, above] () {control} -| ($(ip.0) + (-1.75,-0.5)$) 
             ;
        
        % \path[draw,{Latex[length=2.5mm]}-] ($(mon.0) + (0,-0.2)$) -- ($(com.180) + (0,-0.2)$) node[near end, below] () {4.} ;

        % \path[draw,-{Latex[length=2.5mm]}] ($(com.0) + (0,0.2)$) -- ($(ip.180) + (0,0.2)$) node[midway, above] () {2.} ;
        % \path[draw,{Latex[length=2.5mm]}-] ($(com.0) + (0,-0.2)$) -- ($(ip.180) + (0,-0.2)$) node[midway, below] () {3.} ;

    
    \end{pgfonlayer}

    % FPGA box
    \begin{pgfonlayer}{main}
        % \node[above = 0.2cm of com, xshift=-1.5cm] (fpga) { FPGA };
    \end{pgfonlayer}
    \begin{pgfonlayer}{foreground}
        % \node (f_fpga) [draw=black, fill=gray!20, inner sep=20, fit={(com) (ip) }] {};
    \end{pgfonlayer} 

    

\end{tikzpicture}
    \end{adjustbox}
    \caption{Dataflow inside FPGA for solution A}
    \label{fig:dataflowa}
\end{figure}

Following the path of dataflow it is easy to see that the data is stored
multiple
times: in the receiving FiFo of the UFT core, the block memory, then by the
controller in the transmitting and receiving path, again in block memory and in
a Tx FiFo of the UFT transmitter. This causes latency and high ressource usage.
Because of that, in a second approach (solution B) the datapath was chosen more
directly. Key elements are the AXI4-Stream interfaces. Figure \ref{fig:dataflowb}
shows the new approach using said stream interfaces. The UFT stack outputs
received data comming from the Ethernet line with low latency. The controller
then only buffers the necessary data and starts streaming pixel data to the
Wallis filter as soon as enough data is received. Processed data is buffered in
the controller until the transmitter is ready to send data back.

\begin{figure}[h!]
    \centering
    \begin{adjustbox}{max width=\textwidth}
        \input{images/controller/dataflowb.tikz}
    \end{adjustbox}
    \caption{Dataflow inside FPGA for solution B}
    \label{fig:dataflowb}
\end{figure}

\pagebreak

% ==============================================================================
%                             Communication
% ==============================================================================
\section{Communication}
The UFT core from project 5 serves as a starting point to the communication
problem. Altough it works on a fundamental basis, the following key features are
missing:

\begin{itemize}
	\item Acknowledgment of received data
	\item Retransmission if a packet was not received
	\item Control interface
	\item Stream based data interface
\end{itemize}

In the following chapters these four problems are addressed and the solutions
explained. To read more about the communication core as it was used from the
preceeding project the p5 project report can be consulted \cite{p5report}.
Furthermore the UDP File Transfer (UFT) Protocol Specifications (appendix 
\ref{app:uftspec}) and calculations (appendix \ref{app:uftcalc}) give
information about the communication protocol.

% ==============================================================================
%                             Acknowledge
% ==============================================================================
\subsection{Acknowledge}
The communication is based on the User Datagram Protocol (UDP). This protocol
features low overhead and is simple to implement. The main problem is that the
data sent is not acknowledged hence the sender has no feedback wheather the data
was received or not. This is one reason a custom session protocol (\gls{uft})
was introduced in the preceeding project. It provides command packets for data
packet acknowledgment. Table \ref{tab:uftcommandlist} lists the UFT commands.


\begin{table}[b!]
    \centering
    \begin{tabular}{l l l l l}
        \toprule
        Command & Short & Name & Data 1 & Data 2 \\
        \midrule
        0x00 & FTS & 
        File transfer start & TCID & NSEQ
        \\
        0x01 & FTP &
        File transfer stop & TCID & 0x0000 0000
        \\
        0x02 & ACKFP &
        Acknowledge data packet & TCID & SEQNBR
        \\
        0x03 & ACKFT &
        Acknowledge file transfer & TCID & 0x0000 0000
        \\
        \bottomrule
    \end{tabular}
    \caption{UFT command list}
    \label{tab:uftcommandlist}
\end{table}

The UFT core was extended with the functionality to send acknowledges back to
the PC for data packets and file transfers. For this to work, new connections
inside the UFT block were made. Drawn red in figure \ref{fig:ufttop}, the 
\texttt{uft\_rx\_mem\_ctl} block signals the \texttt{uft\_tx\_ctl} block to send
an acknowledgment. The connection consists of the signals listed in table 
\ref{tab:acksignals}. The tx control block latches the two request signals 
\texttt{ack\_cmd\_nseq} and \texttt{ack\_cmd\_ft} in the case a transmission is
running and an acknowledge request is made. If the tx state machine enters idle
or wait state it checks these latches for an acknowledge request. If a request
is latched, the tx command assembler and tx arbiter are turned on to send an
acknowledge packet. After the packet was sent, the according 
\texttt{ack\_cmd\_nseq\_done}/\texttt{ack\_cmd\_ft\_done} signals are asserted.


\begin{figure}[t!]
    \centering
    \begin{adjustbox}{max width=\textwidth}
        \input{images/dataflow/ufttop.tikz}
    \end{adjustbox}
    \caption{UFT Top Block Design with acknowledge path}
    \label{fig:ufttop}
\end{figure}


\begin{table}[tb!]
    \centering
    \begin{tabular}{l l l p{8.5cm}}
        \toprule
        Signal & in/out & width & Description \\
        \midrule
        \texttt{ack\_cmd\_nseq} & out & 1 &
        Command to send a data packet acknowledge packet
        \\
        \texttt{ack\_cmd\_ft} & out &1 &
        Command to send a file transfer acknowledge packet
        \\
        \texttt{ack\_cmd\_nseq\_done} & in &1 &
        Asserted if the data packet was acknowledged
        \\
        \texttt{ack\_cmd\_ft\_done} & in &1 &
        Asserted if the file transfer was acknowledged
        \\
        \texttt{ack\_seqnbr} & out & 24 &
        What sequence number to acknowledge
        \\
        \texttt{ack\_tcid} & out & 7 &
        What transaction id to acknowledge
        \\
        \texttt{ack\_dst\_port} & out & 16 &
        Destination port of the host to send the acknowledge to
        \\
        \texttt{ack\_dst\_ip} & out & 32 &
        Destination IP address of the host to send the acknowledge to
        \\
        \bottomrule
    \end{tabular}
    \caption{ACK signals}
    \label{tab:acksignals}
\end{table}

% ==============================================================================
%                             Retransmission
% ==============================================================================
\subsection{Retransmission}
Now that the FPGA sends an acknowledge packet for each data packet, the sending
PC can check if all packets were received by the FPGA. Therefore the PC software
was extended with retransmission. Before sending data an array is allocated that
will hold the information if a packet was received. Each element represents a
data packet. Listing \ref{lst:ackbufalloc} shows the allocation of the
acknowledge buffer.

\begin{minipage}{\linewidth}
    \begin{lstlisting}[
        style=CStyle, 
        caption=ack buffer allocation, 
        label=lst:ackbufalloc
        ]
uint8_t *ack_buf = (uint8_t*)malloc( nseq * sizeof(uint8_t) );
memset(ack_buf, 0, nseq * sizeof(uint8_t));\end{lstlisting}
\end{minipage}

\pagebreak
During transmission, the sender checks the receiving buffer for data. If an
acknowledge packet has been received, the acknowledge array is updated.
This allows the sending routine to retransmit packets that were not
acknowledged. Listing \ref{lst:ackupdate} shows the acknowledge buffer update
after packet is received.

\begin{minipage}{\linewidth}
    \begin{lstlisting}[
        style=CStyle, 
        caption=ack update, 
        label=lst:ackupdate
        ]
Recv(sockfd, buf, 1500, 0);
if(get_command(buf) == CONTROLL_ACKFP)
{
    ack_buf[get_command_ackfp_seqnbr(buf)] = 1;
}\end{lstlisting}
\end{minipage}


% ==============================================================================
%                             Control Interface
% ==============================================================================
\subsection{Control Interface}
The data received by the UFT core in solution A is stored in memory and data to
be sent back
is also located in FPGA memory. To instruct the core to send data, a control
interface is required. Tx base address, data size and rx base address are values
that will be sent from the controller to the UFT core as described later in
chapter \ref{ch:control}. For this purpose an AXI4-Lite slave interface was
realized. AXI4-Lite is a stripped-down version of AXI4. It allows single beat,
memory mapped read and write access from master to slave \cite{axispecs}.

Using Vivados \texttt{Create and Package New IP} command, a 16 register
AXI4-Lite slave interface was generated and connected with the according control
signals of the UFT core. Table \ref{tab:uftaxiregmap} shows the register map.
Registers 8 through 15 can be written by the PC using a custom UFT command
packet. This allows parameters to be sent from PC to FPGA.

\begin{table}[tb!]
    \centering
    \begin{tabular}{l l l p{10cm}}
        \toprule
		Nr & R/W & Offset [hex] & Description \\
        \midrule
		0 & RO & 00 & Status register. Bit 0: \texttt{tx\_ready}. Set if transmitter is
		ready to send data. \\
		1 & WO & 04 & Control register. Bit 0: \texttt{tx\_start}. Set 1 to start
		transmitter. \\
		2 & WO & 08 & Receiver base address. Received data is stored starting from this
		address. \\
		3 & WO & 0C & Transmitter base address. Data to send is read from this address.
		\\
		4 & RO & 10 & Rx counter. Counts how many file transfers were received. \\
		5 & WO & 14 & Tx size. How many bytes to send. \\
		6 & & 18 &  Not used\\
		7 & & 1C &  Not used\\
		8 & RO & 20 & User register 0 \\
		9 & RO & 24 & User register 1 \\
		10 & RO & 28 & User register 2 \\
		11 & RO & 2C & User register 3 \\
		12 & RO & 30 & User register 4 \\
		13 & RO & 34 & User register 5 \\
		14 & RO & 38 & User register 6 \\
		15 & RO & 3C & User register 7 \\
        \bottomrule
    \end{tabular}
    \caption{UFT core register map. RO=read only, WO=write only}
    \label{tab:uftaxiregmap}
\end{table}

This new control interface allows the control and status signals from the UFT
core to be read and written using a single normed interface. Figure 
\ref{fig:uftcoreaxilite} shows the complete UFT IP core with AXI4-Lite control
interface, stream and control interface to the UDP core and AXI master burst
interface for memory access (refer to \cite{p5report} for AXI master burst).

\begin{figure}[b!]
    \centering
    \includegraphics[width=0.45\textwidth] {images/dataflow/uftcoreaxilite.png}
    \caption{UFT core with AXI4-Lite interface}
    \label{fig:uftcoreaxilite}
\end{figure}

% ==============================================================================
%                             Data Interface
% ==============================================================================
\subsection{Data Interface} \label{ch:data:com:datainterface}
The UFT core was in the first place intended for file based data transfer from
PC to FPGA and vice versa. This led to the decision to use memory mapped AXI
interface to store the received data and read the data to be sent. It made
sense with the UFT protocol being file oriented. In the progress of developing
the Wallis filter, a stream based approach was pursued to reduce latency and
memory usage. With the UFT core being memory based, a controller had to be
introduced to send the data from memory to the Wallis filter. Early tests
foreshowed that the UFT core with its memory based interface would be the
limiting member in the data processing chain. So the core's data interface was
rewritten to support AXI4-Stream for data receiving and sending.

For solution B, the two \texttt{axi\_master\_burst} blocks were removed and the
code of the \texttt{uft\_rx\_mem\_ctl} and \texttt{uft\_tx\_data\_assembler} was 
changed to directly connect the AXI4-Stream from the UDP stack to the outside.
Advantage of this solution is the reduced latency and less memory usage. One
downside is that the receiver is no longer able to reorder packets if they do
not arrive in order. As long as the application runs on a closed network it can
be assumed that packets will not change order. The changes made for the
streaming interfaces are shown in figure \ref{fig:ufttopstream}.

\begin{figure}[b!]
    \centering
    \begin{adjustbox}{max width=\textwidth}
        % \tikzsetnextfilename{system-overview}
\begin{tikzpicture}[
    rounded corners=0mm,
    entity/.style={
        draw,
        minimum height=1.0cm,
        minimum width=3cm,
        fill=white,
        anchor=north west,
    },
    entityold/.style={
        draw=gray!60,
        minimum height=1.0cm,
        minimum width=3cm,
        fill=gray!20,
        anchor=north west,
    },
]
    %coordinates
    \coordinate (orig)      at (0,0);
    \coordinate (crx)       at (0,0);
    \coordinate (crxmem)    at (5,0);

    \coordinate (ctxctl)    at (0,-2.5);
    \coordinate (ctxcmd)    at (5,-3.5);
    \coordinate (ctxdat)    at (5,-5.5);
    \coordinate (ctxarb)    at (10,-4.5);
    \coordinate (caxilite)  at (-5,-2.5);

    \coordinate (ctxamb)    at (0,-5.5);
    \coordinate (crxamb)    at (10,0);
    %nodes

    \begin{pgfonlayer}{main}
        % entities
        \node[entity, label={uft\_rx}] (rx) at (crx) {};
        \node[entity, label={[name=rxl] uft\_rx\_mem\_ctl}] (rxmem) at (crxmem) {};

        \node[entity, label={[name=ltxctl] uft\_tx\_ctl}] (txctl) at (ctxctl) {};
        \node[entity, label={[name=txcal]uft\_tx\_cmd\_assembler}] (txcmd) at (ctxcmd) {};
        \node[entity, label={uft\_tx\_data\_assembler}] (txdat) at (ctxdat) {};
        \node[entity, label={uft\_tx\_arbiter}] (txarb) at (ctxarb) {};

        \node[entityold, label={axi\_master\_burst}] (ambrx) at (crxamb) {};
        \node[entityold, label={axi\_master\_burst}] (ambtx) at (ctxamb) {};

        % ports
        \path[draw,{Latex[length=2.5mm]}-] ($(rx.180) + (0,1/6)$) -- ($(rx.180) + (-1.0,1/6)$) node[anchor=east] {rx\_hdr};
        \path[draw=blue,line width=0.5mm,{Latex[length=2.5mm]}-] ($(rx.180) + (0,-1/6)$) -- ($(rx.180) + (-1.0,-1/6)$) node[anchor=east] {s\_axi};
        \path[draw,{Latex[length=2.5mm]}-] ($(txctl.180) + (0,0)$) -- ($(txctl.180) + (-1.0,0)$) node[anchor=east] {controll};

        \path[draw=blue,line width=0.5mm,-{Latex[length=2.5mm]}] ($(rxmem.0) + (0,0)$) -- ($(rxmem.0) + (6.5,0/6)$) node[anchor=west] {s\_axi\_rx};
        \path[draw,-{Latex[length=2.5mm]}] ($(txctl.0) + (0,3.5/10)$) -- ($(txctl.0) + (11.5,3.5/10)$) node[anchor=west] {tx\_hdr};
        \path[draw=red,line width=0.5mm,-{Latex[length=2.5mm]}] ($(txarb.0) + (0,0/10)$) -- ($(txarb.0) + (1.5,0/10)$) node[anchor=west] {s\_axi};

        % Interconnects
        \path[draw,-{Latex[length=2.5mm]}] ($(rx.0) + (0,1/6)$) -- ($(rxmem.180) + (0,1/6)$) node[anchor=east] {};
        \path[draw=blue,line width=0.5mm,-{Latex[length=2.5mm]}] ($(rx.0) + (0,-1/6)$) -- ($(rxmem.180) + (0,-1/6)$) node[midway, anchor=north] {s\_axi};
        % \node at ($(rx.180) + (-0.5,-1/6)$) [circle,fill,inner sep=1.5pt]{};

        \path[draw,-{Latex[length=2.5mm]}] ($(txctl.0) + (0,1.5/10)$) -| ($(txcmd.180) + (-0.5,0)$) -- ($(txcmd.180) + (0,0)$) node[anchor=west] {};
        \path[draw,-{Latex[length=2.5mm]}] ($(txctl.0) + (0,-1.5/10)$) -| ($(txarb.180) + (-6,0)$) -- ($(txarb.180) + (0,0)$) node[anchor=west] {};
        \path[draw,-{Latex[length=2.5mm]}] ($(txctl.0) + (0,-3.5/10)$) -| ($(txdat.180) + (-1.5,1/6)$) -- ($(txdat.180) + (0,1/6)$) node[anchor=west] {};

        \path[draw,-{Latex[length=2.5mm]}] ($(txcmd.0) + (0,0)$) -| node[anchor=south] {s\_axi} ($(txarb.180) + (-0.5,1/4)$)  -- ($(txarb.180) + (0,1/4)$);
        \path[draw=red,line width=0.5mm,-{Latex[length=2.5mm]}] ($(txdat.0) + (0,0)$) -| node[anchor=north] {s\_axi}($(txarb.180) + (-0.5,-1/4)$) -- ($(txarb.180) + (0,-1/4)$);

        \path[draw=red,line width=0.5mm,-{Latex[length=2.5mm]}] ($(txdat.180) + (-6,-1/6)$) node[anchor=east] {s\_axi\_tx} -- ($(txdat.180) + (0,-1/6)$) node[anchor=east] {};

        % Ack
        \path[draw,-{Latex[length=2.5mm]}] ($(crxmem.0) + (1.5,-1)$) |- ($(crxmem.0) + (-1,-2)$) -| ($(ctxctl.0) + (2.5,0)$) node[midway, anchor=east] {};


    \end{pgfonlayer}

    % tx box
    \begin{pgfonlayer}{foreground}
        \node [draw, fill=gray!20, inner sep=10, fit={(ltxctl) (txctl) (txcmd) (txdat) (txarb) (txcal)}, label={[label distance=0.0cm]150:uft\_tx\_top}] (tx) {};
    \end{pgfonlayer} 

    % Board box
    \begin{pgfonlayer}{background}
        \node [draw, fill=gray!40, inner sep=10, fit={(tx) (rx) (rxmem) (rxl)}, label=uft\_top] (tx) {};
    \end{pgfonlayer} 

\end{tikzpicture}
    \end{adjustbox}
    \caption{UFT Top Block Design with AXI4-Stream interface}
    \label{fig:ufttopstream}
\end{figure}

% ==============================================================================
%                             Implemented Features
% ==============================================================================
\clearpage
\subsection{Implemented Features}
In the final version of the UFT core used in solution B the AXI4-Lite interface
was removed because no memory addresses had to be exchanged and the solution B
controller was written in VHDL. Adding an AXI4-Lite master interface would have
only made it more complicated. 
\\

To conclude the changes made to the UFT core from the version of project 5:
\begin{itemize}
  \item Added acknowledgment on receiver path
  \item Changed data interfaces from memory mapped to AXI4-Stream
  \item Added user register to exchange configuration data
  \item Bug fixes
\end{itemize}

There are still some features that are not yet implemented but are defined in
the UFT protocol specifications:
\begin{itemize}
  \item Acknowledge check on transmit
  \item Retransmission during send
\end{itemize}

Figure \ref{fig:uftipcoreaxistream} shows the IP core with the AXI4-Stream
ports.

\begin{figure}[h!]
    \centering
    \includegraphics[width=0.45\textwidth] {images/dataflow/uftcorestream.png}
    \caption{UFT IP core with AXI4-Stream interface}
    \label{fig:uftipcoreaxistream}
\end{figure}
% ==============================================================================
%                             Control
% ==============================================================================
\section{Control} \label{ch:control}
Now that the image data is received by the UFT core it has to be sent to the
Wallis filter in the right order. Reordering the pixel data is the main task of
the controller besides caching data and controlling the UFT and Wallis cores.
This chapter describes the tasks to be solved.

As already mentioned at the beginning of chapter \ref{chapt:dataflow}, two
implementations were realized. Solution A was implemented using Vivado HLS and
is documented in chapter \ref{ch:controller:hls} followed by solution B written
in VHDL and described in detail in chapter \ref{ch:controller:vhdl}. 
% The problem is
% stated in the following chapter.

% ==============================================================================
%                             Concept
% ==============================================================================
\subsection{Concept} \label{ch:control:concept}
Image data coming from the PC is sent row wise with the left most pixel sent
first. This is due to the fact that \gls{opencv} (which is used on PC
side to access
images) stores the image data in said layout \cite{opencv_structures}. The
Wallis filter however, requires its input data to be coloumn wise with the top
most pixel sent first as described in chapter \ref{ch:ip:concept}. Figure 
\ref{fig:memproblem} illustrates this issue. The blue ``write'' arrow shows the data
comming from the UFT core and the red ``read'' arrow represents the order the
data has to be sent to the Wallis filter. For illustration purposes a window
length of five and image width of eight is used.

\begin{figure}[h!]
    \centering
    \begin{adjustbox}{scale=0.7}
        % \tikzsetnextfilename{system-overview}
\begin{tikzpicture}[
    rounded corners=0mm,
    triangle/.style = {fill=blue!20, regular polygon, regular polygon sides=3 },
    node rotated/.style = {rotate=180},
    border rotated/.style = {shape border rotate=180}
]
    %coordinates
    \coordinate (orig)      at (0,0);

    \begin{pgfonlayer}{main}
        
        % Write arrows
        % \draw[draw=blue,line width=1.5mm] (8,4.5) .. controls (8,4) and (-1,4) .. (-1,3.5);
        % \path[draw=blue,line width=1.5mm] ($(-1,3.5)$) -- ($(8,3.5)$) node[anchor=east] {};

        % Write path
        \path[draw={rgb:red,1;green,2;blue,3},line width=1.0mm] ($(-2,4.5)$)  -- ($(9,4.5)$);
        \path[draw={rgb:red,1;green,2;blue,3},line width=1.0mm]  (-1,3.5) -- ($(9,3.5)$) ;
        \path[draw={rgb:red,1;green,2;blue,3},line width=1.0mm]  (-1,2.5) -- ($(9,2.5)$) ;
        \path[draw={rgb:red,1;green,2;blue,3},line width=1.0mm]  (-1,1.5) -- ($(9,1.5)$) ;
        \path[draw={rgb:red,1;green,2;blue,3},line width=1.0mm]  (-1,0.5) -- ($(9,0.5)$);
        \path[draw={rgb:red,1;green,2;blue,3},line width=1.0mm,dashed] ($(9,4.5)$)  .. controls (9,4) and (-1,4) .. (-1,3.5);
        \path[draw={rgb:red,1;green,2;blue,3},line width=1.0mm,dashed] ($(9,3.5)$)  .. controls (9,3) and (-1,3) .. (-1,2.5);
        \path[draw={rgb:red,1;green,2;blue,3},line width=1.0mm,dashed] ($(9,2.5)$)  .. controls (9,2) and (-1,2) .. (-1,1.5);
        \path[draw={rgb:red,1;green,2;blue,3},line width=1.0mm,dashed] ($(9,1.5)$)  .. controls (9,1) and (-1,1) .. (-1,0.5);
        % Write triangles
        \node[triangle,shape border rotate=270, fill={rgb:red,1;green,2;blue,3},minimum size=0.1cm] at (-1,0.5) {};
        \node[triangle,shape border rotate=270, fill={rgb:red,1;green,2;blue,3},minimum size=0.1cm] at (-1,1.5) {};
        \node[triangle,shape border rotate=270, fill={rgb:red,1;green,2;blue,3},minimum size=0.1cm] at (-1,2.5) {};
        \node[triangle,shape border rotate=270, fill={rgb:red,1;green,2;blue,3},minimum size=0.1cm] at (-1,3.5) {};
        \node[triangle,shape border rotate=270, fill={rgb:red,1;green,2;blue,3},minimum size=0.1cm] at (-1,4.5) {};
        
        % Read path
        \path[draw={rgb:red,3;green,1;blue,2},line width=1.0mm]  (0.5,6)  -- (0.5,-0.5);
        \path[draw={rgb:red,3;green,1;blue,2},line width=1.0mm]  (1.5,5.5)  -- (1.5,-0.5);
        \path[draw={rgb:red,3;green,1;blue,2},line width=1.0mm]  (2.5,5.5)  -- (2.5,-0.5);
        \path[draw={rgb:red,3;green,1;blue,2},line width=1.0mm]  (3.5,5.5)  -- (3.5,-0.5);
        \path[draw={rgb:red,3;green,1;blue,2},line width=1.0mm]  (4.5,5.5)  -- (4.5,-0.5);

        \path[draw={rgb:red,3;green,1;blue,2},line width=1.0mm,dashed] (0.5,-0.5)  .. controls (1,-0.5) and (1,5.5) .. (1.5,5.5);
        \path[draw={rgb:red,3;green,1;blue,2},line width=1.0mm,dashed] (1.5,-0.5)  .. controls (2,-0.5) and (2,5.5) .. (2.5,5.5);
        \path[draw={rgb:red,3;green,1;blue,2},line width=1.0mm,dashed] (2.5,-0.5)  .. controls (3,-0.5) and (3,5.5) .. (3.5,5.5);
        \path[draw={rgb:red,3;green,1;blue,2},line width=1.0mm,dashed] (3.5,-0.5)  .. controls (4,-0.5) and (4,5.5) .. (4.5,5.5);
        % Read triangles
        \node[triangle, border rotated, fill={rgb:red,3;green,1;blue,2},minimum size=0.1cm] at (0.5,5.5) {};
        \node[triangle, border rotated, fill={rgb:red,3;green,1;blue,2},minimum size=0.1cm] at (1.5,5.5) {};
        \node[triangle, border rotated, fill={rgb:red,3;green,1;blue,2},minimum size=0.1cm] at (2.5,5.5) {};
        \node[triangle, border rotated, fill={rgb:red,3;green,1;blue,2},minimum size=0.1cm] at (3.5,5.5) {};
        \node[triangle, border rotated, fill={rgb:red,3;green,1;blue,2},minimum size=0.1cm] at (4.5,5.5) {};

        % Text
        \node[] (write) at (-2,5) {Write};
        \node[] (read) at (0,6.2) {Read};

        % Braces
        \draw [line width=0.5mm,decorate,decoration={brace,amplitude=10pt},xshift=-4pt,yshift=0pt] (9.5,5) -- (9.5,0) node [black,midway,xshift=0.5cm,anchor=west] {Window length};
        \draw [line width=0.5mm,decorate,decoration={brace,amplitude=10pt},xshift=-0pt,yshift=0pt] (8,-0.5) -- (0,-0.5) node [black,midway,yshift=-0.5cm,anchor=north] {Image width};
        
        % Center pixel
        \draw[black,line width=0.5mm] (2,2) rectangle (3,3);
        % Window size
        \draw[black,line width=0.5mm] (0,0) rectangle (5,5);
        % Axis
        \foreach \x in {0,1,2,3,4}
            \node[anchor=north] at ($(-0.5,5)-(0,\x)$)  {$\x$};

    \end{pgfonlayer}

    % Foreground
    \begin{pgfonlayer}{foreground}
        
    \end{pgfonlayer} 

    % Background
    \begin{pgfonlayer}{background}
        % Grid
        \draw[step=1cm,gray,very thin] (0,0) grid (8,5);
    \end{pgfonlayer} 

\end{tikzpicture}
    \end{adjustbox}
    \caption{Memory reordering problem}
    \label{fig:memproblem}
\end{figure}
\clearpage
This is only one part of the problem. Another issue arises if the
progression of the neighborhood across the input image is observed. We
distinguish two scenarios: A) moving the neighborhood horizontally to the right
and B) advancing the neighborhood to the next row vertically. The first
scenario is illustrated in figure \ref{fig:memproblemgrowthx}. The green arrow
marks the new coloumn of the neighborhood to be sent to the Wallis filter.

\begin{figure}[t!]
    \centering
    \begin{adjustbox}{scale=0.7}
        \input{images/controller/memproblemgrowthx.tikz}
    \end{adjustbox}
    \caption{Scenario A) Data growth in horizontal direction}
    \label{fig:memproblemgrowthx}
\end{figure}

This progression is made for each coloumn of the input image until the last
coloumn was processed. Then scenario B) comes to play. Moving the neighborhood
vertically to the next row by one pixel requires new image data of one row
and $WINDOW\_LENGTH-1$ rows of image data that had already been processed on the
previous row. Figure \ref{fig:memproblemgrowthy} illustrates scenario B.

\begin{figure}[h!]
    \centering
    \begin{adjustbox}{scale=0.7}
        \input{images/controller/memproblemgrowthy.tikz}
    \end{adjustbox}
    \caption{Scenario B) Data growth in vertical direction}
    \label{fig:memproblemgrowthy}
\end{figure}

\pagebreak

From these two scenarios we can conclude two problems the controller has to
solve:
\begin{itemize}
    \item Send the image data in the right order (coloumn wise with
    $WINDOW\_LENGTH$ sized coloumns)
    \item Cache $WINDOW\_LENGTH-1$ rows for the computation of the next image
    row
\end{itemize}

In addition the controller has to start the UFT transmission if an image row has
been processed and configure the Wallis filter with the parameters comming from
the UFT core.



% ==============================================================================
%                             HLS
% ==============================================================================
\subsection{Implementation HLS} \label{ch:controller:hls}
Now that the problems are analysed, the proceeding can be stated. For the first
solution an approach using Vivado HLS was picked. The main motivations for this
decision were the fact that we had learned how fast we can have a working IP
core
using the Vivado HLS workflow and that we wanted to test the ability to
implement a state machine in Vivado HLS. In the following chapters the
requirements for this state machine are listed, a brief insight into the source
code is given and the main hurdles while developing the core are unfolded.

\subsubsection*{Requirements}
Besides the dataflow described in \ref{ch:control:concept} the IP core
interfaces must be defined. Table \ref{tab:controlleraports} lists the IP
interfaces as defined in \texttt{controller.cpp}.
The IP core implements a finite state machine as seen in figure 
\ref{fig:controllerfsm}. After the initial state
\texttt{INIT} where the UFT core is initialized, the state machine goes to its
idle state \texttt{IDLE}. If the UFT core indicates the end of a file
transfer the image width is stored and the state machine switches to the 
\texttt{READ} state. There it fills the first junk of its buffers and
switches to the \texttt{STREAM} state. Now the data is sent from the
internal buffers to the Wallis filter in the correct order. Simultaneously the
processed pixels from the Wallis filter are stored inside a memory. 

\begin{table}[b!]
    \centering
    \begin{tabular}{l l l p{8cm}}
        \toprule
        Variable & Type & Connection to & Description \\
        \midrule
        \texttt{memp} & \texttt{AXI4 Master} & Memory &
        Access to the block memory where the image data is stored
        \\
        \texttt{cbus} & \texttt{AXI4 Master} \footnotemark & UFT core &
        Control the AXI4-Lite slave registers to control the UFT core
        \\
        \texttt{inData} & \texttt{AXI4-Stream} & Wallis filter &
        Processed pixels comming from the Wallis filter
        \\
        \texttt{outData} & \texttt{AXI4-Stream} & Wallis filter &
        Image data sent to the Wallis filter for processing
        \\
        \texttt{rx\_done} & \texttt{ap\_uint<1>} & UFT core &
        Is asserted after a file transfer is complete
        \\
        \texttt{tx\_ready} & \texttt{ap\_uint<1>} & UFT core &
        Is asserted if UFT core is ready to send
        \\
        \bottomrule
    \end{tabular}
    \caption{Controller solution A interface ports}
    \label{tab:controlleraports}
\end{table}
\footnotetext{Vivado HLS does not support AXI4-Lite master, but a AXI4 can
        be converted to AXI4-Lite}

\begin{figure}[t!]
    \centering
    \begin{adjustbox}{scale=1}
        % \tikzsetnextfilename{system-overview}
\begin{tikzpicture}[->,>=stealth',shorten >=1pt,auto,node distance=2.8cm,
                    semithick]

    \tikzstyle{every state}=[fill=white,draw=black,text=black,minimum width=2.2cm]
    
    % states
    \node[initial,state] (A)                    {INIT};
    \node[state]         (B) [right of=A]       {IDLE};
    \node[state]         (C) [above right of=B] {READ};
    \node[state]         (D) [right of=C]       {STREAM};
    \node[state]         (E) [below right of=D]       {WRITE};
    \node[state, align=center]         (F) [below left of=E]       {WAIT\_TO\_ \\ SEND};
    \node[state]         (G) [left of=F]       {SEND};


    % path
    \path   (A) edge              node {} (B)
            (B) edge              node {} (C)
            (C) edge              node {} (D)
            (D) edge              node {} (E)
            (E) edge              node {} (F)
            (F) edge              node {} (G)
            (G) edge              node {} (B);
    \begin{pgfonlayer}{main}

    \end{pgfonlayer}

    \begin{pgfonlayer}{foreground}
    
    \end{pgfonlayer} 


\end{tikzpicture}
    \end{adjustbox}
    \caption{Controller solution A) simplified state machine}
    \label{fig:controllerfsm}
\end{figure}

If all
pixels in one line are processed the state \texttt{WRITE} is activated. The
processed pixels are stored in block memory using the \texttt{memp} AXI4 Master
interface and the state machine switches to the \texttt{WAIT\_TO\_SEND}
state. If the UFT core Tx is ready, the state \texttt{SEND} is activated
where the UFT core is configured and a transmission is started.

\subsubsection*{Realization}
According to Xilinx application note XAPP-1209 \cite{xapp1209} a state
machine in Vivado HLS can be realized by using a switch statement in the C/C++
code.
The code in file \texttt{controller.cpp} implements the state machine behaviour.
The state machine code is very straight foreward. More work went into the memory
layout. Because the memory bus to read and write pixels to and from memory is
AXI4 based, multiple single byte access result in considerably less throughput
than single burst access.
To take advantage of these burst accesses, a ping-pong buffer structure was
realized. There are two buffers, each of the size of $WINDOW\_LENGTH *
AXI\_BURST\_SIZE$ bytes, where $AXI\_BURST\_SIZE$ represents the number of bytes
read
in one burst access. It requires $WINDOW\_LENGTH$ burst reads to fill one
buffer. If one buffer is filled, it can be accessed by single byte access to
read the data in the required order for the Wallis filter, as shown in figure
\ref{fig:memproblem}. During the time the data is read from one buffer, the
other buffer is filled with image data from the AXI4 bus, hence the name
ping-pong buffer. 

Figure \ref{fig:solamemlayout} shows the memory layout of
solution A. The blue and red rectangles represent the two buffers. Visualized is
an AXI burst length of eight and a window length of five.

\clearpage

\begin{figure}[tb!]
    \centering
    \begin{adjustbox}{scale=0.7}
        % \tikzsetnextfilename{system-overview}
\begin{tikzpicture}[
    rounded corners=0mm,
    triangle/.style = {fill=blue!20, regular polygon, regular polygon sides=3 },
    node rotated/.style = {rotate=180},
    border rotated/.style = {shape border rotate=180}
]
    %coordinates
    \coordinate (orig)      at (0,0);

    \begin{pgfonlayer}{main}
        
        % % Write path
        \path[draw={rgb:red,1;green,2;blue,3},-{Latex[length=5mm]},line width=1.0mm] (0.5,4.5)  -- (7.5,4.5);
        \path[draw={rgb:red,3;green,1;blue,2},-{Latex[length=5mm]},line width=1.0mm] (8.5,4.5)  -- (15.5,4.5);
        
        % % Text
        % \node[] (write) at (-2,5) {Write};

        % Braces
        \draw [line width=0.5mm,decorate,decoration={brace,amplitude=10pt},xshift=-4pt,yshift=0pt] (17.5,5) -- (17.5,0) node [black,midway,xshift=0.5cm,anchor=west] {Window length};
        \draw [line width=0.5mm,decorate,decoration={brace,amplitude=10pt},xshift=-0pt,yshift=0pt] (8,-0.5) -- (0,-0.5) node [black,midway,yshift=-0.5cm,anchor=north] {AXI burst length};
        
        % Center pixel
        \draw[black,line width=0.5mm] (2,2) rectangle (3,3);
        % Window size
        \draw[black,line width=0.5mm] (0,0) rectangle (5,5);
        % Buffer A
        \draw[draw={rgb:red,1;green,2;blue,3},line width=1mm] (-0.05,-0.05) rectangle (7.95,5.05);
        % Buffer B
        \draw[draw={rgb:red,3;green,1;blue,2},line width=1mm] (8.05,-0.05) rectangle (16.05,5.05);
        % Axis
        \foreach \x in {0,1,2,3,4}
            \node[anchor=north] at ($(-0.5,5)-(0,\x)$)  {$\x$};
        % Axis
        \foreach \x in {0,1,2,3,4,5,6,7,8,9,10,11,12,13,14,15,16}
            \node[anchor=south] at ($(0.1,5.1)+(\x,0)$)  {$\x$};

    \end{pgfonlayer}

    % Foreground
    \begin{pgfonlayer}{foreground}
        
    \end{pgfonlayer} 

    % Background
    \begin{pgfonlayer}{background}
        % Grid
        \draw[step=1cm,black,thin] (0,0) grid (17,5);
    \end{pgfonlayer} 

\end{tikzpicture}
    \end{adjustbox}
    \caption{Memory layout for solution A}
    \label{fig:solamemlayout}
\end{figure}

Filling the buffer is done in the \texttt{fillBuff} routine
as shown on listing \ref{lst:buf_fill}.

\begin{minipage}{\textwidth}
\begin{lstlisting}[style=CStyle, caption=Buffer fill simplified,
label=lst:buf_fill]
uint32_t i, inOff, outOff;
for(i = 0; i < WINDOW_LEN; i++) {
    inOff = off + i*imgWidth;
    outOff = i*AXI_BURST_SIZE;
    memcpy(&buf[outOff], &memp[inOff], AXI_BURST_SIZE);
}\end{lstlisting}
\end{minipage}

To send the right order of output pixels, two counters are required. The first
counter named \texttt{ms\_rctr} increments after every pixel sent and wraps at
$WINDOW\_LENGTH$. Therefore it counts the current coloumn. The second counter
named \texttt{ms\_cctr} increments every time the \texttt{ms\_rctr} wrappes
around and counts up to $IMAGE\_WIDTH$. From these two counters the address of
the pixel to be sent can be calculated as shown in listing \ref{lst:aaddrcalc}.

\begin{minipage}{\textwidth}
\begin{lstlisting}[style=CStyle, caption=Pixel send address calculation,
label=lst:aaddrcalc]
oPxl.data = outPpBuf[AXI_BURST_SIZE*(ms_rctr++) + ms_cctr];\end{lstlisting}
\end{minipage}

Receiving a pixel from the Wallis core is simply storing the pixel at the next
memory location of the
output buffer. Listing \ref{lst:runinstream} shows the receiveing of processed
pixels.

\begin{minipage}{\textwidth}
\begin{lstlisting}[style=CStyle, caption=Pixel read store,
label=lst:runinstream]
iPxl = inData.read();
out_mem[sm_ctr++] = (uint8_t)iPxl.data;\end{lstlisting}
\end{minipage}

Everytime a pixel is read or written a corresponding counter is incremented.
After both counter hit their compare value, the stream state is exited and the
data is copied back to memory.

\pagebreak
\subsubsection*{Hurdles}
The C/C++ code was written in little time. With less than 300 lines it
is a manageable amount of code.
The C/C++ and RTL-simulation yielded no errors but the
implemented core together with the Wallis filter core did not work as expected.
Thanks to the \gls{ila} the signals connecting the controller
and Wallis core could be inspected. Soon it was discovered that the controller
only output one pixel and then stopped. It was discovered that this was due to
the fact that a \texttt{read()} operation on a AXI4-Stream in Vivado HLS is
blocking, meaning that the synthesis tool implements the code in a  manner that
if no data is
valid on the stream, the code will wait until there is data on the stream. This
led to the code waiting for an input byte and no output was generated. This
issue was fixed by wrapping the read access with a query weather the input
stream holds data. Listing \ref{lst:runinstream2} shows the adjusted pixel read.
The same applied to the sending of data. It is first checked wheather the output
stream is not full before sending new data.

\begin{minipage}{\textwidth}
\begin{lstlisting}[style=CStyle, caption=Pixel read store with query,
label=lst:runinstream2]
if(!inData.empty()) {
    iPxl = inData.read();
    out_mem[sm_ctr++] = (uint8_t)iPxl.data;
}\end{lstlisting}
\end{minipage}

% \clearpage
\subsubsection*{Conclusion}
To conclude the findings using a Vivado HLS based approach, the thesis that
with little effort a working solution can be produced is confirmed. The required
interfaces (AXI4-Master and AXI4-Stream) were implemented with a single line of
code and the statemachine with a simple switch statement. After
clearing some hurdles the controller core worked as expected with two drawbacks:

\vspace{1ex}
\textbf{1. Throughput:} The controller core outputs data at a rate of one byte
every 9 clock cycles as can be seen in the result of the simulation.
In theory, using a block memory as internal buffer, it
should be possible to output one byte per clock cycle. This issue could not be
resolved in a forseeable time. The analysis view of Vivado HLS would give
indications on what calculation is resulting in a reduced throughput but the
analysis results could not be interpreted.

% \begin{figure}[tb!]
%     \centering
%     \includegraphics[width=0.6\textwidth]{images/controller/hlscontrolleroutput.png}
%     \caption{Controller solution A) output data}
%     \label{fig:shlowhlsoutput}
% \end{figure}

\vspace{1ex}
\textbf{2. Memory layout:} In solution A the image data is stored multiple time
as already mentioned in chapter \ref{ch:concept}. The data is stored twice in
the controller core alone: during read in the ping-pong buffer and during write
in the output buffer. This increases the ressource usage.

\vspace{1ex}
These two issues were the motivation to rethink the memory layout and
functionality of the controller core which led to solution B, a controller
implemented in VHDL.

% \subsubsection*{IF-Statement} \label{ch:data:if}

% \begin{minipage}{\textwidth}
% \begin{lstlisting}[style=CStyle, caption=Buffer switching reloading without else statement, label=lst:buf_false]
% if( (outPpBuf == ppBufA) && (ms_pctr < (inLineSize-PIN_PONG_BUF_SIZE)) ) {
%     if(!ppBufBrdy) {
%         ...
%     }
% }
% if( (outPpBuf == ppBufB) && (ms_pctr < (inLineSize-PIN_PONG_BUF_SIZE)) ) {
%     if(!ppBufArdy) {
%         ...
%     }
% }
% \end{lstlisting}
% \end{minipage}

% \begin{minipage}{\textwidth}
% \begin{lstlisting}[style=CStyle, caption=Buffer switching reloading with else statement, label=lst:buf_right]
% if (ms_pctr < (inLineSize-PIN_PONG_BUF_SIZE)) {
% 	if( (outPpBuf == ppBufA)  ) {
% 		if(!ppBufBrdy) {
% 			...
% 		}
% 	}
% 	else {
% 		if(!ppBufArdy) {
% 			...
% 		}
% 	}
% }
% \end{lstlisting}
% \end{minipage}

% ==============================================================================
%                             VHDL
% ==============================================================================
\clearpage
\subsection{Implementation VHDL} \label{ch:controller:vhdl}
In a new implementation of the controller, the two drawbacks of solution A are
addressed. The following chapter is structered likewise starting with the new
requirements, how the core was realized, the hurdles accross the way and a
conclusion of the performance of the new controller core.

The Wallis filter core was built from ground up with AXI4-Stream interfaces. A
slave interface for input data and a master interface for output data. This
stream based approach should be implemented in the controller as well to reduce
buffering to a minimum to decrease memory usage. The use of streaming interfaces
required a new memory structure. These two aspects form the requierements of the
new controller core.

\subsubsection*{Requirements}
With the new stream based approach the UFT communication core first had to be
altered to support stream interfaces. This is explained in chapter 
\ref{ch:data:com:datainterface}. With the AXI4-Lite configuration interface
removed, the
control signals are replaced with arbitrary signals. The new interface
definition is dissected in table \ref{tab:controllerbports}.

Furthermore the order of pixel to be sent to the Wallis filter remain the same
as described in chapter \ref{ch:control:concept}. A stream based approach also eliminates the need for a statemachine. The entire
controller is to be stateless.

\begin{table}[h!]
    \centering
    \begin{tabular}{l l l p{8cm}}
        \toprule
        Name & Type & To & Description \\
        \midrule
        %%% UFT Rx
        \texttt{uft\_i\_axis} & \texttt{AXI4-Stream} & UFT Rx &
        Receive data stream from UFT core
        \\

        \texttt{uft\_rx\_done} & \texttt{arb} & UFT Rx &
        Signals a complete UFT transfer
        \\  
        % \texttt{uft\_rx\_row\_num} & \texttt{arb} & UFT Rx &
        % asdf
        % \\  
        % \texttt{uft\_rx\_row\_num\_valid} & \texttt{arb} & UFT Rx &
        % asdf
        % \\  
        % \texttt{uft\_rx\_row\_size} & \texttt{arb} & UFT Rx &
        % asdf
        % \\  
        % \texttt{uft\_rx\_row\_size\_valid} & \texttt{arb} & UFT Rx &
        % asdf
        % \\  
        \texttt{uft\_user\_regX} & \texttt{arb} & UFT Rx &
        UFT user register values. They can be set from PC and serve as channel
        for Wallis parameters
        \\  
        \midrule
        %%% UFT Tx
        \texttt{uft\_o\_axis} & \texttt{AXI4-Stream} & UFT Tx &
        Transmit data stream to UFT core
        \\
        \texttt{uft\_tx\_start} & \texttt{arb} & UFT Tx &
        Asserted to initialize a transmission
        \\
        \texttt{uft\_tx\_ready} & \texttt{arb} & UFT Tx &
        Signals that the UFT core is ready to transmit
        \\
        % \texttt{uft\_tx\_row\_num} & \texttt{arb} & UFT Tx &
        % asdf
        % \\
        \texttt{uft\_tx\_data\_size} & \texttt{arb} & UFT Tx &
        How many bytes to transmit
        \\
        \midrule
        %%% Wallis
        \texttt{wa\_i\_axis} & \texttt{AXI4-Stream} & Wallis &
        Image input data to the Wallis core
        \\
        \texttt{wa\_o\_axis} & \texttt{AXI4-Stream} & Wallis &
        Processed image data from the Wallis core
        \\
        \texttt{wa\_par} & \texttt{arb} & Wallis &
        Parameters for the Wallis filter
        \\
        \bottomrule
    \end{tabular}
    \caption{Controller solution B interface ports (unlisted interfaces
    are not used)}
    \label{tab:controllerbports}
\end{table}

\clearpage
\subsubsection*{Realization}
The controller core consists of three entities joined in a top entity. Figure 
\ref{fig:dctop} shows a block diagram of the controller. The 
\texttt{dc\_control} entity takes control of housekeeping such as initiating a
UFT transmission after complete processing and resetting all instances on a new
input image. The processed pixels are buffered in the \texttt{axis\_fifo}. This
is necessary because if an Ethernet transmission is started, every clock a byte
has to be valid. The third entity, \texttt{dc\_mmu} is where the image data is
cached and sent to the Wallis filter in the right order, hence the name memory
management unit.
\\

The memory management unit works similar to a large FiFo with the addition that
elements can be read multiple times. To explain its functionality, the operation
of a simple FiFo has to be clarified. A FiFo has three main components: the
memory, a read pointer and a write pointer. Figure \ref{fig:fifo} shows the
structure of a FiFo. Five states can be identified. State \rom{1} is the initial
state after reset, both read and write pointers point to the first element in
the array. If a write occurs, the write pointer advances which is observed in
state \rom{2}. The filled elements in the array represent values that can be
read. As long as the write pointer is ahead of the read pointer, data can be
read. The next state appears when the write pointer wraps around the length of
the memory to the first element as shown in state \rom{3}. After several read
accesses the read pointer wrappes around as well and we can observe state 
\rom{4} which is equivalent to state \rom{2}. If the read pointer caught up to
the write pointer state \rom{5} is observed.

\begin{figure}[t!]
    \centering
    \begin{adjustbox}{max width=1\textwidth}
        % \tikzsetnextfilename{system-overview}
\begin{tikzpicture}[
    rounded corners=0mm,
    entity/.style={
        draw,
        minimum height=1.0cm,
        minimum width=3cm,
        fill=white,
        anchor=north west,
    },
]
    %coordinates
    \coordinate (cctl)      at (0,4);
    \coordinate (cmmu)      at (1,2);
    \coordinate (cfifo)     at (1,0);


    %nodes

    \begin{pgfonlayer}{main}
        % entities
        \node[entity, label={dc\_control}] (ctl) at (cctl) {};
        \node[entity, label={dc\_mmu}] (mmu) at (cmmu) {};
        \node[entity, label={axis\_fifo}] (fifo) at (cfifo) {};

        % % ports
        \path[draw,{Latex[length=2.5mm]}-{Latex[length=2.5mm]}] ($(ctl.180) + (0,0)$) -- ($(ctl.180) + (-2.0,0)$) node[anchor=east] {uft control};

        \path[draw,{Latex[length=2.5mm]}-{Latex[length=2.5mm]}] ($(ctl.0) + (0,0)$) -- ($(ctl.0) + (3.0,0)$) node[anchor=west] {wallis control};
        \path[draw,-{Latex[length=2.5mm]}] ($(mmu.0) + (0,0)$) -- ($(mmu.0) + (2.0,0)$) node[anchor=west] {wallis data in};
        \path[draw,{Latex[length=2.5mm]}-] ($(mmu.180) + (0,-1/6)$) -- ($(mmu.180) + (-3,-1/6)$) node[anchor=east] {uft rx};
        \path[draw,-{Latex[length=2.5mm]}] ($(fifo.180) + (0,-1/6)$) -- ($(fifo.180) + (-3,-1/6)$) node[anchor=east] {uft tx};
        \path[draw,{Latex[length=2.5mm]}-] ($(fifo.0) + (0,0)$) -- ($(fifo.0) + (2,0)$) node[anchor=west] {wallis data out};


        % % Interconnects
        \path[draw,-{Latex[length=2.5mm]}] ($(ctl.270) + (-1.3,0)$) |- ($(fifo.180) + (0,1/6)$) node[anchor=east] {};
        \path[draw,-{Latex[length=2.5mm]}] ($(ctl.270) + (-1.1,0)$) |- ($(mmu.180) + (0,1/6)$) node[anchor=east] {};
        

    \end{pgfonlayer}

    \begin{pgfonlayer}{foreground}
        
    \end{pgfonlayer} 

    % Board box
    \begin{pgfonlayer}{background}
        \node [draw, fill=gray!40, inner sep=15, fit={(ctl) (mmu) (fifo)}, label=dc\_top] (dctop) {};
    \end{pgfonlayer} 

\end{tikzpicture}
    \end{adjustbox}
    \caption{Controller solution B) block diagram}
    \label{fig:dctop}
\end{figure}

\clearpage

\begin{figure}[tb!]
    \centering
    \begin{adjustbox}{max width=\textwidth}
        % \tikzsetnextfilename{system-overview}
\begin{tikzpicture}[]

    % \coordinate (cff1)      at (0,3);
    % \coordinate (cff2)      at (4,3);
    % \coordinate (cff3)      at (8,3);
    % \coordinate (cff4)      at (2,0);
    % \coordinate (cff5)      at (6,0);
    \coordinate (cff1)      at (0,0);
    \coordinate (cff2)      at (3.5,0);
    \coordinate (cff3)      at (7,0);
    \coordinate (cff4)      at (10.5,0);
    \coordinate (cff5)      at (14,0);

    

    \begin{pgfonlayer}{main}
        
        %% State 1
        % Grid
        % 
        \foreach \x in {0,1,2,3,4,5}
            \filldraw[fill=white, draw=black] ($(cff1) + \x*(0.5,0)$) rectangle ($(cff1) + \x*(0.5,0) + (0.5,1)$);
        % Read and write pointer
        \path[draw,-{Latex[length=2.5mm]}] ($(cff1) + (0,0) + (0.25,1.5)$) -- ($(cff1) + (0,0) + (0.25,1)$) node[near start, anchor=east] {w};
        \path[draw,-{Latex[length=2.5mm]}] ($(cff1) + (0,0) + (0.25,-0.5)$) -- ($(cff1) + (0,0) + (0.25,0)$) node[near start, anchor=east] {r};
        % Filled elements


        
        %% State 2
        % Grid
        % 
        \foreach \x in {0,1,2,3,4,5}
            \filldraw[fill=white, draw=black] ($(cff2) + \x*(0.5,0)$) rectangle ($(cff2) + \x*(0.5,0) + (0.5,1)$);
        % Read and write pointer
        \path[draw,-{Latex[length=2.5mm]}] ($(cff2) + (1,0) + (0.25,1.5)$) -- ($(cff2) + (1,0) + (0.25,1)$) node[near start, anchor=east] {w};
        \path[draw,-{Latex[length=2.5mm]}] ($(cff2) + (0,0) + (0.25,-0.5)$) -- ($(cff2) + (0,0) + (0.25,0)$) node[near start, anchor=east] {r};
        % Filled elements
        \filldraw[fill=gray, draw=black] ($(cff2) + 0*(0.5,0)$) rectangle ($(cff2) + 0*(0.5,0) + (0.5,1)$);
        \filldraw[fill=gray, draw=black] ($(cff2) + 1*(0.5,0)$) rectangle ($(cff2) + 1*(0.5,0) + (0.5,1)$);

        
        %% State 3
        % Grid
        % 
        \foreach \x in {0,1,2,3,4,5}
            \filldraw[fill=white, draw=black] ($(cff3) + \x*(0.5,0)$) rectangle ($(cff3) + \x*(0.5,0) + (0.5,1)$);
        % Read and write pointer
        \path[draw,-{Latex[length=2.5mm]}] ($(cff3) + (0.5,0) + (0.25,1.5)$) -- ($(cff3) + (0.5,0) + (0.25,1)$) node[near start, anchor=east] {w};
        \path[draw,-{Latex[length=2.5mm]}] ($(cff3) + (1.5,0) + (0.25,-0.5)$) -- ($(cff3) + (1.5,0) + (0.25,0)$) node[near start, anchor=east] {r};
        % Filled elements
        \filldraw[fill=gray, draw=black] ($(cff3) + 0*(0.5,0)$) rectangle ($(cff3) + 0*(0.5,0) + (0.5,1)$);
        \filldraw[fill=gray, draw=black] ($(cff3) + 3*(0.5,0)$) rectangle ($(cff3) + 3*(0.5,0) + (0.5,1)$);
        \filldraw[fill=gray, draw=black] ($(cff3) + 4*(0.5,0)$) rectangle ($(cff3) + 4*(0.5,0) + (0.5,1)$);
        \filldraw[fill=gray, draw=black] ($(cff3) + 5*(0.5,0)$) rectangle ($(cff3) + 5*(0.5,0) + (0.5,1)$);


        
        %% State 4
        % Grid
        % 
        \foreach \x in {0,1,2,3,4,5}
            \filldraw[fill=white, draw=black] ($(cff4) + \x*(0.5,0)$) rectangle ($(cff4) + \x*(0.5,0) + (0.5,1)$);
        % Read and write pointer
        \path[draw,-{Latex[length=2.5mm]}] ($(cff4) + (1.5,0) + (0.25,1.5)$) -- ($(cff4) + (1.5,0) + (0.25,1)$) node[near start, anchor=east] {w};
        \path[draw,-{Latex[length=2.5mm]}] ($(cff4) + (0,0) + (0.25,-0.5)$) -- ($(cff4) + (0,0) + (0.25,0)$) node[near start, anchor=east] {r};
        % Filled elements
        \filldraw[fill=gray, draw=black] ($(cff4) + 0*(0.5,0)$) rectangle ($(cff4) + 0*(0.5,0) + (0.5,1)$);
        \filldraw[fill=gray, draw=black] ($(cff4) + 1*(0.5,0)$) rectangle ($(cff4) + 1*(0.5,0) + (0.5,1)$);
        \filldraw[fill=gray, draw=black] ($(cff4) + 2*(0.5,0)$) rectangle ($(cff4) + 2*(0.5,0) + (0.5,1)$);


        
        %% State 5
        % Grid
        % 
        \foreach \x in {0,1,2,3,4,5}
            \filldraw[fill=white, draw=black] ($(cff5) + \x*(0.5,0)$) rectangle ($(cff5) + \x*(0.5,0) + (0.5,1)$);
        % Read and write pointer
        \path[draw,-{Latex[length=2.5mm]}] ($(cff5) + (1.5,0) + (0.25,1.5)$) -- ($(cff5) + (1.5,0) + (0.25,1)$) node[near start, anchor=east] {w};
        \path[draw,-{Latex[length=2.5mm]}] ($(cff5) + (1.5,0) + (0.25,-0.5)$) -- ($(cff5) + (1.5,0) + (0.25,0)$) node[near start, anchor=east] {r};
        % Filled elements

        %% State names
        \node[align=right,anchor=south east] (s1) at ($(cff1) + (0,2.5)$) {State:}; 
        \node[align=center,anchor=south] (s1) at ($(cff1) + (1.5,2.5)$) {\rom{1}}; 
        \node[align=center,anchor=south] (s1) at ($(cff2) + (1.5,2.5)$) {\rom{2}}; 
        \node[align=center,anchor=south] (s1) at ($(cff3) + (1.5,2.5)$) {\rom{3}}; 
        \node[align=center,anchor=south] (s1) at ($(cff4) + (1.5,2.5)$) {\rom{4}}; 
        \node[align=center,anchor=south] (s1) at ($(cff5) + (1.5,2.5)$) {\rom{5}};  
        %% Looped
        \node[align=right,anchor=south east] (s1) at ($(cff1) + (0,2)$) {Looped:}; 
        \node[align=center,anchor=south] (s1) at ($(cff1) + (1.5,2)$) {false}; 
        \node[align=center,anchor=south] (s1) at ($(cff2) + (1.5,2)$) {false}; 
        \node[align=center,anchor=south] (s1) at ($(cff3) + (1.5,2)$) {true}; 
        \node[align=center,anchor=south] (s1) at ($(cff4) + (1.5,2)$) {false}; 
        \node[align=center,anchor=south] (s1) at ($(cff5) + (1.5,2)$) {false};        


        % \draw[step=1cm,gray,very thin] (0,0) grid (20,2);

        % \node[label={uft\_rx}] (rx) at (cff4) {};

        % \path[draw,-{Latex[length=2.5mm]}] ($(txctl.0) + (0,1.5/10)$) -| ($(txcmd.180) + (-0.5,0)$) -- ($(txcmd.180) + (0,0)$) node[anchor=west] {};
    \end{pgfonlayer}

    \begin{pgfonlayer}{foreground}
    
    \end{pgfonlayer} 


\end{tikzpicture}

    
    \end{adjustbox}
    \caption{FiFo structure}
    \label{fig:fifo}
\end{figure}

From these five states we can conclude conditions for when a read or write access is
permitted. To do this a variable named \texttt{looped} is introduced. It is set
if the write pointer wraps around (as seen in state \rom{3}) and clear if the
read pointer wraps (state \rom{4}). Listing \ref{lst:fiforeadcond} shows the
read condition.

\begin{minipage}{\linewidth}
    \begin{lstlisting}[
        style=VHDLStyle, 
        caption=FiFo read condition, 
        label=lst:fiforeadcond
        ]
if ((looped = true) or (w_ptr /= r_ptr)) then
    DataOut <= Memory(r_ptr);
end if;\end{lstlisting}
\end{minipage}

Similarly, listing \ref{lst:fifowritecond} shows the write condition.

\begin{minipage}{\linewidth}
    \begin{lstlisting}[
        style=VHDLStyle, 
        caption=FiFo write condition, 
        label=lst:fifowritecond
        ]
if ((looped = false) or (w_ptr /= r_ptr)) then
    Memory(w_ptr) := DataIn;
end if;\end{lstlisting}
\end{minipage}

Now that the mechanics of a FiFo are dissected, the memory layout of the
MMU can be illustrated. Figure \ref{fig:solbmemlayout} shows the memory layout.
At the core of the memory management unit is a two
dimensional array (it is implemented as a simple array in VHDL for better
synthesis results). Every line represents an input image line and every coloumn an
input
image coloumn. The idea is to have a full block RAM for each line which is why
from now on a line is called cache. The input image data is received line by
line. Each input line is stored in a cache. 

\clearpage
If at least $WINDOW\_LENGTH$ lines
are received, there is enough data in the memory to start a Wallis operation for
one input line. If one input line is processed the Wallis neighborhood
window slides one pixel downwards. Observed in the Y-axis (top to bottom) this
mechanism resembles
a FiFo with \texttt{cache\_w\_ptr} being the write pointer and 
\texttt{cache\_r\_ptr} representing the read pointer. The major difference to a
simple FiFo is that the distance between the read and write pointer may not
become zero because the Wallis filter requires $WINDOW\_LENGTH$ lines to process
data.

\begin{figure}[tb!]
    \centering
    \begin{adjustbox}{max width=0.95\textwidth}
        % \tikzsetnextfilename{system-overview}
\begin{tikzpicture}[
    rounded corners=0mm,
    triangle/.style = {fill=blue!20, regular polygon, regular polygon sides=3 },
    node rotated/.style = {rotate=180},
    border rotated/.style = {shape border rotate=180}
]
    %coordinates
    \coordinate (orig)      at (0,0);

    \begin{pgfonlayer}{main}
        
        % % Write path
        % \path[draw={rgb:red,1;green,2;blue,3},-{Latex[length=5mm]},line width=1.0mm] (0.5,4.5)  -- (7.5,4.5);
        % \path[draw={rgb:red,3;green,1;blue,2},-{Latex[length=5mm]},line width=1.0mm] (8.5,4.5)  -- (15.5,4.5);
        
        % % Text
        % \node[] (write) at (-2,5) {Write};

        % Braces
        \draw [line width=0.5mm,decorate,decoration={brace,amplitude=10pt},xshift=-4pt,yshift=0pt] (12.5,6) -- (12.5,0) node [black,midway,xshift=0.5cm,anchor=west] {Cache N Lines};
        \draw [line width=0.5mm,decorate,decoration={brace,amplitude=10pt},xshift=-0pt,yshift=0pt] (9,-2) -- (0,-2) node [black,midway,yshift=-0.5cm,anchor=north] {Block RAM size};
        
        % Center pixel
        \draw[black,line width=0.5mm] (2,3) rectangle (3,4);
        % Window size
        \draw[black,line width=0.5mm] (0,1) rectangle (5,6);
        
        % Buffer A
        % \draw[draw={rgb:red,1;green,2;blue,3},line width=1mm] (-0.05,-0.05) rectangle (7.95,5.05);
        
        % Buffer B
        % \draw[draw={rgb:red,3;green,1;blue,2},line width=1mm] (8.05,-0.05) rectangle (16.05,5.05);
        
        % Left Arrows
        \path[draw,-{Latex[length=2.5mm]}] (-1,5.5) -- (0,5.5) 
            node[near start,anchor=east,align=right,xshift=-0.2cm] {\texttt{cache\_w\_ptr}};
        % \path[draw,-{Latex[length=2.5mm]}] (-1,0.5) -- (0,0.5) 
        %     node[near start,anchor=east,align=right,xshift=-0.2cm] {};
        % \foreach \y in {1.5,2.5,3.5,4.5}            
        %     \node[circle,fill=gray,minimum size=0.2cm,inner sep=0pt] () at (-0.5,\y) {};

        % Top arrows
        \path[draw,-{Latex[length=2.5mm]}] (0.5,7.0) -- (0.5,6) 
            node[near start,anchor=west,align=right,xshift=-0.2cm,rotate=90] {\texttt{col\_w\_ptr}};
        % \path[draw,-{Latex[length=2.5mm]}] (8.5,7.0) -- (8.5,6) 
        %     node[near start,anchor=west,align=right,xshift=-0.2cm,rotate=90] {};
        % \foreach \x in {1.5,2.5,3.5,4.5,5.5,6.5,7.5}            
        %     \node[circle,fill=gray,minimum size=0.2cm,inner sep=0pt] () at (\x,6.5) {};

        % Right arrows
        \path[draw,-{Latex[length=2.5mm]}] (10,5.5) -- (9,5.5) 
            node[near start,anchor=west,align=left,xshift=0.2cm] {\texttt{cache\_r\_base}};
        \path[draw,-{Latex[length=2.5mm]}] (10,1.5) -- (9,1.5) 
            node[near start,anchor=west,align=left,xshift=0.2cm] {\texttt{cache\_r\_tip}};
        \path[draw,-{Latex[length=2.5mm]}] (10,4.5) -- (9,4.5) 
            node[near start,anchor=west,align=left,xshift=0.2cm] {\texttt{cache\_r\_ptr}};
        % \foreach \y in {2.5,3.5}            
        %     \node[circle,fill=gray,minimum size=0.2cm,inner sep=0pt] () at (9.5,\y) {};

        % Bot arrows
        \path[draw,-{Latex[length=2.5mm]}] (0.5,-1) -- (0.5,0) 
            node[near start,anchor=east,align=left,xshift=0.2cm,yshift=.5cm,rotate=90] {\texttt{col\_r\_ptr}};
        % \path[draw,-{Latex[length=2.5mm]}] (8.5,-1) -- (8.5,0) 
        %     node[near start,anchor=east,align=left,xshift=0.2cm,rotate=90] {};
        % \foreach \x in {1.5,2.5,3.5,4.5,5.5,6.5,7.5}            
        %     \node[circle,fill=gray,minimum size=0.2cm,inner sep=0pt] () at (\x,-0.5) {};


        % Axis
        \foreach \y in {0,1,2,3,4,5}
            \node[anchor=north] at ($(-0.2,6)-(0,\y)$)  {$\y$};
        % Axis
        \foreach \x in {0,1,2,3,4,5,6,7,8}
            \node[anchor=south] at ($(0.1,6.0)+(\x,0)$)  {$\x$};

    \end{pgfonlayer}

    % Foreground
    \begin{pgfonlayer}{foreground}
        
    \end{pgfonlayer} 

    % Background
    \begin{pgfonlayer}{background}
        % Grid
        \draw[step=1cm,black,thin] (0,0) grid (9,6);
    \end{pgfonlayer} 

\end{tikzpicture}
    \end{adjustbox}
    \caption{Memory layout for solution B}
    \label{fig:solbmemlayout}
\end{figure}

The write mechanism consists of two pointers. The \texttt{cache\_w\_ptr} pointer
points to the current cache in which the input line is stored. The 
\texttt{col\_w\_ptr} points to the location in the cache of the current pixel to
be stored. \texttt{col\_w\_ptr} is incremented every pixel and set to zero at
the start of a new line. \texttt{cache\_w\_ptr} is incremented if an image line
is received and wraps at $CACHE\_N\_LINES$.

The read mechanism consists of a total of four pointers. The \texttt{col\_r\_ptr}
pointer points to the current image coloumn that is sent to the Wallis filter.
It is incremented after $WINDOW\_LENGTH$ pixels are sent and set to zero at
the beginning of a new line. \texttt{cache\_r\_base} points to the top most
cache of image
data and \texttt{cache\_r\_tip} to the bottom most. They are both incremented
after an image line is processed and wrap at $CACHE\_N\_LINES$. The 
\texttt{cache\_r\_ptr} pointer points to the current cache from which the pixel
is read and increments if a pixel is sent to the Wallis core. It wraps to zero
if it reaches $CACHE\_N\_LINES$ and wraps to \texttt{cache\_r\_base} if it
reaches \texttt{cache\_r\_tip}. 

With the pointer mechanics dissected, new conditions for read and write can be
concluded. Listing \ref{lst:mmureadcond} shows the read condition. The
difference to the simple FiFo condition is that if not looped the write pointer
must be greater than the read tip. This ensures that always enough data is in
memory for a Wallis operation.

\begin{minipage}{\linewidth}
    \begin{lstlisting}[
        style=VHDLStyle, 
        caption=MMU read condition, 
        label=lst:mmureadcond
        ]
if ((looped = true) or (cache_w_ptr > cache_r_tip)) then
    DataOut <= Memory(BRAM_SIZE*cache_r_ptr+row_r_ptr);
end if;\end{lstlisting}
\end{minipage}

Listing \ref{lst:mmuwritecond} shows the write condition. It only differs in
that the write pointer may not be equal to the read base instead of the read
pointer as in the case of the simple FiFo.

\begin{minipage}{\linewidth}
    \begin{lstlisting}[
        style=VHDLStyle, 
        caption=MMU write condition, 
        label=lst:mmuwritecond
        ]
if ((looped = false) or (cache_w_ptr /= cache_r_base)) then
    Memory(BRAM_SIZE*cache_w_ptr+row_w_ptr) := DataIn;
end if;\end{lstlisting}
\end{minipage}

The looped flag is set if \texttt{cache\_w\_ptr} wraps to zero and is cleared if
\texttt{cache\_r\_tip} wraps to zero.

These read and write conditions are used to access the memory and to drive the
side channels of the AXI4-Stream (\texttt{tready} and \texttt{tvalid}). The 
\texttt{tlast} signal is used to indicate the end of a line. Therefore the MMU
can operate with an input and output stream and information about the image
width and window length.

The VHDL code for the memory management unit is mainly composed of processes.
Each pointer is incremented and wrapped in its own process. The memory read and
write accesses are described in processes as well. Finally there are two more
processes: \texttt{p\_out\_pix\_ctr} counts the number of pixels sent and
\texttt{p\_out\_pix\_m1} calculates the number of output pixels to send.



\subsubsection*{Hurdles}
After figuring out how to increment and wrap which pointer and when to allow
reads and writes, the memory management unit worked well in simulation. The
\texttt{dc\_control} and \texttt{axis\_fifo} block were quickly implemented. The
core was tested using VHDL testbenches. A big advantage was that the controller
and Wallis core could be simulated before being implemented on the FPGA which
made fault finding much faster. With this new stateless approach timing issues
began to arise. 

Address calculations had to be done within one clock cycle. While Vivado HLS
optimizes variables and datapaths, this has to be done manually in a VHDL based
approach. For example on critical path was the calculation of 
\texttt{n\_out\_pix\_m1}. This signal is required to compare the output pixel
counter and to signal the last pixel sent to the Wallis core. The calculation is
shown in equation \ref{eq:noutpix}.

\begin{equation}
    n_{out\_pix\_m1} = img\_width \cdot win\_len - 1
    \label{eq:noutpix}
\end{equation}

The parameters $win\_len$ and $img\_width$ were chosen of a size that the
multiplication could
be implemented in a DSP block. The subtraction of one is also done in the same
DSP block. Synthesis results showed that this DSP calculation requires
approximately 3.3ns
which is almost already half of the targeted clock period of 8ns. Together with
the datapath and following logic, this resulted in a critical path exceeding
8ns. To fix this,
this calculation had to be done in a clocked process so that a latch is
inferred. Such timing issues had to be fixed for several signals.

% timing
% 
\subsubsection*{Conclusion}
Because the requirements were already dissected from solution A there was only a
new memory layout to be realised for solution B. With the main difference being
that now the behaviour was not described in C/C++ language but in VHDL. The
total
amount of time spent on the solutions were approximately the same. What had
changed is on what the time was spent. While writing the code required the same
amount of time, debugging and fault finding differed. The HLS solution required
time in understanding how the synthesis tool translated the C/C++ code (referring
to the AXI4-Stream blocking access problem) while it did not require thoughts on
timing
and critical paths. With VHDL these accesses had to be implemented manually and
worked on first try but some time had to be spent on fixing timing issues.

Performance wise, there is the expected increase in throughtput. Output pixels
can be sent to the Wallis filter with every clock cycle as soon as there is
enough image data in the cache. Figure \ref{fig:tracesolb} shows the output data
stream with a $WINDOW\_LENGTH$ of three. 

Another advantage of this solution is that the Wallis and controller components
could be simulated together before implementation. This was not possible with
the HLS based approach. Most of the issues could be resolved during simulation.

\begin{figure}[h!]
    \centering
    \includegraphics[width=\textwidth]{images/controller/vhdlcontrollerout.png}
    \caption{Solution B output data stream}
    \label{fig:tracesolb}
\end{figure}




% ------------------------------------------------------------------------->>> %

% <<< ------------------------------------------------------------ SCALABILITY %
% ==============================================================================
%
%                             Scalability
%
% ==============================================================================
\chapter{Scalability} \label{chapt:scalability}
The code for this project is written from the beginning with the possibility to
scale it up in mind. The main idea being that the throughput can be increased.
Scalability in regard to this project can be divided into two terms:
\begin{itemize}
    \item Inside FPGA
    \item Across multiple FPGAs
\end{itemize}

In the following sections these two aspects are dissected theoretically.

\section{Inside FPGA}
Every FPGA has a given amount of ressources (LUTs, memory and so forth, as
described in table \ref{tab:XC7A200T}). The inside FPGA scalability aims towards
optimal usage of these ressources. Before methods for scalability can be
compared it must be clarified what can be scaled. The \gls{diip} project
consists of three main parts. The communication, controller and image processing
parts. The communication part is only implemented once to handle the
communication to the PC and therefore can not be scaled. The image processing
part however can be implemented multiple times inside the FPGA to increase
throughput. Therefore the conrtoller part splits up the incomming image data and
distributes it to the image processing cores. To summarize the inside FPGA
scalability:
\begin{itemize}
    \item Implement the communication part once
    \item Implement the image processing part as many times as the available
    ressources allow it
    \item Adapt the controller to distribute data accross the processing cores
\end{itemize}

Figure \ref{fig:insidefpgascaleconceptbd} clarifies the concept. Note that all
image processing cores are equal.

\begin{figure}[tb!]
    \centering
    \begin{adjustbox}{max width=\linewidth}
        % \tikzsetnextfilename{system-overview}
\begin{tikzpicture}[
    rounded corners=0mm,
]
    %coordinates
    \coordinate (corig)      at (0,0);
    \coordinate (cmonitor)   at (0,0);
    \coordinate (ccom)       at (5,0);
    \coordinate (cip)        at (10,0);


    %nodes

    \begin{pgfonlayer}{main}

        \node[draw, fill=white, minimum width=3cm, minimum height=2cm, anchor=west, text width=2.8cm, align=center] (com) at (ccom) {Controller};

        \node[draw, fill=white, minimum width=3cm, minimum height=1cm, anchor=west, text width=2.8cm, align=center, above =1cm of com] (commu) {Communication};

        \node[draw, fill=white, minimum width=3cm, minimum height=1cm, anchor=west, text width=2.8cm, align=center, right = 1cm of com, yshift=2.5cm] (ip1) {Image\\Processing};
        \node[draw, fill=white, minimum width=3cm, minimum height=1cm, anchor=west, text width=2.8cm, align=center, right = 1cm of com, yshift=1.0cm] (ip2) {Image\\Processing};
        \node[draw, fill=white, minimum width=3cm, minimum height=1cm, anchor=west, text width=2.8cm, align=center, right = 1cm of com, yshift=-0.5cm] (ip3) {Image\\Processing};
        
        \node[circle,fill=black,minimum size=0.2cm,inner sep=0pt, below = 0.3cm of ip3] (dt1)  {};
        \node[circle,fill=black,minimum size=0.2cm,inner sep=0pt, below = 0.2cm of dt1] (dt2)  {};
        \node[circle,fill=black,minimum size=0.2cm,inner sep=0pt, below = 0.2cm of dt2] (dt3)  {};

        % \node[] (eth) at ($(cmonitor) + (4.5, 1.0)$) {LAN};
        
        % \draw[line width = 0.5mm] ($(eth) + (0,-1.0)$) ellipse (0.2cm and 0.5cm);
    \end{pgfonlayer}

    % FPGA box
    \begin{pgfonlayer}{main}
        \node[above = 2.4cm of com, xshift=-1.2cm] (fpga) { FPGA };
    \end{pgfonlayer}
    \begin{pgfonlayer}{foreground}
        \node (f_fpga) [draw=black, fill=gray!20, inner sep=10, fit={(com) (ip1) (ip2) (ip3) (dt2) (dt1) (dt3)}] {};
    \end{pgfonlayer} 

    
    \path[draw,{Latex[length=2.5mm]}-{Latex[length=2.5mm]}] ($(commu.180) + (-1.5,0)$) -- ($(commu.180) + (0,0)$) node[near start, left, anchor=east,xshift=-0.5cm] () {PC} ;
    \path[draw,{Latex[length=2.5mm]}-{Latex[length=2.5mm]}] ($(com.90) + (0,0)$) -- ($(commu.270) + (0,0)$) node[near start, left, anchor=east,xshift=-0.5cm] () {} ;

    \path[draw,{Latex[length=2.5mm]}-{Latex[length=2.5mm]}] 
        ($(com.0) + (0,0.7)$) -| ($(ip1.180) + (-0.6,0)$) -- ($(ip1.180) + (0,0)$)
         node[near start, left, anchor=east,xshift=-0.5cm] () {} ;
    \path[draw,{Latex[length=2.5mm]}-{Latex[length=2.5mm]}] 
        ($(com.0) + (0,0.1)$) -| ($(ip2.180) + (-0.4,0)$) -- ($(ip2.180) + (0,0)$) 
        node[near start, left, anchor=east,xshift=-0.5cm] () {} ;
    \path[draw,{Latex[length=2.5mm]}-{Latex[length=2.5mm]}] ($(com.0) + (0,-0.5)$) -- ($(ip3.180) + (0,0)$) node[near start, left, anchor=east,xshift=-0.5cm] () {} ;


\end{tikzpicture}
    \end{adjustbox}
    \caption{Block diagram of a inside FPGA scaled solution}
    \label{fig:insidefpgascaleconceptbd}
\end{figure}

Now that the concept is clarified, two possible solutions are compared to
distribute the image data accross the image processing cores, proposal A and B.
They differ in the way in what order the data is sent to the FPGA and how it is
cached inside the FPGA. For each proposal the following metrics are calculated.

\begin{description}
    \item[Initial size $s_i$]\hfill \\
    The number of pixels that have to be sent to the FPGA before beginning
    interational operation.
    \item[Iteration size $s_r$]\hfill \\
    How many pixels that have to be sent to start a new iteration.
    \item[Store size $s_s$]\hfill \\
    How many pixels have to be cached inside the FPGA.
    \item[Number of inits per image $n_i$]\hfill \\
    Denotes the number of initial data transfers of size $s_i$ have to be made
    per image.
    \item[Total tx size $s_{tx}$]\hfill \\
    The total number of pixels sent to the FPGA to calculate one image.
\end{description}


\begin{table}[tb!]
    \centering
    \begin{tabular}{p{0.45\textwidth} p{0.45\textwidth}}
        \toprule
        \multicolumn{1}{c}{Solution A} & \multicolumn{1}{c}{Solution B} \\
        \midrule
        Insert figure & Insert figure \\\midrule
        \textbf{Concept} & \\
        Image data is sent line wise. Each image processing core handles one
        input line, starts as soon as enough data is transfered and jumps $N$ lines
        downwards after completion.
        &
        Image data is sent coloumn wise with $w_l+(N-1)$ pixels per coloumn. All
        image processing cores start processing at the same time and progress
        $N$ lines downwards after completion.
        \\\midrule
        \textbf{Initial size} & \\
        {\( 
            s_i = i_w(w_l+(N-1))
        \)}
        &
        {\( 
            s_i = w_l(w_l+(N-1))
        \)}
        \\\midrule
        \textbf{Iterarion size} & \\
        {\( 
            s_r  = i_w
        \)}
        &
        {\( 
            s_r  = w_l+(N-1)
        \)}
        \\\midrule
        \textbf{Store size} & \\
        {\( 
            s_s  = i_w(w_l+(N-1))
        \)}
        &
        {\( 
            s_s  = w_l(w_l+(N-1))
        \)}
        \\\midrule
        \textbf{Number of inits per image} & \\
        {\( 
            n_i  = 1
        \)}
        &
        {\( 
            n_i  = \frac{1}{N}(i_w-w_l+1)
        \)}
        \\\midrule
        \textbf{Total tx size} & \\
        {\( 
            s_{tx}  = i_w \cdot i_h
        \)}
        &
        {\( 
            s_{tx}  = n_i s_i + (i_w-w_l)s_r n_i
        \)}
        \\
        \bottomrule
    \end{tabular}
    \caption{Inside FPGA scalability methods}
    \label{tab:insidefpgascalability}
\end{table}

Before comparing the two proposals another metric has to be taken into
consideration: The effective throughput of the image processing core. The way
the wallis filter works is that it requires a neighbourhood of pixels. For each
line the core processes, the initial neighbourhood has to be resent. Equation
\ref{eq:theomaxvhdlwallise} derived in appendix 
\ref{app:derivations:theomaxvhdlwallis} is used to calculate the effective
throughput $b_r$ of a Wallis filter core. To simplify, the image height $i_h$ is
considered to be much larger than the window length $w_l$ and therefore the
equation can be noted as shown in equatio \ref{eq:vhdlwallismaxb}. Using the
throughput of the VHDL Wallis filter implementation ($b_w=125Mp/s$) and the
window length $w_l=21$ the effective bandwidth $b_r$ is calculated:

\begin{align}
    b_r  & \approx \frac{b_w}{w_l} = 5.95 Mp/s
    \label{eq:vhdlwallismaxb}
\end{align}

Under the assumption that the real throughput $b_r$ scales proportionally with the
number of image processing cores implemented, equation 
\ref{eq:scaledrealttotalhroughput} can be noted to represent the total image
processing throughput $b_t$.

\begin{align}
    b_t  & = N \cdot b_r
    \label{eq:scaledrealttotalhroughput}
\end{align}

Considering the maximum throughput of the Gigabit-Ethernet connection of
$b_e=125MB/s$ \footnote{True Ethernet throughput is less than 125MB/s
considering packet overhead} equation \ref{eq:maxethernetthrouhgput} can be derived to calculate the
maximum number of image processing cores that can be implemented before the
Ethernet communication link is saturated.

\begin{align}
    b_e \geq b_t = N \cdot b_r \Rightarrow N \leq \frac{b_e}{b_r} \approx 
    \frac{b_e w_l}{b_w} = w_l = 21
    \label{eq:maxethernetthrouhgput}
\end{align}

To calculate the free memory the results from \ref{asdf} \todo{ref to ressource
VHDL} are dissected to calculate the available memory. Table \ref{tab:membudget}
lists the required block memory usage.

\begin{table}[h!]
    \centering
    \begin{tabular}{l r}
        \toprule
        Item & BRAM tiles \\
        \midrule
        Total available & 365 \\
        TEMAC support & -2 \\
        UDP IP Core & -1.5 \\
        21 divider generators & -10.5 \\
        21 Wallis cores & -10.5 \\
        \midrule
        Free & 340.5\\
        \bottomrule
    \end{tabular}
    \caption{FPGA block memory budget}
    \label{tab:membudget}
\end{table}

After using up memory for the communication and image processing cores, 340.5
block RAM tiles with 36Kb each results in 1'532KB of remaining block RAM that
can be used for image caching. Table \ref{tab:parsum} summarizes the
parameters that are used to compare the two inside FPGA scale proposals.

\begin{table}[h!]
    \centering
    \begin{tabular}{l r l}
        \toprule
        Parameter & Value & Description\\
        \midrule
        $N$ & 21 & Number of image processing cores to implement \\
        $s_b$ & 1'532KB & Block memory storage available for cache \\
        $w_l$ & 21 & Window length of the Wallis filter \\
        \bottomrule
    \end{tabular}
    \caption{Parameter summary}
    \label{tab:parsum}
\end{table}


\section{Across FPGA}
text

\section{Usecase}


% ------------------------------------------------------------------------->>> %

% <<< ----------------------------------------------- BENCHMARK & VERIFICATION %
% ==============================================================================
%
%                             Verification & Benchmark
%
% ==============================================================================
\chapter{Verification \& Benchmark} \label{chapt:ver_bench}

\section{Verification} \label{ch:verification}

\section{Benchmark} \label{ch:benchmark}

% ------------------------------------------------------------------------->>> %

% <<< ------------------------------------------------------------- CONCLUSION %
% ==============================================================================
%
%                             Conclusion
%
% ==============================================================================
\chapter{Conclusion}
In this last chapter the project results are briefly summarized and
a short outlook for possible future work is given.

% ==============================================================================
%
%                             IP
%
% ==============================================================================
\section{Image Processing}
A new image processing core was implemented that locally optimizes contrast 
(Wallis filter). Multiple implementations have been validated and compared.
Using high level synthesis to describe an algorithm has proven that the desired
operation can be achieved in little time. Nevertheless the theoretical possible
throughput could not be achieved. The main reason being that the dataflow was
not described well enough in C/C++ language to achieve the desired throughput.
Therefore a VHDL implementation was written that processes pixels at 125Mp/s and
uses about three percent of FPGA ressources.

% ==============================================================================
%
%                             Dataflow
%
% ==============================================================================
\section{Dataflow}
The existing communication core was extended with acknowledge, user registers
and AXI4-Stream interfaces. The new image processing algorithm requires the
input data in a specific order. Therefore a controller was implemented using two
different design flows. The HLS approach showed that complex interfaces such as
AXI4 can be implemented with a few lines of code and that state machines can be
implemented as well. The unfavourable memory management and the lack of high
throughput
led to the implementation of a VHDL based controller and memory management unit.
It caches necessary image data to obviate multiple image transmissions and can
support the full throughput of the image processing core.

% ==============================================================================
%
%                             Overall
%
% ==============================================================================
\section{Overall}
The final product called \gls{diip} is based on the VHDL implementations of
image processing and dataflow parts. Images can be sent from the PC to the FPGA
where
the Wallis operation is applied and the image is sent back. Image data
throughput of up to 4.1MB/s have been measured. The system is designed with
scalability in mind. A dedicated theoretical examination shows that if the
image processing core was implemented 21 times in the FPGA the full bandwidth of
gigabit Ethernet could be used to process image data yielding 125MB/s
throughput. Furthermore if multiple FPGAs were used, the total throughput would
scale proportional to the number of FPGAs used.

% ==============================================================================
%
%                             Working with High Level Synthesis
%
% ==============================================================================
\section{Working with High Level Synthesis}
The time spent on working with \gls{vivadohls} has shown several advantages and
disadvantages. It has shown that thinking close to hardwre is crucial. Using
C/C++ as language is a trap to fall back and think of the code as if it would be
executed sequentially. The function to be implemented should be split into
building blocks the same way as it would be done using a hardware description
language. If the code was written with the exact same line of thought as a HDL
approach, then the same throughput should be achievable. And if that is managed,
HLS would bring significant advantages like the simple implementation of complex
interfaces (e.g. AXI4) and the reduced time in testbench development.

% ==============================================================================
%
%                             Future Work
%
% ==============================================================================
\section{Future Work}
The scalability is proven on paper and can now be implemented onto the FPGA.
The controller core can be extended to send the data to multiple Wallis filter
cores. After writing a computer application that can handle the fast transfer
speeds the true benefit of using FPGA for image processing can be shown in
praxis. Because the data transfer is Ethernet based a cluster of FPGAs can be
built to compete against high performance CPU and GPU based image processing
pipelines. Another method to increase throughput would be to implement a VHDL
256bit image processing core that would be able to process 21 pixels per clock
cycle and produce one pixel per clock cycle on the output. This would omit the
scalability inside the FPGA because the Ethernet link would already be
saturated. Furthermore the HLS implementations could be improved by describing
the dataflow more hardware oriented and proove that using HLS the same
throughput can be achieved as when using a hardware description language.


% ------------------------------------------------------------------------->>> %

%%%%%%%%%%%%%%%%%%%%%%%%%%%%%%%%%%%%%%%%%%%%%%%%%%%%%%%%%%%%%%%%%%%%%%%%%%%%%%%%
%                        	    R E S O U R C E  							   %
%%%%%%%%%%%%%%%%%%%%%%%%%%%%%%%%%%%%%%%%%%%%%%%%%%%%%%%%%%%%%%%%%%%%%%%%%%%%%%%%
\phantomsection

%%%%%%%%%%%%%%%%%%%%%%%%%%%%%%%%%%%%%%%%%%%%%%%%%%%%%%%%%%%%%%%%%%%%%%%%%%%%%%%%
%                             G L O S S A R Y                                  %
%%%%%%%%%%%%%%%%%%%%%%%%%%%%%%%%%%%%%%%%%%%%%%%%%%%%%%%%%%%%%%%%%%%%%%%%%%%%%%%%
% \clearpage{\pagestyle{empty}\cleardoublepage}
\printglossaries
% \clearpage{\pagestyle{empty}\cleardoublepage}

% <<< ----------------------------------------------------------- BIBLIOGRAPHY %
% \addcontentsline{toc}{chapter}{Bibliography}
% \bibliographystyle{IEEEtran}
% \bibliography{bibtex}

%%---BIBLIOGRAPHY------------------------------------------------------------------------
{\sloppypar
\label{sec:lit}
\selectlanguage{english}                %ngerman or english
\printbibliography[heading=bibintoc]
}

% ------------------------------------------------------------------------->>> %

% <<< -------------------------------------------------------- LIST of FIGURES %
\listoffigures
\listoftables
% \lstlistoflistings

\clearpage{\pagestyle{empty}\cleardoublepage}
% ------------------------------------------------------------------------->>> %
 

%%%%%%%%%%%%%%%%%%%%%%%%%%%%%%%%%%%%%%%%%%%%%%%%%%%%%%%%%%%%%%%%%%%%%%%%%%%%%%%%
%                             A P P E N D I C E S							   %
%%%%%%%%%%%%%%%%%%%%%%%%%%%%%%%%%%%%%%%%%%%%%%%%%%%%%%%%%%%%%%%%%%%%%%%%%%%%%%%%
\clearpage{\pagestyle{empty}\cleardoublepage}
\begin{appendix}
% ==============================================================================
%
%                             Appendices
%
% ==============================================================================
\chapter{Appendices}\label{chp:appendices}

\section{Aufgabenstellung 2018 P6 Distributed FPGA} \label{app:aufgabenstellung}
\includepdf[pages=-,scale=1,linktodoc=false]{appendices/Aufgabenstellung_2017_P5_Distributed_FPGA_comb.pdf}

\section{Technicial requirements} \label{app:technicial_requirements}
\includepdf[pages=-,scale=1,linktodoc=false]{appendices/technical_requirements_P6.pdf}

\section{UDP File Transfer} \label{app:uftspec}
\includepdf[pages=-,scale=1,linktodoc=false]{appendices/UDP_File_Transfer.pdf}

\section{UDP file transfer calculation} \label{app:uftcalc}
\includepdf[pages=-,scale=1,linktodoc=false]{appendices/uftcalc.pdf}


\section{Derivations} \label{app:derivations}
\subsection{Theoretical maximum throughput of VHDL solution} \label{app:derivations:theomax}

\begin{align}
    b  & = \frac{i_p}{t_t} \\
       & = \frac{i_w i_h}{t_i+(i_h-w_l+1)t_r} \\
       & = \frac{i_w i_h}{(\frac{i_w}{b_e}+d_l)w_l+(i_h-w_l+1)(\frac{i_w}
       {b_e}+d_l)} \\
       & = \frac{i_w i_h}{(\frac{i_w}{b_e}+d_l)(i_h+1)} \\
       & = \frac{i_w i_h}{d_l(i_h+1)+\frac{i_w}{b_e}(i_h+1)} \\
       & \approx \frac{i_w}{d_l+\frac{i_w}{b_e}}
    \label{eq:theomaxb}
\end{align}
\begin{tabular}{rl}
    $b     =$ & theoretical throughput of VHDL solution \\
    $i_p   =$ & total image pixels \\
    $t_t   =$ & total processing time \\
    $i_w   =$ & image width \\
    $i_h   =$ & image height \\
    $t_i   =$ & time to send the initial lines \\
    $w_l   =$ & windows length \\
    $t_r   =$ & iterration time to process one line \\
    $b_e   =$ & ethernet throughput \\
    $d_l   =$ & delay between sending two image lines \\
\end{tabular} \\

\subsection{Theoretical maximum throughput of VHDL Wallis core} 
\label{app:derivations:theomaxvhdlwallis}

\begin{align}
    b  & = \frac{i_p}{t_w} &\\
       & = \frac{i_wi_h}{\frac{n_{ps}}{b_w}} &\\
       & = \frac{i_w i_h b_w}{w_l i_w (i_h-w_l+1)} &\\
       & = \frac{i_h}{w_l (i_h-w_l+1)} &\\
       & = \frac{b_w}{w_l (1-\frac{w_l}{i_h}+\frac{1}{i_h})} &\\
       & \approx \frac{b_w}{w_l} &(i_h \to \infty)
    \label{eq:theomaxvhdlwallis}
\end{align}
\begin{tabular}{rl}
    $b     =$ & theoretical throughput of VHDL Wallis filter \\
    $i_p   =$ & total image pixels \\
    $t_w   =$ & total processing time of Wallis filter \\
    $i_w   =$ & image width \\
    $i_h   =$ & image height \\
    $n_{ps}=$ & Number of pixels processed by Wallis filter on input \\
    $b_w   =$ & throughput of Wallis filter \\
    $w_l   =$ & window length \\
\end{tabular} \\
\end{appendix}
\clearpage{\pagestyle{empty}\cleardoublepage}


% ==============================================================================
%
%                             B A C K M A T T E R
%
% ==============================================================================
\backmatter
\clearpage{\pagestyle{empty}\cleardoublepage}
% % Indexing: memman.pdf, pp. 302ff.
% \printindex
% %>>>
\newpage\null\thispagestyle{empty}\newpage
\end{document}

