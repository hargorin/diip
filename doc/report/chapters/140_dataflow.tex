% ==============================================================================
%
%                             Dataflow
%
% ==============================================================================
\chapter{Dataflow} \label{chapt:dataflow}
With the Wallis filter implemented on the FPGA the problem occurs that the
image data has to be sent from the computer to the FPGA and back. The following
chapter explains the realization of said dataflow. It is split into two main
parts: the communcation and control parts. But before diving into them the
concept of the dataflow is briefly described in the following chapter.\\

During the work on the project it has been discovered that the initial approach
to
the problem will not lead to the optimum solution. This is why a second
implementation was made that differs in both communication and control parts. To
prevent confusion these two solutions are referred to as solution A (the first
approach) and solution B (the second, more performant approach).

% ==============================================================================
%                             Concept
% ==============================================================================
\section{Concept} \label{ch:concept}
As seen in the introduction the image data is sent to the FPGA over Ethernet.
This means that on the FPGA side an IP stack has to be implemented to handle
Ethernet
communication. The received data is sent to the Wallis filter for
processing and its results are sent back to the host PC. To distinguish
the
data transmission from the dataflow inside the FPGA, the dataflow was split into
two blocks: the communication and control block. The communication block handles
the Ethernet communication while the control block feeds the data in the right
order to the Wallis filter and does housekeeping work. Figure \ref{fig:dataflowa}
shows the dataflow for solution A.
\clearpage
Image data received from the UFT core is stored directly into FPGA block memory
through AXI4 memory mapped interfaces. The UFT core signals a complete file
transfer by asserting \texttt{rx\_done}. This is when the controller starts
reading the data from memory and sending it through AXI4-Stream to the Wallis
filter. The processed data is again stored in block memory. If all data has been
processed, the controller configures the UFT core with the amount of data to
send back and starts the transmission.

\begin{figure}[t!]
    \centering
    \begin{adjustbox}{max width=\textwidth}
        % \tikzsetnextfilename{system-overview}

%-----ABoxes
%-----#1 height, #2 width, #3 aspect, #4 name of the node, #5
%-----coordinate, #6 label
\def\memory[#1,#2,#3,#4,#5]#6{%
  \node[draw, cylinder, alias=cyl, shape border rotate=90, aspect=#3, %
  minimum height=#1, minimum width=#2, outer sep=-0.5\pgflinewidth, %
  color=white!40!black, left color=white!70, right color=white!80, middle
  color=white] (#4) at #5 {};%
  \node at #5 {#6};%
  \fill [white!30] let \p1 = ($(cyl.before top)!0.5!(cyl.after top)$), \p2 =
  (cyl.top), \p3 = (cyl.before top), \n1={veclen(\x3-\x1,\y3-\y1)},
  \n2={veclen(\x2-\x1,\y2-\y1)} in (\p1) ellipse (\n1 and \n2); }

\begin{tikzpicture}[
    rounded corners=0mm,
]
    %coordinates
    \coordinate (ccom)       at (0,0);
    \coordinate (cbram)       at (6,0);
    \coordinate (cip)       at (3,-2);


    %nodes

    \begin{pgfonlayer}{main}

        % Blocks
        \memory[45,40,1.6,bram,(cbram)] {BRAM};

        \node[draw, fill=white, minimum width=3.5cm, minimum height=1cm, anchor=west, text width=2.8cm, align=center] (com) at (ccom) {UFT core};

        \node[draw, fill=white, minimum width=3.5cm, minimum height=1cm, anchor=west, text width=2.8cm, align=center, below = 1cm of bram] (control)  {Controller};

        \node[draw, fill=white, minimum width=3cm, minimum height=1cm, anchor=west, text width=2.8cm, align=center, right = 2cm of control] (ip)  {Image\\Processing};
        
        % Paths
        % UFT to BRAM
        \path[draw,{Latex[length=2.5mm]}-{Latex[length=2.5mm]}] 
            ($(ccom.0) + (3.5,0)$) -- ($(cbram.180) + (-0.7,0.0)$) 
            node [midway, above] () {AXI} ;
        % control to BRAM
        \path[draw,{Latex[length=2.5mm]}-{Latex[length=2.5mm]}] 
            ($(control.0) + (-1.75,0.5)$) -- ($(cbram.180) + (0,-0.7)$) 
            node [midway, right] () {AXI} ;
        % control to ip
        \path[draw,-{Latex[length=2.5mm]}] 
            ($(control.0) + (0,0.3)$) -- ($(ip.180) + (0,0.3)$) 
            node [midway, above] () {Stream} ;
        \path[draw,-{Latex[length=2.5mm]}] 
            ($(ip.180) + (0,-0.3)$) -- ($(control.0) + (0,-0.3)$)
            node [midway, above] () {Stream} ;
        % Control paths
        \path[draw,-{Latex[length=2.5mm]}] 
            ($(control.0) + (-3.5,0)$) -| ($(ccom.0) + (1.75,-0.5)$) 
            node [near start, above] () {control} ;
        \path[draw,-{Latex[length=2.5mm]}] 
            ($(control.0) + (-1.75,-0.5)$) |- ($(control.0) + (0,-1)$) node [near end, above] () {control} -| ($(ip.0) + (-1.75,-0.5)$) 
             ;
        
        % \path[draw,{Latex[length=2.5mm]}-] ($(mon.0) + (0,-0.2)$) -- ($(com.180) + (0,-0.2)$) node[near end, below] () {4.} ;

        % \path[draw,-{Latex[length=2.5mm]}] ($(com.0) + (0,0.2)$) -- ($(ip.180) + (0,0.2)$) node[midway, above] () {2.} ;
        % \path[draw,{Latex[length=2.5mm]}-] ($(com.0) + (0,-0.2)$) -- ($(ip.180) + (0,-0.2)$) node[midway, below] () {3.} ;

    
    \end{pgfonlayer}

    % FPGA box
    \begin{pgfonlayer}{main}
        % \node[above = 0.2cm of com, xshift=-1.5cm] (fpga) { FPGA };
    \end{pgfonlayer}
    \begin{pgfonlayer}{foreground}
        % \node (f_fpga) [draw=black, fill=gray!20, inner sep=20, fit={(com) (ip) }] {};
    \end{pgfonlayer} 

    

\end{tikzpicture}
    \end{adjustbox}
    \caption{Dataflow inside FPGA for solution A}
    \label{fig:dataflowa}
\end{figure}

Following the path of dataflow it is easy to see that the data is stored
multiple
times: in the receiving FiFo of the UFT core, the block memory, then by the
controller in the transmitting and receiving path, again in block memory and in
a Tx FiFo of the UFT transmitter. This causes latency and high ressource usage.
Because of that, in a second approach (solution B) the datapath was chosen more
directly. Key elements are the AXI4-Stream interfaces. Figure \ref{fig:dataflowb}
shows the new approach using said stream interfaces. The UFT stack outputs
received data comming from the Ethernet line with low latency. The controller
then only buffers the necessary data and starts streaming pixel data to the
Wallis filter as soon as enough data is received. Processed data is buffered in
the controller until the transmitter is ready to send data back.

\begin{figure}[h!]
    \centering
    \begin{adjustbox}{max width=\textwidth}
        \input{images/controller/dataflowb.tikz}
    \end{adjustbox}
    \caption{Dataflow inside FPGA for solution B}
    \label{fig:dataflowb}
\end{figure}

\pagebreak

% ==============================================================================
%                             Communication
% ==============================================================================
\section{Communication}
The UFT core from project 5 serves as a starting point to the communication
problem. Altough it works on a fundamental basis, the following key features are
missing:

\begin{itemize}
	\item Acknowledgment of received data
	\item Retransmission if a packet was not received
	\item Control interface
	\item Stream based data interface
\end{itemize}

In the following chapters these four problems are addressed and the solutions
explained. To read more about the communication core as it was used from the
preceeding project the p5 project report can be consulted \cite{p5report}.
Furthermore the UDP File Transfer (UFT) Protocol Specifications (appendix 
\ref{app:uftspec}) and calculations (appendix \ref{app:uftcalc}) give
information about the communication protocol.

% ==============================================================================
%                             Acknowledge
% ==============================================================================
\subsection{Acknowledge}
The communication is based on the User Datagram Protocol (UDP). This protocol
features low overhead and is simple to implement. The main problem is that the
data sent is not acknowledged hence the sender has no feedback wheather the data
was received or not. This is one reason a custom session protocol (\gls{uft})
was introduced in the preceeding project. It provides command packets for data
packet acknowledgment. Table \ref{tab:uftcommandlist} lists the UFT commands.


\begin{table}[b!]
    \centering
    \begin{tabular}{l l l l l}
        \toprule
        Command & Short & Name & Data 1 & Data 2 \\
        \midrule
        0x00 & FTS & 
        File transfer start & TCID & NSEQ
        \\
        0x01 & FTP &
        File transfer stop & TCID & 0x0000 0000
        \\
        0x02 & ACKFP &
        Acknowledge data packet & TCID & SEQNBR
        \\
        0x03 & ACKFT &
        Acknowledge file transfer & TCID & 0x0000 0000
        \\
        \bottomrule
    \end{tabular}
    \caption{UFT command list}
    \label{tab:uftcommandlist}
\end{table}

The UFT core was extended with the functionality to send acknowledges back to
the PC for data packets and file transfers. For this to work, new connections
inside the UFT block were made. Drawn red in figure \ref{fig:ufttop}, the 
\texttt{uft\_rx\_mem\_ctl} block signals the \texttt{uft\_tx\_ctl} block to send
an acknowledgment. The connection consists of the signals listed in table 
\ref{tab:acksignals}. The tx control block latches the two request signals 
\texttt{ack\_cmd\_nseq} and \texttt{ack\_cmd\_ft} in the case a transmission is
running and an acknowledge request is made. If the tx state machine enters idle
or wait state it checks these latches for an acknowledge request. If a request
is latched, the tx command assembler and tx arbiter are turned on to send an
acknowledge packet. After the packet was sent, the according 
\texttt{ack\_cmd\_nseq\_done}/\texttt{ack\_cmd\_ft\_done} signals are asserted.


\begin{figure}[t!]
    \centering
    \begin{adjustbox}{max width=\textwidth}
        \input{images/dataflow/ufttop.tikz}
    \end{adjustbox}
    \caption{UFT Top Block Design with acknowledge path}
    \label{fig:ufttop}
\end{figure}


\begin{table}[tb!]
    \centering
    \begin{tabular}{l l l p{8.5cm}}
        \toprule
        Signal & in/out & width & Description \\
        \midrule
        \texttt{ack\_cmd\_nseq} & out & 1 &
        Command to send a data packet acknowledge packet
        \\
        \texttt{ack\_cmd\_ft} & out &1 &
        Command to send a file transfer acknowledge packet
        \\
        \texttt{ack\_cmd\_nseq\_done} & in &1 &
        Asserted if the data packet was acknowledged
        \\
        \texttt{ack\_cmd\_ft\_done} & in &1 &
        Asserted if the file transfer was acknowledged
        \\
        \texttt{ack\_seqnbr} & out & 24 &
        What sequence number to acknowledge
        \\
        \texttt{ack\_tcid} & out & 7 &
        What transaction id to acknowledge
        \\
        \texttt{ack\_dst\_port} & out & 16 &
        Destination port of the host to send the acknowledge to
        \\
        \texttt{ack\_dst\_ip} & out & 32 &
        Destination IP address of the host to send the acknowledge to
        \\
        \bottomrule
    \end{tabular}
    \caption{ACK signals}
    \label{tab:acksignals}
\end{table}

% ==============================================================================
%                             Retransmission
% ==============================================================================
\subsection{Retransmission}
Now that the FPGA sends an acknowledge packet for each data packet, the sending
PC can check if all packets were received by the FPGA. Therefore the PC software
was extended with retransmission. Before sending data an array is allocated that
will hold the information if a packet was received. Each element represents a
data packet. Listing \ref{lst:ackbufalloc} shows the allocation of the
acknowledge buffer.

\begin{minipage}{\linewidth}
    \begin{lstlisting}[
        style=CStyle, 
        caption=ack buffer allocation, 
        label=lst:ackbufalloc
        ]
uint8_t *ack_buf = (uint8_t*)malloc( nseq * sizeof(uint8_t) );
memset(ack_buf, 0, nseq * sizeof(uint8_t));\end{lstlisting}
\end{minipage}

\pagebreak
During transmission, the sender checks the receiving buffer for data. If an
acknowledge packet has been received, the acknowledge array is updated.
This allows the sending routine to retransmit packets that were not
acknowledged. Listing \ref{lst:ackupdate} shows the acknowledge buffer update
after packet is received.

\begin{minipage}{\linewidth}
    \begin{lstlisting}[
        style=CStyle, 
        caption=ack update, 
        label=lst:ackupdate
        ]
Recv(sockfd, buf, 1500, 0);
if(get_command(buf) == CONTROLL_ACKFP)
{
    ack_buf[get_command_ackfp_seqnbr(buf)] = 1;
}\end{lstlisting}
\end{minipage}


% ==============================================================================
%                             Control Interface
% ==============================================================================
\subsection{Control Interface}
The data received by the UFT core in solution A is stored in memory and data to
be sent back
is also located in FPGA memory. To instruct the core to send data, a control
interface is required. Tx base address, data size and rx base address are values
that will be sent from the controller to the UFT core as described later in
chapter \ref{ch:control}. For this purpose an AXI4-Lite slave interface was
realized. AXI4-Lite is a stripped-down version of AXI4. It allows single beat,
memory mapped read and write access from master to slave \cite{axispecs}.

Using Vivados \texttt{Create and Package New IP} command, a 16 register
AXI4-Lite slave interface was generated and connected with the according control
signals of the UFT core. Table \ref{tab:uftaxiregmap} shows the register map.
Registers 8 through 15 can be written by the PC using a custom UFT command
packet. This allows parameters to be sent from PC to FPGA.

\begin{table}[tb!]
    \centering
    \begin{tabular}{l l l p{10cm}}
        \toprule
		Nr & R/W & Offset [hex] & Description \\
        \midrule
		0 & RO & 00 & Status register. Bit 0: \texttt{tx\_ready}. Set if transmitter is
		ready to send data. \\
		1 & WO & 04 & Control register. Bit 0: \texttt{tx\_start}. Set 1 to start
		transmitter. \\
		2 & WO & 08 & Receiver base address. Received data is stored starting from this
		address. \\
		3 & WO & 0C & Transmitter base address. Data to send is read from this address.
		\\
		4 & RO & 10 & Rx counter. Counts how many file transfers were received. \\
		5 & WO & 14 & Tx size. How many bytes to send. \\
		6 & & 18 &  Not used\\
		7 & & 1C &  Not used\\
		8 & RO & 20 & User register 0 \\
		9 & RO & 24 & User register 1 \\
		10 & RO & 28 & User register 2 \\
		11 & RO & 2C & User register 3 \\
		12 & RO & 30 & User register 4 \\
		13 & RO & 34 & User register 5 \\
		14 & RO & 38 & User register 6 \\
		15 & RO & 3C & User register 7 \\
        \bottomrule
    \end{tabular}
    \caption{UFT core register map. RO=read only, WO=write only}
    \label{tab:uftaxiregmap}
\end{table}

This new control interface allows the control and status signals from the UFT
core to be read and written using a single normed interface. Figure 
\ref{fig:uftcoreaxilite} shows the complete UFT IP core with AXI4-Lite control
interface, stream and control interface to the UDP core and AXI master burst
interface for memory access (refer to \cite{p5report} for AXI master burst).

\begin{figure}[b!]
    \centering
    \includegraphics[width=0.45\textwidth] {images/dataflow/uftcoreaxilite.png}
    \caption{UFT core with AXI4-Lite interface}
    \label{fig:uftcoreaxilite}
\end{figure}

% ==============================================================================
%                             Data Interface
% ==============================================================================
\subsection{Data Interface} \label{ch:data:com:datainterface}
The UFT core was in the first place intended for file based data transfer from
PC to FPGA and vice versa. This led to the decision to use memory mapped AXI
interface to store the received data and read the data to be sent. It made
sense with the UFT protocol being file oriented. In the progress of developing
the Wallis filter, a stream based approach was pursued to reduce latency and
memory usage. With the UFT core being memory based, a controller had to be
introduced to send the data from memory to the Wallis filter. Early tests
foreshowed that the UFT core with its memory based interface would be the
limiting member in the data processing chain. So the core's data interface was
rewritten to support AXI4-Stream for data receiving and sending.

For solution B, the two \texttt{axi\_master\_burst} blocks were removed and the
code of the \texttt{uft\_rx\_mem\_ctl} and \texttt{uft\_tx\_data\_assembler} was 
changed to directly connect the AXI4-Stream from the UDP stack to the outside.
Advantage of this solution is the reduced latency and less memory usage. One
downside is that the receiver is no longer able to reorder packets if they do
not arrive in order. As long as the application runs on a closed network it can
be assumed that packets will not change order. The changes made for the
streaming interfaces are shown in figure \ref{fig:ufttopstream}.

\begin{figure}[b!]
    \centering
    \begin{adjustbox}{max width=\textwidth}
        % \tikzsetnextfilename{system-overview}
\begin{tikzpicture}[
    rounded corners=0mm,
    entity/.style={
        draw,
        minimum height=1.0cm,
        minimum width=3cm,
        fill=white,
        anchor=north west,
    },
    entityold/.style={
        draw=gray!60,
        minimum height=1.0cm,
        minimum width=3cm,
        fill=gray!20,
        anchor=north west,
    },
]
    %coordinates
    \coordinate (orig)      at (0,0);
    \coordinate (crx)       at (0,0);
    \coordinate (crxmem)    at (5,0);

    \coordinate (ctxctl)    at (0,-2.5);
    \coordinate (ctxcmd)    at (5,-3.5);
    \coordinate (ctxdat)    at (5,-5.5);
    \coordinate (ctxarb)    at (10,-4.5);
    \coordinate (caxilite)  at (-5,-2.5);

    \coordinate (ctxamb)    at (0,-5.5);
    \coordinate (crxamb)    at (10,0);
    %nodes

    \begin{pgfonlayer}{main}
        % entities
        \node[entity, label={uft\_rx}] (rx) at (crx) {};
        \node[entity, label={[name=rxl] uft\_rx\_mem\_ctl}] (rxmem) at (crxmem) {};

        \node[entity, label={[name=ltxctl] uft\_tx\_ctl}] (txctl) at (ctxctl) {};
        \node[entity, label={[name=txcal]uft\_tx\_cmd\_assembler}] (txcmd) at (ctxcmd) {};
        \node[entity, label={uft\_tx\_data\_assembler}] (txdat) at (ctxdat) {};
        \node[entity, label={uft\_tx\_arbiter}] (txarb) at (ctxarb) {};

        \node[entityold, label={axi\_master\_burst}] (ambrx) at (crxamb) {};
        \node[entityold, label={axi\_master\_burst}] (ambtx) at (ctxamb) {};

        % ports
        \path[draw,{Latex[length=2.5mm]}-] ($(rx.180) + (0,1/6)$) -- ($(rx.180) + (-1.0,1/6)$) node[anchor=east] {rx\_hdr};
        \path[draw=blue,line width=0.5mm,{Latex[length=2.5mm]}-] ($(rx.180) + (0,-1/6)$) -- ($(rx.180) + (-1.0,-1/6)$) node[anchor=east] {s\_axi};
        \path[draw,{Latex[length=2.5mm]}-] ($(txctl.180) + (0,0)$) -- ($(txctl.180) + (-1.0,0)$) node[anchor=east] {controll};

        \path[draw=blue,line width=0.5mm,-{Latex[length=2.5mm]}] ($(rxmem.0) + (0,0)$) -- ($(rxmem.0) + (6.5,0/6)$) node[anchor=west] {s\_axi\_rx};
        \path[draw,-{Latex[length=2.5mm]}] ($(txctl.0) + (0,3.5/10)$) -- ($(txctl.0) + (11.5,3.5/10)$) node[anchor=west] {tx\_hdr};
        \path[draw=red,line width=0.5mm,-{Latex[length=2.5mm]}] ($(txarb.0) + (0,0/10)$) -- ($(txarb.0) + (1.5,0/10)$) node[anchor=west] {s\_axi};

        % Interconnects
        \path[draw,-{Latex[length=2.5mm]}] ($(rx.0) + (0,1/6)$) -- ($(rxmem.180) + (0,1/6)$) node[anchor=east] {};
        \path[draw=blue,line width=0.5mm,-{Latex[length=2.5mm]}] ($(rx.0) + (0,-1/6)$) -- ($(rxmem.180) + (0,-1/6)$) node[midway, anchor=north] {s\_axi};
        % \node at ($(rx.180) + (-0.5,-1/6)$) [circle,fill,inner sep=1.5pt]{};

        \path[draw,-{Latex[length=2.5mm]}] ($(txctl.0) + (0,1.5/10)$) -| ($(txcmd.180) + (-0.5,0)$) -- ($(txcmd.180) + (0,0)$) node[anchor=west] {};
        \path[draw,-{Latex[length=2.5mm]}] ($(txctl.0) + (0,-1.5/10)$) -| ($(txarb.180) + (-6,0)$) -- ($(txarb.180) + (0,0)$) node[anchor=west] {};
        \path[draw,-{Latex[length=2.5mm]}] ($(txctl.0) + (0,-3.5/10)$) -| ($(txdat.180) + (-1.5,1/6)$) -- ($(txdat.180) + (0,1/6)$) node[anchor=west] {};

        \path[draw,-{Latex[length=2.5mm]}] ($(txcmd.0) + (0,0)$) -| node[anchor=south] {s\_axi} ($(txarb.180) + (-0.5,1/4)$)  -- ($(txarb.180) + (0,1/4)$);
        \path[draw=red,line width=0.5mm,-{Latex[length=2.5mm]}] ($(txdat.0) + (0,0)$) -| node[anchor=north] {s\_axi}($(txarb.180) + (-0.5,-1/4)$) -- ($(txarb.180) + (0,-1/4)$);

        \path[draw=red,line width=0.5mm,-{Latex[length=2.5mm]}] ($(txdat.180) + (-6,-1/6)$) node[anchor=east] {s\_axi\_tx} -- ($(txdat.180) + (0,-1/6)$) node[anchor=east] {};

        % Ack
        \path[draw,-{Latex[length=2.5mm]}] ($(crxmem.0) + (1.5,-1)$) |- ($(crxmem.0) + (-1,-2)$) -| ($(ctxctl.0) + (2.5,0)$) node[midway, anchor=east] {};


    \end{pgfonlayer}

    % tx box
    \begin{pgfonlayer}{foreground}
        \node [draw, fill=gray!20, inner sep=10, fit={(ltxctl) (txctl) (txcmd) (txdat) (txarb) (txcal)}, label={[label distance=0.0cm]150:uft\_tx\_top}] (tx) {};
    \end{pgfonlayer} 

    % Board box
    \begin{pgfonlayer}{background}
        \node [draw, fill=gray!40, inner sep=10, fit={(tx) (rx) (rxmem) (rxl)}, label=uft\_top] (tx) {};
    \end{pgfonlayer} 

\end{tikzpicture}
    \end{adjustbox}
    \caption{UFT Top Block Design with AXI4-Stream interface}
    \label{fig:ufttopstream}
\end{figure}

% ==============================================================================
%                             Implemented Features
% ==============================================================================
\clearpage
\subsection{Implemented Features}
In the final version of the UFT core used in solution B the AXI4-Lite interface
was removed because no memory addresses had to be exchanged and the solution B
controller was written in VHDL. Adding an AXI4-Lite master interface would have
only made it more complicated. 
\\

To conclude the changes made to the UFT core from the version of project 5:
\begin{itemize}
  \item Added acknowledgment on receiver path
  \item Changed data interfaces from memory mapped to AXI4-Stream
  \item Added user register to exchange configuration data
  \item Bug fixes
\end{itemize}

There are still some features that are not yet implemented but are defined in
the UFT protocol specifications:
\begin{itemize}
  \item Acknowledge check on transmit
  \item Retransmission during send
\end{itemize}

Figure \ref{fig:uftipcoreaxistream} shows the IP core with the AXI4-Stream
ports.

\begin{figure}[h!]
    \centering
    \includegraphics[width=0.45\textwidth] {images/dataflow/uftcorestream.png}
    \caption{UFT IP core with AXI4-Stream interface}
    \label{fig:uftipcoreaxistream}
\end{figure}
% ==============================================================================
%                             Control
% ==============================================================================
\section{Control} \label{ch:control}
Now that the image data is received by the UFT core it has to be sent to the
Wallis filter in the right order. Reordering the pixel data is the main task of
the controller besides caching data and controlling the UFT and Wallis cores.
This chapter describes the tasks to be solved.

As already mentioned at the beginning of chapter \ref{chapt:dataflow}, two
implementations were realized. Solution A was implemented using Vivado HLS and
is documented in chapter \ref{ch:controller:hls} followed by solution B written
in VHDL and described in detail in chapter \ref{ch:controller:vhdl}. 
% The problem is
% stated in the following chapter.

% ==============================================================================
%                             Concept
% ==============================================================================
\subsection{Concept} \label{ch:control:concept}
Image data coming from the PC is sent row wise with the left most pixel sent
first. This is due to the fact that \gls{opencv} (which is used on PC
side to access
images) stores the image data in said layout \cite{opencv_structures}. The
Wallis filter however, requires its input data to be coloumn wise with the top
most pixel sent first as described in chapter \ref{ch:ip:concept}. Figure 
\ref{fig:memproblem} illustrates this issue. The blue ``write'' arrow shows the data
comming from the UFT core and the red ``read'' arrow represents the order the
data has to be sent to the Wallis filter. For illustration purposes a window
length of five and image width of eight is used.

\begin{figure}[h!]
    \centering
    \begin{adjustbox}{scale=0.7}
        % \tikzsetnextfilename{system-overview}
\begin{tikzpicture}[
    rounded corners=0mm,
    triangle/.style = {fill=blue!20, regular polygon, regular polygon sides=3 },
    node rotated/.style = {rotate=180},
    border rotated/.style = {shape border rotate=180}
]
    %coordinates
    \coordinate (orig)      at (0,0);

    \begin{pgfonlayer}{main}
        
        % Write arrows
        % \draw[draw=blue,line width=1.5mm] (8,4.5) .. controls (8,4) and (-1,4) .. (-1,3.5);
        % \path[draw=blue,line width=1.5mm] ($(-1,3.5)$) -- ($(8,3.5)$) node[anchor=east] {};

        % Write path
        \path[draw={rgb:red,1;green,2;blue,3},line width=1.0mm] ($(-2,4.5)$)  -- ($(9,4.5)$);
        \path[draw={rgb:red,1;green,2;blue,3},line width=1.0mm]  (-1,3.5) -- ($(9,3.5)$) ;
        \path[draw={rgb:red,1;green,2;blue,3},line width=1.0mm]  (-1,2.5) -- ($(9,2.5)$) ;
        \path[draw={rgb:red,1;green,2;blue,3},line width=1.0mm]  (-1,1.5) -- ($(9,1.5)$) ;
        \path[draw={rgb:red,1;green,2;blue,3},line width=1.0mm]  (-1,0.5) -- ($(9,0.5)$);
        \path[draw={rgb:red,1;green,2;blue,3},line width=1.0mm,dashed] ($(9,4.5)$)  .. controls (9,4) and (-1,4) .. (-1,3.5);
        \path[draw={rgb:red,1;green,2;blue,3},line width=1.0mm,dashed] ($(9,3.5)$)  .. controls (9,3) and (-1,3) .. (-1,2.5);
        \path[draw={rgb:red,1;green,2;blue,3},line width=1.0mm,dashed] ($(9,2.5)$)  .. controls (9,2) and (-1,2) .. (-1,1.5);
        \path[draw={rgb:red,1;green,2;blue,3},line width=1.0mm,dashed] ($(9,1.5)$)  .. controls (9,1) and (-1,1) .. (-1,0.5);
        % Write triangles
        \node[triangle,shape border rotate=270, fill={rgb:red,1;green,2;blue,3},minimum size=0.1cm] at (-1,0.5) {};
        \node[triangle,shape border rotate=270, fill={rgb:red,1;green,2;blue,3},minimum size=0.1cm] at (-1,1.5) {};
        \node[triangle,shape border rotate=270, fill={rgb:red,1;green,2;blue,3},minimum size=0.1cm] at (-1,2.5) {};
        \node[triangle,shape border rotate=270, fill={rgb:red,1;green,2;blue,3},minimum size=0.1cm] at (-1,3.5) {};
        \node[triangle,shape border rotate=270, fill={rgb:red,1;green,2;blue,3},minimum size=0.1cm] at (-1,4.5) {};
        
        % Read path
        \path[draw={rgb:red,3;green,1;blue,2},line width=1.0mm]  (0.5,6)  -- (0.5,-0.5);
        \path[draw={rgb:red,3;green,1;blue,2},line width=1.0mm]  (1.5,5.5)  -- (1.5,-0.5);
        \path[draw={rgb:red,3;green,1;blue,2},line width=1.0mm]  (2.5,5.5)  -- (2.5,-0.5);
        \path[draw={rgb:red,3;green,1;blue,2},line width=1.0mm]  (3.5,5.5)  -- (3.5,-0.5);
        \path[draw={rgb:red,3;green,1;blue,2},line width=1.0mm]  (4.5,5.5)  -- (4.5,-0.5);

        \path[draw={rgb:red,3;green,1;blue,2},line width=1.0mm,dashed] (0.5,-0.5)  .. controls (1,-0.5) and (1,5.5) .. (1.5,5.5);
        \path[draw={rgb:red,3;green,1;blue,2},line width=1.0mm,dashed] (1.5,-0.5)  .. controls (2,-0.5) and (2,5.5) .. (2.5,5.5);
        \path[draw={rgb:red,3;green,1;blue,2},line width=1.0mm,dashed] (2.5,-0.5)  .. controls (3,-0.5) and (3,5.5) .. (3.5,5.5);
        \path[draw={rgb:red,3;green,1;blue,2},line width=1.0mm,dashed] (3.5,-0.5)  .. controls (4,-0.5) and (4,5.5) .. (4.5,5.5);
        % Read triangles
        \node[triangle, border rotated, fill={rgb:red,3;green,1;blue,2},minimum size=0.1cm] at (0.5,5.5) {};
        \node[triangle, border rotated, fill={rgb:red,3;green,1;blue,2},minimum size=0.1cm] at (1.5,5.5) {};
        \node[triangle, border rotated, fill={rgb:red,3;green,1;blue,2},minimum size=0.1cm] at (2.5,5.5) {};
        \node[triangle, border rotated, fill={rgb:red,3;green,1;blue,2},minimum size=0.1cm] at (3.5,5.5) {};
        \node[triangle, border rotated, fill={rgb:red,3;green,1;blue,2},minimum size=0.1cm] at (4.5,5.5) {};

        % Text
        \node[] (write) at (-2,5) {Write};
        \node[] (read) at (0,6.2) {Read};

        % Braces
        \draw [line width=0.5mm,decorate,decoration={brace,amplitude=10pt},xshift=-4pt,yshift=0pt] (9.5,5) -- (9.5,0) node [black,midway,xshift=0.5cm,anchor=west] {Window length};
        \draw [line width=0.5mm,decorate,decoration={brace,amplitude=10pt},xshift=-0pt,yshift=0pt] (8,-0.5) -- (0,-0.5) node [black,midway,yshift=-0.5cm,anchor=north] {Image width};
        
        % Center pixel
        \draw[black,line width=0.5mm] (2,2) rectangle (3,3);
        % Window size
        \draw[black,line width=0.5mm] (0,0) rectangle (5,5);
        % Axis
        \foreach \x in {0,1,2,3,4}
            \node[anchor=north] at ($(-0.5,5)-(0,\x)$)  {$\x$};

    \end{pgfonlayer}

    % Foreground
    \begin{pgfonlayer}{foreground}
        
    \end{pgfonlayer} 

    % Background
    \begin{pgfonlayer}{background}
        % Grid
        \draw[step=1cm,gray,very thin] (0,0) grid (8,5);
    \end{pgfonlayer} 

\end{tikzpicture}
    \end{adjustbox}
    \caption{Memory reordering problem}
    \label{fig:memproblem}
\end{figure}
\clearpage
This is only one part of the problem. Another issue arises if the
progression of the neighborhood across the input image is observed. We
distinguish two scenarios: A) moving the neighborhood horizontally to the right
and B) advancing the neighborhood to the next row vertically. The first
scenario is illustrated in figure \ref{fig:memproblemgrowthx}. The green arrow
marks the new coloumn of the neighborhood to be sent to the Wallis filter.

\begin{figure}[t!]
    \centering
    \begin{adjustbox}{scale=0.7}
        \input{images/controller/memproblemgrowthx.tikz}
    \end{adjustbox}
    \caption{Scenario A) Data growth in horizontal direction}
    \label{fig:memproblemgrowthx}
\end{figure}

This progression is made for each coloumn of the input image until the last
coloumn was processed. Then scenario B) comes to play. Moving the neighborhood
vertically to the next row by one pixel requires new image data of one row
and $WINDOW\_LENGTH-1$ rows of image data that had already been processed on the
previous row. Figure \ref{fig:memproblemgrowthy} illustrates scenario B.

\begin{figure}[h!]
    \centering
    \begin{adjustbox}{scale=0.7}
        \input{images/controller/memproblemgrowthy.tikz}
    \end{adjustbox}
    \caption{Scenario B) Data growth in vertical direction}
    \label{fig:memproblemgrowthy}
\end{figure}

\pagebreak

From these two scenarios we can conclude two problems the controller has to
solve:
\begin{itemize}
    \item Send the image data in the right order (coloumn wise with
    $WINDOW\_LENGTH$ sized coloumns)
    \item Cache $WINDOW\_LENGTH-1$ rows for the computation of the next image
    row
\end{itemize}

In addition the controller has to start the UFT transmission if an image row has
been processed and configure the Wallis filter with the parameters comming from
the UFT core.



% ==============================================================================
%                             HLS
% ==============================================================================
\subsection{Implementation HLS} \label{ch:controller:hls}
Now that the problems are analysed, the proceeding can be stated. For the first
solution an approach using Vivado HLS was picked. The main motivations for this
decision were the fact that we had learned how fast we can have a working IP
core
using the Vivado HLS workflow and that we wanted to test the ability to
implement a state machine in Vivado HLS. In the following chapters the
requirements for this state machine are listed, a brief insight into the source
code is given and the main hurdles while developing the core are unfolded.

\subsubsection*{Requirements}
Besides the dataflow described in \ref{ch:control:concept} the IP core
interfaces must be defined. Table \ref{tab:controlleraports} lists the IP
interfaces as defined in \texttt{controller.cpp}.
The IP core implements a finite state machine as seen in figure 
\ref{fig:controllerfsm}. After the initial state
\texttt{INIT} where the UFT core is initialized, the state machine goes to its
idle state \texttt{IDLE}. If the UFT core indicates the end of a file
transfer the image width is stored and the state machine switches to the 
\texttt{READ} state. There it fills the first junk of its buffers and
switches to the \texttt{STREAM} state. Now the data is sent from the
internal buffers to the Wallis filter in the correct order. Simultaneously the
processed pixels from the Wallis filter are stored inside a memory. 

\begin{table}[b!]
    \centering
    \begin{tabular}{l l l p{8cm}}
        \toprule
        Variable & Type & Connection to & Description \\
        \midrule
        \texttt{memp} & \texttt{AXI4 Master} & Memory &
        Access to the block memory where the image data is stored
        \\
        \texttt{cbus} & \texttt{AXI4 Master} \footnotemark & UFT core &
        Control the AXI4-Lite slave registers to control the UFT core
        \\
        \texttt{inData} & \texttt{AXI4-Stream} & Wallis filter &
        Processed pixels comming from the Wallis filter
        \\
        \texttt{outData} & \texttt{AXI4-Stream} & Wallis filter &
        Image data sent to the Wallis filter for processing
        \\
        \texttt{rx\_done} & \texttt{ap\_uint<1>} & UFT core &
        Is asserted after a file transfer is complete
        \\
        \texttt{tx\_ready} & \texttt{ap\_uint<1>} & UFT core &
        Is asserted if UFT core is ready to send
        \\
        \bottomrule
    \end{tabular}
    \caption{Controller solution A interface ports}
    \label{tab:controlleraports}
\end{table}
\footnotetext{Vivado HLS does not support AXI4-Lite master, but a AXI4 can
        be converted to AXI4-Lite}

\begin{figure}[t!]
    \centering
    \begin{adjustbox}{scale=1}
        % \tikzsetnextfilename{system-overview}
\begin{tikzpicture}[->,>=stealth',shorten >=1pt,auto,node distance=2.8cm,
                    semithick]

    \tikzstyle{every state}=[fill=white,draw=black,text=black,minimum width=2.2cm]
    
    % states
    \node[initial,state] (A)                    {INIT};
    \node[state]         (B) [right of=A]       {IDLE};
    \node[state]         (C) [above right of=B] {READ};
    \node[state]         (D) [right of=C]       {STREAM};
    \node[state]         (E) [below right of=D]       {WRITE};
    \node[state, align=center]         (F) [below left of=E]       {WAIT\_TO\_ \\ SEND};
    \node[state]         (G) [left of=F]       {SEND};


    % path
    \path   (A) edge              node {} (B)
            (B) edge              node {} (C)
            (C) edge              node {} (D)
            (D) edge              node {} (E)
            (E) edge              node {} (F)
            (F) edge              node {} (G)
            (G) edge              node {} (B);
    \begin{pgfonlayer}{main}

    \end{pgfonlayer}

    \begin{pgfonlayer}{foreground}
    
    \end{pgfonlayer} 


\end{tikzpicture}
    \end{adjustbox}
    \caption{Controller solution A) simplified state machine}
    \label{fig:controllerfsm}
\end{figure}

If all
pixels in one line are processed the state \texttt{WRITE} is activated. The
processed pixels are stored in block memory using the \texttt{memp} AXI4 Master
interface and the state machine switches to the \texttt{WAIT\_TO\_SEND}
state. If the UFT core Tx is ready, the state \texttt{SEND} is activated
where the UFT core is configured and a transmission is started.

\subsubsection*{Realization}
According to Xilinx application note XAPP-1209 \cite{xapp1209} a state
machine in Vivado HLS can be realized by using a switch statement in the C/C++
code.
The code in file \texttt{controller.cpp} implements the state machine behaviour.
The state machine code is very straight foreward. More work went into the memory
layout. Because the memory bus to read and write pixels to and from memory is
AXI4 based, multiple single byte access result in considerably less throughput
than single burst access.
To take advantage of these burst accesses, a ping-pong buffer structure was
realized. There are two buffers, each of the size of $WINDOW\_LENGTH *
AXI\_BURST\_SIZE$ bytes, where $AXI\_BURST\_SIZE$ represents the number of bytes
read
in one burst access. It requires $WINDOW\_LENGTH$ burst reads to fill one
buffer. If one buffer is filled, it can be accessed by single byte access to
read the data in the required order for the Wallis filter, as shown in figure
\ref{fig:memproblem}. During the time the data is read from one buffer, the
other buffer is filled with image data from the AXI4 bus, hence the name
ping-pong buffer. 

Figure \ref{fig:solamemlayout} shows the memory layout of
solution A. The blue and red rectangles represent the two buffers. Visualized is
an AXI burst length of eight and a window length of five.

\clearpage

\begin{figure}[tb!]
    \centering
    \begin{adjustbox}{scale=0.7}
        % \tikzsetnextfilename{system-overview}
\begin{tikzpicture}[
    rounded corners=0mm,
    triangle/.style = {fill=blue!20, regular polygon, regular polygon sides=3 },
    node rotated/.style = {rotate=180},
    border rotated/.style = {shape border rotate=180}
]
    %coordinates
    \coordinate (orig)      at (0,0);

    \begin{pgfonlayer}{main}
        
        % % Write path
        \path[draw={rgb:red,1;green,2;blue,3},-{Latex[length=5mm]},line width=1.0mm] (0.5,4.5)  -- (7.5,4.5);
        \path[draw={rgb:red,3;green,1;blue,2},-{Latex[length=5mm]},line width=1.0mm] (8.5,4.5)  -- (15.5,4.5);
        
        % % Text
        % \node[] (write) at (-2,5) {Write};

        % Braces
        \draw [line width=0.5mm,decorate,decoration={brace,amplitude=10pt},xshift=-4pt,yshift=0pt] (17.5,5) -- (17.5,0) node [black,midway,xshift=0.5cm,anchor=west] {Window length};
        \draw [line width=0.5mm,decorate,decoration={brace,amplitude=10pt},xshift=-0pt,yshift=0pt] (8,-0.5) -- (0,-0.5) node [black,midway,yshift=-0.5cm,anchor=north] {AXI burst length};
        
        % Center pixel
        \draw[black,line width=0.5mm] (2,2) rectangle (3,3);
        % Window size
        \draw[black,line width=0.5mm] (0,0) rectangle (5,5);
        % Buffer A
        \draw[draw={rgb:red,1;green,2;blue,3},line width=1mm] (-0.05,-0.05) rectangle (7.95,5.05);
        % Buffer B
        \draw[draw={rgb:red,3;green,1;blue,2},line width=1mm] (8.05,-0.05) rectangle (16.05,5.05);
        % Axis
        \foreach \x in {0,1,2,3,4}
            \node[anchor=north] at ($(-0.5,5)-(0,\x)$)  {$\x$};
        % Axis
        \foreach \x in {0,1,2,3,4,5,6,7,8,9,10,11,12,13,14,15,16}
            \node[anchor=south] at ($(0.1,5.1)+(\x,0)$)  {$\x$};

    \end{pgfonlayer}

    % Foreground
    \begin{pgfonlayer}{foreground}
        
    \end{pgfonlayer} 

    % Background
    \begin{pgfonlayer}{background}
        % Grid
        \draw[step=1cm,black,thin] (0,0) grid (17,5);
    \end{pgfonlayer} 

\end{tikzpicture}
    \end{adjustbox}
    \caption{Memory layout for solution A}
    \label{fig:solamemlayout}
\end{figure}

Filling the buffer is done in the \texttt{fillBuff} routine
as shown on listing \ref{lst:buf_fill}.

\begin{minipage}{\textwidth}
\begin{lstlisting}[style=CStyle, caption=Buffer fill simplified,
label=lst:buf_fill]
uint32_t i, inOff, outOff;
for(i = 0; i < WINDOW_LEN; i++) {
    inOff = off + i*imgWidth;
    outOff = i*AXI_BURST_SIZE;
    memcpy(&buf[outOff], &memp[inOff], AXI_BURST_SIZE);
}\end{lstlisting}
\end{minipage}

To send the right order of output pixels, two counters are required. The first
counter named \texttt{ms\_rctr} increments after every pixel sent and wraps at
$WINDOW\_LENGTH$. Therefore it counts the current coloumn. The second counter
named \texttt{ms\_cctr} increments every time the \texttt{ms\_rctr} wrappes
around and counts up to $IMAGE\_WIDTH$. From these two counters the address of
the pixel to be sent can be calculated as shown in listing \ref{lst:aaddrcalc}.

\begin{minipage}{\textwidth}
\begin{lstlisting}[style=CStyle, caption=Pixel send address calculation,
label=lst:aaddrcalc]
oPxl.data = outPpBuf[AXI_BURST_SIZE*(ms_rctr++) + ms_cctr];\end{lstlisting}
\end{minipage}

Receiving a pixel from the Wallis core is simply storing the pixel at the next
memory location of the
output buffer. Listing \ref{lst:runinstream} shows the receiveing of processed
pixels.

\begin{minipage}{\textwidth}
\begin{lstlisting}[style=CStyle, caption=Pixel read store,
label=lst:runinstream]
iPxl = inData.read();
out_mem[sm_ctr++] = (uint8_t)iPxl.data;\end{lstlisting}
\end{minipage}

Everytime a pixel is read or written a corresponding counter is incremented.
After both counter hit their compare value, the stream state is exited and the
data is copied back to memory.

\pagebreak
\subsubsection*{Hurdles}
The C/C++ code was written in little time. With less than 300 lines it
is a manageable amount of code.
The C/C++ and RTL-simulation yielded no errors but the
implemented core together with the Wallis filter core did not work as expected.
Thanks to the \gls{ila} the signals connecting the controller
and Wallis core could be inspected. Soon it was discovered that the controller
only output one pixel and then stopped. It was discovered that this was due to
the fact that a \texttt{read()} operation on a AXI4-Stream in Vivado HLS is
blocking, meaning that the synthesis tool implements the code in a  manner that
if no data is
valid on the stream, the code will wait until there is data on the stream. This
led to the code waiting for an input byte and no output was generated. This
issue was fixed by wrapping the read access with a query weather the input
stream holds data. Listing \ref{lst:runinstream2} shows the adjusted pixel read.
The same applied to the sending of data. It is first checked wheather the output
stream is not full before sending new data.

\begin{minipage}{\textwidth}
\begin{lstlisting}[style=CStyle, caption=Pixel read store with query,
label=lst:runinstream2]
if(!inData.empty()) {
    iPxl = inData.read();
    out_mem[sm_ctr++] = (uint8_t)iPxl.data;
}\end{lstlisting}
\end{minipage}

% \clearpage
\subsubsection*{Conclusion}
To conclude the findings using a Vivado HLS based approach, the thesis that
with little effort a working solution can be produced is confirmed. The required
interfaces (AXI4-Master and AXI4-Stream) were implemented with a single line of
code and the statemachine with a simple switch statement. After
clearing some hurdles the controller core worked as expected with two drawbacks:

\vspace{1ex}
\textbf{1. Throughput:} The controller core outputs data at a rate of one byte
every 9 clock cycles as can be seen in the result of the simulation.
In theory, using a block memory as internal buffer, it
should be possible to output one byte per clock cycle. This issue could not be
resolved in a forseeable time. The analysis view of Vivado HLS would give
indications on what calculation is resulting in a reduced throughput but the
analysis results could not be interpreted.

% \begin{figure}[tb!]
%     \centering
%     \includegraphics[width=0.6\textwidth]{images/controller/hlscontrolleroutput.png}
%     \caption{Controller solution A) output data}
%     \label{fig:shlowhlsoutput}
% \end{figure}

\vspace{1ex}
\textbf{2. Memory layout:} In solution A the image data is stored multiple time
as already mentioned in chapter \ref{ch:concept}. The data is stored twice in
the controller core alone: during read in the ping-pong buffer and during write
in the output buffer. This increases the ressource usage.

\vspace{1ex}
These two issues were the motivation to rethink the memory layout and
functionality of the controller core which led to solution B, a controller
implemented in VHDL.

% \subsubsection*{IF-Statement} \label{ch:data:if}

% \begin{minipage}{\textwidth}
% \begin{lstlisting}[style=CStyle, caption=Buffer switching reloading without else statement, label=lst:buf_false]
% if( (outPpBuf == ppBufA) && (ms_pctr < (inLineSize-PIN_PONG_BUF_SIZE)) ) {
%     if(!ppBufBrdy) {
%         ...
%     }
% }
% if( (outPpBuf == ppBufB) && (ms_pctr < (inLineSize-PIN_PONG_BUF_SIZE)) ) {
%     if(!ppBufArdy) {
%         ...
%     }
% }
% \end{lstlisting}
% \end{minipage}

% \begin{minipage}{\textwidth}
% \begin{lstlisting}[style=CStyle, caption=Buffer switching reloading with else statement, label=lst:buf_right]
% if (ms_pctr < (inLineSize-PIN_PONG_BUF_SIZE)) {
% 	if( (outPpBuf == ppBufA)  ) {
% 		if(!ppBufBrdy) {
% 			...
% 		}
% 	}
% 	else {
% 		if(!ppBufArdy) {
% 			...
% 		}
% 	}
% }
% \end{lstlisting}
% \end{minipage}

% ==============================================================================
%                             VHDL
% ==============================================================================
\clearpage
\subsection{Implementation VHDL} \label{ch:controller:vhdl}
In a new implementation of the controller, the two drawbacks of solution A are
addressed. The following chapter is structered likewise starting with the new
requirements, how the core was realized, the hurdles accross the way and a
conclusion of the performance of the new controller core.

The Wallis filter core was built from ground up with AXI4-Stream interfaces. A
slave interface for input data and a master interface for output data. This
stream based approach should be implemented in the controller as well to reduce
buffering to a minimum to decrease memory usage. The use of streaming interfaces
required a new memory structure. These two aspects form the requierements of the
new controller core.

\subsubsection*{Requirements}
With the new stream based approach the UFT communication core first had to be
altered to support stream interfaces. This is explained in chapter 
\ref{ch:data:com:datainterface}. With the AXI4-Lite configuration interface
removed, the
control signals are replaced with arbitrary signals. The new interface
definition is dissected in table \ref{tab:controllerbports}.

Furthermore the order of pixel to be sent to the Wallis filter remain the same
as described in chapter \ref{ch:control:concept}. A stream based approach also eliminates the need for a statemachine. The entire
controller is to be stateless.

\begin{table}[h!]
    \centering
    \begin{tabular}{l l l p{8cm}}
        \toprule
        Name & Type & To & Description \\
        \midrule
        %%% UFT Rx
        \texttt{uft\_i\_axis} & \texttt{AXI4-Stream} & UFT Rx &
        Receive data stream from UFT core
        \\

        \texttt{uft\_rx\_done} & \texttt{arb} & UFT Rx &
        Signals a complete UFT transfer
        \\  
        % \texttt{uft\_rx\_row\_num} & \texttt{arb} & UFT Rx &
        % asdf
        % \\  
        % \texttt{uft\_rx\_row\_num\_valid} & \texttt{arb} & UFT Rx &
        % asdf
        % \\  
        % \texttt{uft\_rx\_row\_size} & \texttt{arb} & UFT Rx &
        % asdf
        % \\  
        % \texttt{uft\_rx\_row\_size\_valid} & \texttt{arb} & UFT Rx &
        % asdf
        % \\  
        \texttt{uft\_user\_regX} & \texttt{arb} & UFT Rx &
        UFT user register values. They can be set from PC and serve as channel
        for Wallis parameters
        \\  
        \midrule
        %%% UFT Tx
        \texttt{uft\_o\_axis} & \texttt{AXI4-Stream} & UFT Tx &
        Transmit data stream to UFT core
        \\
        \texttt{uft\_tx\_start} & \texttt{arb} & UFT Tx &
        Asserted to initialize a transmission
        \\
        \texttt{uft\_tx\_ready} & \texttt{arb} & UFT Tx &
        Signals that the UFT core is ready to transmit
        \\
        % \texttt{uft\_tx\_row\_num} & \texttt{arb} & UFT Tx &
        % asdf
        % \\
        \texttt{uft\_tx\_data\_size} & \texttt{arb} & UFT Tx &
        How many bytes to transmit
        \\
        \midrule
        %%% Wallis
        \texttt{wa\_i\_axis} & \texttt{AXI4-Stream} & Wallis &
        Image input data to the Wallis core
        \\
        \texttt{wa\_o\_axis} & \texttt{AXI4-Stream} & Wallis &
        Processed image data from the Wallis core
        \\
        \texttt{wa\_par} & \texttt{arb} & Wallis &
        Parameters for the Wallis filter
        \\
        \bottomrule
    \end{tabular}
    \caption{Controller solution B interface ports (unlisted interfaces
    are not used)}
    \label{tab:controllerbports}
\end{table}

\clearpage
\subsubsection*{Realization}
The controller core consists of three entities joined in a top entity. Figure 
\ref{fig:dctop} shows a block diagram of the controller. The 
\texttt{dc\_control} entity takes control of housekeeping such as initiating a
UFT transmission after complete processing and resetting all instances on a new
input image. The processed pixels are buffered in the \texttt{axis\_fifo}. This
is necessary because if an Ethernet transmission is started, every clock a byte
has to be valid. The third entity, \texttt{dc\_mmu} is where the image data is
cached and sent to the Wallis filter in the right order, hence the name memory
management unit.
\\

The memory management unit works similar to a large FiFo with the addition that
elements can be read multiple times. To explain its functionality, the operation
of a simple FiFo has to be clarified. A FiFo has three main components: the
memory, a read pointer and a write pointer. Figure \ref{fig:fifo} shows the
structure of a FiFo. Five states can be identified. State \rom{1} is the initial
state after reset, both read and write pointers point to the first element in
the array. If a write occurs, the write pointer advances which is observed in
state \rom{2}. The filled elements in the array represent values that can be
read. As long as the write pointer is ahead of the read pointer, data can be
read. The next state appears when the write pointer wraps around the length of
the memory to the first element as shown in state \rom{3}. After several read
accesses the read pointer wrappes around as well and we can observe state 
\rom{4} which is equivalent to state \rom{2}. If the read pointer caught up to
the write pointer state \rom{5} is observed.

\begin{figure}[t!]
    \centering
    \begin{adjustbox}{max width=1\textwidth}
        % \tikzsetnextfilename{system-overview}
\begin{tikzpicture}[
    rounded corners=0mm,
    entity/.style={
        draw,
        minimum height=1.0cm,
        minimum width=3cm,
        fill=white,
        anchor=north west,
    },
]
    %coordinates
    \coordinate (cctl)      at (0,4);
    \coordinate (cmmu)      at (1,2);
    \coordinate (cfifo)     at (1,0);


    %nodes

    \begin{pgfonlayer}{main}
        % entities
        \node[entity, label={dc\_control}] (ctl) at (cctl) {};
        \node[entity, label={dc\_mmu}] (mmu) at (cmmu) {};
        \node[entity, label={axis\_fifo}] (fifo) at (cfifo) {};

        % % ports
        \path[draw,{Latex[length=2.5mm]}-{Latex[length=2.5mm]}] ($(ctl.180) + (0,0)$) -- ($(ctl.180) + (-2.0,0)$) node[anchor=east] {uft control};

        \path[draw,{Latex[length=2.5mm]}-{Latex[length=2.5mm]}] ($(ctl.0) + (0,0)$) -- ($(ctl.0) + (3.0,0)$) node[anchor=west] {wallis control};
        \path[draw,-{Latex[length=2.5mm]}] ($(mmu.0) + (0,0)$) -- ($(mmu.0) + (2.0,0)$) node[anchor=west] {wallis data in};
        \path[draw,{Latex[length=2.5mm]}-] ($(mmu.180) + (0,-1/6)$) -- ($(mmu.180) + (-3,-1/6)$) node[anchor=east] {uft rx};
        \path[draw,-{Latex[length=2.5mm]}] ($(fifo.180) + (0,-1/6)$) -- ($(fifo.180) + (-3,-1/6)$) node[anchor=east] {uft tx};
        \path[draw,{Latex[length=2.5mm]}-] ($(fifo.0) + (0,0)$) -- ($(fifo.0) + (2,0)$) node[anchor=west] {wallis data out};


        % % Interconnects
        \path[draw,-{Latex[length=2.5mm]}] ($(ctl.270) + (-1.3,0)$) |- ($(fifo.180) + (0,1/6)$) node[anchor=east] {};
        \path[draw,-{Latex[length=2.5mm]}] ($(ctl.270) + (-1.1,0)$) |- ($(mmu.180) + (0,1/6)$) node[anchor=east] {};
        

    \end{pgfonlayer}

    \begin{pgfonlayer}{foreground}
        
    \end{pgfonlayer} 

    % Board box
    \begin{pgfonlayer}{background}
        \node [draw, fill=gray!40, inner sep=15, fit={(ctl) (mmu) (fifo)}, label=dc\_top] (dctop) {};
    \end{pgfonlayer} 

\end{tikzpicture}
    \end{adjustbox}
    \caption{Controller solution B) block diagram}
    \label{fig:dctop}
\end{figure}

\clearpage

\begin{figure}[tb!]
    \centering
    \begin{adjustbox}{max width=\textwidth}
        % \tikzsetnextfilename{system-overview}
\begin{tikzpicture}[]

    % \coordinate (cff1)      at (0,3);
    % \coordinate (cff2)      at (4,3);
    % \coordinate (cff3)      at (8,3);
    % \coordinate (cff4)      at (2,0);
    % \coordinate (cff5)      at (6,0);
    \coordinate (cff1)      at (0,0);
    \coordinate (cff2)      at (3.5,0);
    \coordinate (cff3)      at (7,0);
    \coordinate (cff4)      at (10.5,0);
    \coordinate (cff5)      at (14,0);

    

    \begin{pgfonlayer}{main}
        
        %% State 1
        % Grid
        % 
        \foreach \x in {0,1,2,3,4,5}
            \filldraw[fill=white, draw=black] ($(cff1) + \x*(0.5,0)$) rectangle ($(cff1) + \x*(0.5,0) + (0.5,1)$);
        % Read and write pointer
        \path[draw,-{Latex[length=2.5mm]}] ($(cff1) + (0,0) + (0.25,1.5)$) -- ($(cff1) + (0,0) + (0.25,1)$) node[near start, anchor=east] {w};
        \path[draw,-{Latex[length=2.5mm]}] ($(cff1) + (0,0) + (0.25,-0.5)$) -- ($(cff1) + (0,0) + (0.25,0)$) node[near start, anchor=east] {r};
        % Filled elements


        
        %% State 2
        % Grid
        % 
        \foreach \x in {0,1,2,3,4,5}
            \filldraw[fill=white, draw=black] ($(cff2) + \x*(0.5,0)$) rectangle ($(cff2) + \x*(0.5,0) + (0.5,1)$);
        % Read and write pointer
        \path[draw,-{Latex[length=2.5mm]}] ($(cff2) + (1,0) + (0.25,1.5)$) -- ($(cff2) + (1,0) + (0.25,1)$) node[near start, anchor=east] {w};
        \path[draw,-{Latex[length=2.5mm]}] ($(cff2) + (0,0) + (0.25,-0.5)$) -- ($(cff2) + (0,0) + (0.25,0)$) node[near start, anchor=east] {r};
        % Filled elements
        \filldraw[fill=gray, draw=black] ($(cff2) + 0*(0.5,0)$) rectangle ($(cff2) + 0*(0.5,0) + (0.5,1)$);
        \filldraw[fill=gray, draw=black] ($(cff2) + 1*(0.5,0)$) rectangle ($(cff2) + 1*(0.5,0) + (0.5,1)$);

        
        %% State 3
        % Grid
        % 
        \foreach \x in {0,1,2,3,4,5}
            \filldraw[fill=white, draw=black] ($(cff3) + \x*(0.5,0)$) rectangle ($(cff3) + \x*(0.5,0) + (0.5,1)$);
        % Read and write pointer
        \path[draw,-{Latex[length=2.5mm]}] ($(cff3) + (0.5,0) + (0.25,1.5)$) -- ($(cff3) + (0.5,0) + (0.25,1)$) node[near start, anchor=east] {w};
        \path[draw,-{Latex[length=2.5mm]}] ($(cff3) + (1.5,0) + (0.25,-0.5)$) -- ($(cff3) + (1.5,0) + (0.25,0)$) node[near start, anchor=east] {r};
        % Filled elements
        \filldraw[fill=gray, draw=black] ($(cff3) + 0*(0.5,0)$) rectangle ($(cff3) + 0*(0.5,0) + (0.5,1)$);
        \filldraw[fill=gray, draw=black] ($(cff3) + 3*(0.5,0)$) rectangle ($(cff3) + 3*(0.5,0) + (0.5,1)$);
        \filldraw[fill=gray, draw=black] ($(cff3) + 4*(0.5,0)$) rectangle ($(cff3) + 4*(0.5,0) + (0.5,1)$);
        \filldraw[fill=gray, draw=black] ($(cff3) + 5*(0.5,0)$) rectangle ($(cff3) + 5*(0.5,0) + (0.5,1)$);


        
        %% State 4
        % Grid
        % 
        \foreach \x in {0,1,2,3,4,5}
            \filldraw[fill=white, draw=black] ($(cff4) + \x*(0.5,0)$) rectangle ($(cff4) + \x*(0.5,0) + (0.5,1)$);
        % Read and write pointer
        \path[draw,-{Latex[length=2.5mm]}] ($(cff4) + (1.5,0) + (0.25,1.5)$) -- ($(cff4) + (1.5,0) + (0.25,1)$) node[near start, anchor=east] {w};
        \path[draw,-{Latex[length=2.5mm]}] ($(cff4) + (0,0) + (0.25,-0.5)$) -- ($(cff4) + (0,0) + (0.25,0)$) node[near start, anchor=east] {r};
        % Filled elements
        \filldraw[fill=gray, draw=black] ($(cff4) + 0*(0.5,0)$) rectangle ($(cff4) + 0*(0.5,0) + (0.5,1)$);
        \filldraw[fill=gray, draw=black] ($(cff4) + 1*(0.5,0)$) rectangle ($(cff4) + 1*(0.5,0) + (0.5,1)$);
        \filldraw[fill=gray, draw=black] ($(cff4) + 2*(0.5,0)$) rectangle ($(cff4) + 2*(0.5,0) + (0.5,1)$);


        
        %% State 5
        % Grid
        % 
        \foreach \x in {0,1,2,3,4,5}
            \filldraw[fill=white, draw=black] ($(cff5) + \x*(0.5,0)$) rectangle ($(cff5) + \x*(0.5,0) + (0.5,1)$);
        % Read and write pointer
        \path[draw,-{Latex[length=2.5mm]}] ($(cff5) + (1.5,0) + (0.25,1.5)$) -- ($(cff5) + (1.5,0) + (0.25,1)$) node[near start, anchor=east] {w};
        \path[draw,-{Latex[length=2.5mm]}] ($(cff5) + (1.5,0) + (0.25,-0.5)$) -- ($(cff5) + (1.5,0) + (0.25,0)$) node[near start, anchor=east] {r};
        % Filled elements

        %% State names
        \node[align=right,anchor=south east] (s1) at ($(cff1) + (0,2.5)$) {State:}; 
        \node[align=center,anchor=south] (s1) at ($(cff1) + (1.5,2.5)$) {\rom{1}}; 
        \node[align=center,anchor=south] (s1) at ($(cff2) + (1.5,2.5)$) {\rom{2}}; 
        \node[align=center,anchor=south] (s1) at ($(cff3) + (1.5,2.5)$) {\rom{3}}; 
        \node[align=center,anchor=south] (s1) at ($(cff4) + (1.5,2.5)$) {\rom{4}}; 
        \node[align=center,anchor=south] (s1) at ($(cff5) + (1.5,2.5)$) {\rom{5}};  
        %% Looped
        \node[align=right,anchor=south east] (s1) at ($(cff1) + (0,2)$) {Looped:}; 
        \node[align=center,anchor=south] (s1) at ($(cff1) + (1.5,2)$) {false}; 
        \node[align=center,anchor=south] (s1) at ($(cff2) + (1.5,2)$) {false}; 
        \node[align=center,anchor=south] (s1) at ($(cff3) + (1.5,2)$) {true}; 
        \node[align=center,anchor=south] (s1) at ($(cff4) + (1.5,2)$) {false}; 
        \node[align=center,anchor=south] (s1) at ($(cff5) + (1.5,2)$) {false};        


        % \draw[step=1cm,gray,very thin] (0,0) grid (20,2);

        % \node[label={uft\_rx}] (rx) at (cff4) {};

        % \path[draw,-{Latex[length=2.5mm]}] ($(txctl.0) + (0,1.5/10)$) -| ($(txcmd.180) + (-0.5,0)$) -- ($(txcmd.180) + (0,0)$) node[anchor=west] {};
    \end{pgfonlayer}

    \begin{pgfonlayer}{foreground}
    
    \end{pgfonlayer} 


\end{tikzpicture}

    
    \end{adjustbox}
    \caption{FiFo structure}
    \label{fig:fifo}
\end{figure}

From these five states we can conclude conditions for when a read or write access is
permitted. To do this a variable named \texttt{looped} is introduced. It is set
if the write pointer wraps around (as seen in state \rom{3}) and clear if the
read pointer wraps (state \rom{4}). Listing \ref{lst:fiforeadcond} shows the
read condition.

\begin{minipage}{\linewidth}
    \begin{lstlisting}[
        style=VHDLStyle, 
        caption=FiFo read condition, 
        label=lst:fiforeadcond
        ]
if ((looped = true) or (w_ptr /= r_ptr)) then
    DataOut <= Memory(r_ptr);
end if;\end{lstlisting}
\end{minipage}

Similarly, listing \ref{lst:fifowritecond} shows the write condition.

\begin{minipage}{\linewidth}
    \begin{lstlisting}[
        style=VHDLStyle, 
        caption=FiFo write condition, 
        label=lst:fifowritecond
        ]
if ((looped = false) or (w_ptr /= r_ptr)) then
    Memory(w_ptr) := DataIn;
end if;\end{lstlisting}
\end{minipage}

Now that the mechanics of a FiFo are dissected, the memory layout of the
MMU can be illustrated. Figure \ref{fig:solbmemlayout} shows the memory layout.
At the core of the memory management unit is a two
dimensional array (it is implemented as a simple array in VHDL for better
synthesis results). Every line represents an input image line and every coloumn an
input
image coloumn. The idea is to have a full block RAM for each line which is why
from now on a line is called cache. The input image data is received line by
line. Each input line is stored in a cache. 

\clearpage
If at least $WINDOW\_LENGTH$ lines
are received, there is enough data in the memory to start a Wallis operation for
one input line. If one input line is processed the Wallis neighborhood
window slides one pixel downwards. Observed in the Y-axis (top to bottom) this
mechanism resembles
a FiFo with \texttt{cache\_w\_ptr} being the write pointer and 
\texttt{cache\_r\_ptr} representing the read pointer. The major difference to a
simple FiFo is that the distance between the read and write pointer may not
become zero because the Wallis filter requires $WINDOW\_LENGTH$ lines to process
data.

\begin{figure}[tb!]
    \centering
    \begin{adjustbox}{max width=0.95\textwidth}
        % \tikzsetnextfilename{system-overview}
\begin{tikzpicture}[
    rounded corners=0mm,
    triangle/.style = {fill=blue!20, regular polygon, regular polygon sides=3 },
    node rotated/.style = {rotate=180},
    border rotated/.style = {shape border rotate=180}
]
    %coordinates
    \coordinate (orig)      at (0,0);

    \begin{pgfonlayer}{main}
        
        % % Write path
        % \path[draw={rgb:red,1;green,2;blue,3},-{Latex[length=5mm]},line width=1.0mm] (0.5,4.5)  -- (7.5,4.5);
        % \path[draw={rgb:red,3;green,1;blue,2},-{Latex[length=5mm]},line width=1.0mm] (8.5,4.5)  -- (15.5,4.5);
        
        % % Text
        % \node[] (write) at (-2,5) {Write};

        % Braces
        \draw [line width=0.5mm,decorate,decoration={brace,amplitude=10pt},xshift=-4pt,yshift=0pt] (12.5,6) -- (12.5,0) node [black,midway,xshift=0.5cm,anchor=west] {Cache N Lines};
        \draw [line width=0.5mm,decorate,decoration={brace,amplitude=10pt},xshift=-0pt,yshift=0pt] (9,-2) -- (0,-2) node [black,midway,yshift=-0.5cm,anchor=north] {Block RAM size};
        
        % Center pixel
        \draw[black,line width=0.5mm] (2,3) rectangle (3,4);
        % Window size
        \draw[black,line width=0.5mm] (0,1) rectangle (5,6);
        
        % Buffer A
        % \draw[draw={rgb:red,1;green,2;blue,3},line width=1mm] (-0.05,-0.05) rectangle (7.95,5.05);
        
        % Buffer B
        % \draw[draw={rgb:red,3;green,1;blue,2},line width=1mm] (8.05,-0.05) rectangle (16.05,5.05);
        
        % Left Arrows
        \path[draw,-{Latex[length=2.5mm]}] (-1,5.5) -- (0,5.5) 
            node[near start,anchor=east,align=right,xshift=-0.2cm] {\texttt{cache\_w\_ptr}};
        % \path[draw,-{Latex[length=2.5mm]}] (-1,0.5) -- (0,0.5) 
        %     node[near start,anchor=east,align=right,xshift=-0.2cm] {};
        % \foreach \y in {1.5,2.5,3.5,4.5}            
        %     \node[circle,fill=gray,minimum size=0.2cm,inner sep=0pt] () at (-0.5,\y) {};

        % Top arrows
        \path[draw,-{Latex[length=2.5mm]}] (0.5,7.0) -- (0.5,6) 
            node[near start,anchor=west,align=right,xshift=-0.2cm,rotate=90] {\texttt{col\_w\_ptr}};
        % \path[draw,-{Latex[length=2.5mm]}] (8.5,7.0) -- (8.5,6) 
        %     node[near start,anchor=west,align=right,xshift=-0.2cm,rotate=90] {};
        % \foreach \x in {1.5,2.5,3.5,4.5,5.5,6.5,7.5}            
        %     \node[circle,fill=gray,minimum size=0.2cm,inner sep=0pt] () at (\x,6.5) {};

        % Right arrows
        \path[draw,-{Latex[length=2.5mm]}] (10,5.5) -- (9,5.5) 
            node[near start,anchor=west,align=left,xshift=0.2cm] {\texttt{cache\_r\_base}};
        \path[draw,-{Latex[length=2.5mm]}] (10,1.5) -- (9,1.5) 
            node[near start,anchor=west,align=left,xshift=0.2cm] {\texttt{cache\_r\_tip}};
        \path[draw,-{Latex[length=2.5mm]}] (10,4.5) -- (9,4.5) 
            node[near start,anchor=west,align=left,xshift=0.2cm] {\texttt{cache\_r\_ptr}};
        % \foreach \y in {2.5,3.5}            
        %     \node[circle,fill=gray,minimum size=0.2cm,inner sep=0pt] () at (9.5,\y) {};

        % Bot arrows
        \path[draw,-{Latex[length=2.5mm]}] (0.5,-1) -- (0.5,0) 
            node[near start,anchor=east,align=left,xshift=0.2cm,yshift=.5cm,rotate=90] {\texttt{col\_r\_ptr}};
        % \path[draw,-{Latex[length=2.5mm]}] (8.5,-1) -- (8.5,0) 
        %     node[near start,anchor=east,align=left,xshift=0.2cm,rotate=90] {};
        % \foreach \x in {1.5,2.5,3.5,4.5,5.5,6.5,7.5}            
        %     \node[circle,fill=gray,minimum size=0.2cm,inner sep=0pt] () at (\x,-0.5) {};


        % Axis
        \foreach \y in {0,1,2,3,4,5}
            \node[anchor=north] at ($(-0.2,6)-(0,\y)$)  {$\y$};
        % Axis
        \foreach \x in {0,1,2,3,4,5,6,7,8}
            \node[anchor=south] at ($(0.1,6.0)+(\x,0)$)  {$\x$};

    \end{pgfonlayer}

    % Foreground
    \begin{pgfonlayer}{foreground}
        
    \end{pgfonlayer} 

    % Background
    \begin{pgfonlayer}{background}
        % Grid
        \draw[step=1cm,black,thin] (0,0) grid (9,6);
    \end{pgfonlayer} 

\end{tikzpicture}
    \end{adjustbox}
    \caption{Memory layout for solution B}
    \label{fig:solbmemlayout}
\end{figure}

The write mechanism consists of two pointers. The \texttt{cache\_w\_ptr} pointer
points to the current cache in which the input line is stored. The 
\texttt{col\_w\_ptr} points to the location in the cache of the current pixel to
be stored. \texttt{col\_w\_ptr} is incremented every pixel and set to zero at
the start of a new line. \texttt{cache\_w\_ptr} is incremented if an image line
is received and wraps at $CACHE\_N\_LINES$.

The read mechanism consists of a total of four pointers. The \texttt{col\_r\_ptr}
pointer points to the current image coloumn that is sent to the Wallis filter.
It is incremented after $WINDOW\_LENGTH$ pixels are sent and set to zero at
the beginning of a new line. \texttt{cache\_r\_base} points to the top most
cache of image
data and \texttt{cache\_r\_tip} to the bottom most. They are both incremented
after an image line is processed and wrap at $CACHE\_N\_LINES$. The 
\texttt{cache\_r\_ptr} pointer points to the current cache from which the pixel
is read and increments if a pixel is sent to the Wallis core. It wraps to zero
if it reaches $CACHE\_N\_LINES$ and wraps to \texttt{cache\_r\_base} if it
reaches \texttt{cache\_r\_tip}. 

With the pointer mechanics dissected, new conditions for read and write can be
concluded. Listing \ref{lst:mmureadcond} shows the read condition. The
difference to the simple FiFo condition is that if not looped the write pointer
must be greater than the read tip. This ensures that always enough data is in
memory for a Wallis operation.

\begin{minipage}{\linewidth}
    \begin{lstlisting}[
        style=VHDLStyle, 
        caption=MMU read condition, 
        label=lst:mmureadcond
        ]
if ((looped = true) or (cache_w_ptr > cache_r_tip)) then
    DataOut <= Memory(BRAM_SIZE*cache_r_ptr+row_r_ptr);
end if;\end{lstlisting}
\end{minipage}

Listing \ref{lst:mmuwritecond} shows the write condition. It only differs in
that the write pointer may not be equal to the read base instead of the read
pointer as in the case of the simple FiFo.

\begin{minipage}{\linewidth}
    \begin{lstlisting}[
        style=VHDLStyle, 
        caption=MMU write condition, 
        label=lst:mmuwritecond
        ]
if ((looped = false) or (cache_w_ptr /= cache_r_base)) then
    Memory(BRAM_SIZE*cache_w_ptr+row_w_ptr) := DataIn;
end if;\end{lstlisting}
\end{minipage}

The looped flag is set if \texttt{cache\_w\_ptr} wraps to zero and is cleared if
\texttt{cache\_r\_tip} wraps to zero.

These read and write conditions are used to access the memory and to drive the
side channels of the AXI4-Stream (\texttt{tready} and \texttt{tvalid}). The 
\texttt{tlast} signal is used to indicate the end of a line. Therefore the MMU
can operate with an input and output stream and information about the image
width and window length.

The VHDL code for the memory management unit is mainly composed of processes.
Each pointer is incremented and wrapped in its own process. The memory read and
write accesses are described in processes as well. Finally there are two more
processes: \texttt{p\_out\_pix\_ctr} counts the number of pixels sent and
\texttt{p\_out\_pix\_m1} calculates the number of output pixels to send.



\subsubsection*{Hurdles}
After figuring out how to increment and wrap which pointer and when to allow
reads and writes, the memory management unit worked well in simulation. The
\texttt{dc\_control} and \texttt{axis\_fifo} block were quickly implemented. The
core was tested using VHDL testbenches. A big advantage was that the controller
and Wallis core could be simulated before being implemented on the FPGA which
made fault finding much faster. With this new stateless approach timing issues
began to arise. 

Address calculations had to be done within one clock cycle. While Vivado HLS
optimizes variables and datapaths, this has to be done manually in a VHDL based
approach. For example on critical path was the calculation of 
\texttt{n\_out\_pix\_m1}. This signal is required to compare the output pixel
counter and to signal the last pixel sent to the Wallis core. The calculation is
shown in equation \ref{eq:noutpix}.

\begin{equation}
    n_{out\_pix\_m1} = img\_width \cdot win\_len - 1
    \label{eq:noutpix}
\end{equation}

The parameters $win\_len$ and $img\_width$ were chosen of a size that the
multiplication could
be implemented in a DSP block. The subtraction of one is also done in the same
DSP block. Synthesis results showed that this DSP calculation requires
approximately 3.3ns
which is almost already half of the targeted clock period of 8ns. Together with
the datapath and following logic, this resulted in a critical path exceeding
8ns. To fix this,
this calculation had to be done in a clocked process so that a latch is
inferred. Such timing issues had to be fixed for several signals.

% timing
% 
\subsubsection*{Conclusion}
Because the requirements were already dissected from solution A there was only a
new memory layout to be realised for solution B. With the main difference being
that now the behaviour was not described in C/C++ language but in VHDL. The
total
amount of time spent on the solutions were approximately the same. What had
changed is on what the time was spent. While writing the code required the same
amount of time, debugging and fault finding differed. The HLS solution required
time in understanding how the synthesis tool translated the C/C++ code (referring
to the AXI4-Stream blocking access problem) while it did not require thoughts on
timing
and critical paths. With VHDL these accesses had to be implemented manually and
worked on first try but some time had to be spent on fixing timing issues.

Performance wise, there is the expected increase in throughtput. Output pixels
can be sent to the Wallis filter with every clock cycle as soon as there is
enough image data in the cache. Figure \ref{fig:tracesolb} shows the output data
stream with a $WINDOW\_LENGTH$ of three. 

Another advantage of this solution is that the Wallis and controller components
could be simulated together before implementation. This was not possible with
the HLS based approach. Most of the issues could be resolved during simulation.

\begin{figure}[h!]
    \centering
    \includegraphics[width=\textwidth]{images/controller/vhdlcontrollerout.png}
    \caption{Solution B output data stream}
    \label{fig:tracesolb}
\end{figure}



