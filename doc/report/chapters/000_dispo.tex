% @Author: noah
% @Date:   2017-11-30 15:32:15
% @Last Modified by:   noah
% @Last Modified time: 2017-11-30 16:42:12

% -----------------------------------------------------------------------------
% -----------------------------------------------------------------------------
\chapter{Project Report}
% -----------------------------------------------------------------------------

% -----------------------------------------------------------------------------
\section{Theoretical Background}
% -----------------------------------------------------------------------------
\subsection{image}
In diesem Abschnitt geht es um das Bild, wie es erstellt wird und zur Verfügung steht.
\subsubsection{filter}

\subsubsection{sensor / bayer pattern}
Wie wird ein Bild aufgenommen und wie steht zur Verfügung, RGB-Werte in intensitäten

\subsubsection{filter}
In diesem Abschnitt werden Filter erklärt, was dazu nötig ist --> ganzes Bild / 9 Pixel oder Grauwert / RGB

% -----------------------------------------------------------------------------
\subsection{Ethernet}
In diesem Abschnitt geht es um das Ethernet-Protokoll. Das OSI Modell wird erleutert und die
einzelnen Schichten.
\subsubsection{Physical} 
PHY
\subsubsection{Data Link} 
MAC
\subsubsection{Network} 
IP
\subsubsection{Transport} 
TCP, UDP        

% -----------------------------------------------------------------------------
\subsection{FPGA}
\subsubsection{ressources available}

\subsubsection{criterias}
Area, Speed, latency, throughput


% -----------------------------------------------------------------------------
\subsection{FPGA busses}
\subsubsection{AXI4}
Grobe Einführung in den AXI4 Bus.

\subsubsection{AXI4 Stream}
Das AXI Stream interface ist sehr simpel und wird in diesem Kapitel erleutert. 
Es wird in der Kommunikationskomponente verwendet wenn Daten schnell verarbeitet 
werden müssen.
% -----------------------------------------------------------------------------
\section{Mission}
% -----------------------------------------------------------------------------
%%%%%%%%%%%%%%%%%%%%%%%%%%%%%%%%%%%%%%%%
\subsection{Ausgangslage}
Es geht um die Ausgangslage, was eingehalten werden muss und uns zur Verfügung steht. 

\subsubsection{FPGA board}
HW overview (board)

\subsubsection{Ethernet}
kommunikation von PC auf den FPGA und zurück ist über ethernet zu realisieren 

\subsubsection{Image}
grösse, scalability der Software damit es auf grosse Bilder angewendet werden kann

\subsubsection{HLS}
Image Filter wird mit HLS synthetisiert und ip core generiert

%%%%%%%%%%%%%%%%%%%%%%%%%%%%%%%%%%%%%%%%
\subsection{Possible Solutions}
\subsubsection{Image Processing}
Verschiedene Image Algorithmen werden aufgezeigt und was dazu vom Bild vorhanden sein muss: 
debayering, edge dedection, color space transformation

\subsubsection{Communication}
Die Vor- und Nachteile folgender Punkte werden aufgezeigt:
UDP, TCP, fertige IP's, bestehende Protokolle

%%%%%%%%%%%%%%%%%%%%%%%%%%%%%%%%%%%%%%%%
\subsection{Doncept}
\subsubsection{Top-Level block design}
Blockschaltbild wird aufgezeigt (PC - UDP - Ethernet - BRAM - Filter - BRAM - Ethernet - PC)

Blöcke werden beschrieben und Entscheide werden dargestellt von possible solutions

% -----------------------------------------------------------------------------
\section{Image Processing}
% -----------------------------------------------------------------------------
\subsection{requirements}
Was ist für die Sobel-Edge-Detection notwendig und wie müssen die Pixelwerte vorliegen

\subsection{concept}
Das Konzept des Programmierten Filters. Wie das Filter genau funktioniert. 

\subsection{scalability inside FPGA}
Was wurde gemacht, damit der Filter im FPGA auf mehrere aufgeteilt werden kann. 

\subsection{scalability to distribute onto multiple FPGAs}
Wie das Bild auf mehrere FPGAs aufgeteilt werden kann.

% -----------------------------------------------------------------------------
\section{Communication}
% -----------------------------------------------------------------------------
\subsection{Vivado Design Flow}
IP Integrator erklären, Block Design, ...

\subsection{Requirements}
Was für Anforderungen sind gestellt an die Kommunikation: Speed, Skalierbarkeit,
Einfache Implementation auf FPGA, wenig Overhead, Basierend auf UDP, kein Datenverlust.

\subsection{Concept}
Es wird ein eigenes Protokoll auf UDP aufgebeaut und dieses im FPGA implementiert. 
Als Eingang reicht eine Adresse und Dateigrösse, der Rest übernimmt der UFT Stack.
Das Empfangen erfolgt gleichermassen.

\subsection{Protocol Specification}
Das UFT Protokoll wird erleutert und ein Kommunikationsablauf gezeigt.

\subsection{Used Cores}
Der Tri Mode Ethernet MAC und der UDP IP Stack wird verwendet. Ein- und Ausgänge 
beschreiben und wie sie verwendet werden sollen.

\subsection{UDP File Transfer Stack}
Blockdiagramm der FPGA-Implementation.
\subsubsection{Reception}
Blockschaltbild des Empfangspfades
\textbf{Rx}
Haupt Empfängerblock, erkennt Daten/Command und entschlüsselt die Datenfelder.
\textbf{Rx memory control}
Speichert die empfangenen Daten in einem FiFo und sendet sie danach an den AXI
Master Burst block.
\subsubsection{Transmission}
Blockschaltbild des Senderpfades.
\textbf{Command assembler}
Erstellt ein Command Packet.
\textbf{Data assembler}
Erstellt ein Data Packet.
\textbf{Arbiter}
Entscheidet welche der beiden Paketgenerator senden darf.
\textbf{Control}
Steuert den Sendeprozess.
\subsubsection{Implemented Features}
Es ist noch nicht der gesamte UFT Stack im FPGA implementiert. Hier wird beschrieben
was funktioniert und unter welchen Einbussen (keine Verifikation dass die Daten
angekommen sind, keine Checksummengeneration)

% -----------------------------------------------------------------------------
\section{Verification}
% -----------------------------------------------------------------------------
%%%%%%%%%%%%%%%%%%%%%%%%%%%%%%%%%%%%%%%%
\subsection{image processing}
\subsubsection{C-Simulation}
Eine Testbench für die Simulation muss vorhanden sein, um die Software auf dem PC zu testen.

\subsubsection{Synthesis}
Der Code wird synthetisiert und kann analysiert werden. Daraus erhält man Informationen über die Latenz, Area, Timing, ...

\subsubsection{Co-Simulation}
Der Code wird Hardwaremässig auf dem PC simuliert. Die C- und Co-Simulation müssen übereinstimmen

\subsubsection{Simulation on FPGA}
Der generierte IP-Core wird mittels JTAG-Interface getestet. Dafür ist eine Testbench nötig, welche Daten auf den FPGA schreibt und liest.

%%%%%%%%%%%%%%%%%%%%%%%%%%%%%%%%%%%%%%%%
\subsection{Communication}
\subsubsection{HDL Testbench}
Die einzelnen Entities werden mittels VHDL-Testbenches simuliert und der Signalverlauf
kontrolliert. Datestreams werden in Dateien abgelegt und so verifiziert. Mit dem
Computer werden Datenstreams generiert womit die DUV stimuliert wird.

\subsubsection{PC to FPGA Communication}
Das implementierte Design wird in einem Netzwerk mit einem Computer getestet.
Die empfangenen Pakete werden mit einen Paketanalyzer validiert. Zum Stimulus
werden mit C-Code Pakete generiert. Mit tcl-Skripts können die auf dem FPGA
empfangenen Daten validiert werden Schlussendlich werden komplette Dateien 
versendet.

\subsubsection{Synthesis and Implementation Reports}
Durch diese Reports kann sichergestellt werden, dass keine unnötigen Latches
erstellt werden, kein Speicher verschwendet wird...

%%%%%%%%%%%%%%%%%%%%%%%%%%%%%%%%%%%%%%%%
\subsection{Complete System}
An das Gesamtsystem werden Bilder gesendet, verarbeitet und wieder empfangen. 
Dieselben Bilder werden mit einem C-Programm verarbeitet und danach verglichen.

% -----------------------------------------------------------------------------
\section{Conclusion}
% -----------------------------------------------------------------------------
Was funktioniert, was funktioniert nicht? Wie kann weitergefahren werden?

% -----------------------------------------------------------------------------
% -----------------------------------------------------------------------------
\chapter{Developer Guide}
% -----------------------------------------------------------------------------
% -----------------------------------------------------------------------------

% -----------------------------------------------------------------------------
\section{HLS}
% -----------------------------------------------------------------------------
% -----------------------------------------------------------------------------
\subsection{Workflow}
Es wird beschrieben, was die Vorteile von HLS sind

\subsubsection{Pragmas}
interfaces (BRAM, FIFO), Unroll, Pipline, Resource, ...

\subsubsection{Validation}
Wie ist der Workflow im HSL: C-Simulation, Synthese, Co-Simulation, IP Core

% -----------------------------------------------------------------------------
\section{Folder structure}
% -----------------------------------------------------------------------------
Aufzeigen des git repositories, welche Dateien wo zu finden sind. Wo können neue
Komponenten hinzugefügt werden und auf was ist zu achten.

\subsection{Cores}
Wie sind die cores verzeichnisse aufgebaut und wie kann ein neuer Core geschrieben 
werden.

\subsection{Projects}
Das Blockdesign wird mittels tcl erstellt. Wie kann ein neues Projekt erstellt werden.

\subsection{Configuration and Constraints}
Wo befinden sich die Constraints und die Pin definitionen. Dies ist wichtig,
falls die Hardware angepasst werden will.

% -----------------------------------------------------------------------------
\section{Build process}
% -----------------------------------------------------------------------------

\subsection{Requirements}
Welche toolchains werden verwendet, welches Betriebssystem.

\subsection{Process}
Wie kann das Projekt von einer clean-copy kompiliert werden.

% -----------------------------------------------------------------------------
\section{HDL Simulation}
% -----------------------------------------------------------------------------
Wie kann mittels make command die Simulation eines Cores gestartet und die Ergebnisse
verifiziert werden.

% -----------------------------------------------------------------------------
\section{On FPGA Debugging}
% -----------------------------------------------------------------------------

\subsection{JTAG to AXI Master}
Mit diesem IP kann per JTAG auf den Adressbereiches des AXI Busses zugegriffen werden.
So können einzelne AXI Cores auf dem FPGA getestet werden. Mit tcl Skripts 
wird die Bedienung einfach.

\subsection{System ILA}
Der System ILA erlaubt es, interne Signale zu untersuchen und Busse zu kontrollieren.
So können Buszugriffe und andere low-level Vorgänge validiert werden.

% -----------------------------------------------------------------------------
\section{AXI Bus}
% -----------------------------------------------------------------------------

\subsection{AXI Master Burst}
Mit diesem IP kann einfacher auf den AXI Bus zugegriffen werden. Zu Simulationszwecken
wurde ein Model diese IP erstellt. So können Cores simuliert werden.


