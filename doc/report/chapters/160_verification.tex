% ==============================================================================
%
%                             Verification & Benchmark
%
% ==============================================================================
\chapter{Verification \& Benchmark} \label{chapt:ver_bench}
With the image processing an dataflow parts implemented, they can be verified
and benchmarked. The next chapters hold the verification process of both the
image processing part and the dataflow part. After both components are verified
they are benchmarked as one unity against a computer based implementation in
chapter \ref{ch:benchmark}.


% ==============================================================================
%
%                             Verification
%
% ==============================================================================
\section{Verification} \label{ch:verification}
The verification process ensures that all components work as expected. It is
split into the image processing and dataflow parts. They were tested
independantly to reduce complexity and simulation time. The system as a unity is
then tested in chapter \ref{ch:benchmark} benchmark.

% ==============================================================================
%
%                             Image Processing
%
% ==============================================================================
\subsection{Image Procession}\label{ch:verification:imageprocessing}

% ==============================================================================
%
%                             Dataflow
%
% ==============================================================================
\subsection{Dataflow}\label{ch:verification:dataflow}
The dataflow part is again divided into two parts, the communication and
controller parts. While in the communication part only the most recent version
is tested (reffering to chapter \ref{chapt:dataflow} explaining the solotion
A and B with streaming interface), both versions of the controller are verified.

\subsubsection*{Communication}
The communication part was in a large part taken from the last semester project
and was been thoroughly tested and validated in the project report 
\cite{p5report}. The three new implemented features are verified in this
chapter. They consist of:
\begin{itemize}
    \item Acknowledge
    \item User registers
    \item Stream interface
\end{itemize}

\vspace{1ex}
\textbf{Acknowledge:} To test the acknowledge function, a file was sent to the
FPGA from the computer
and the Ethernet traffic monitored with Wireshark Netowrk Protocol Analyzer.
Wiresharks packet dissections are exported to a json file that was then analyzed
using the \texttt{uftcheck} utility. It was written to analyze network pacekts
for UFT transfers. The first lines of output yielded:

    \begin{adjustbox}{max width=1.7\textwidth}
\begin{minipage}{2\linewidth}
    \begin{lstlisting}[
        % basicstyle=\small, %or \small or \footnotesize etc.
        % columns=fullflexible,
        % frame=single,
        % breaklines=false,
        % style=CStyle, 
        % caption=ack buffer allocation, 
        % label=lst:ackbufalloc
        ]
+------+----------------------+----------------------+-----+------------+------+------+
| Pack |         From         |          To          | D/C |  Control   | TCID | SEQ  | 
+------+----------------------+----------------------+-----+------------+------+------+
|  1   | 192.168.5.10 (50719) | 192.168.5.9 (42042)  |  C  |  FT Start  |  12  | 1036 |
|  2   | 192.168.5.10 (50719) | 192.168.5.9 (42042)  |  D  |            |  12  |  0   |
|  3   | 192.168.5.9 (42042)  | 192.168.5.10 (50719) |  C  | ACK packet |  12  |  0   |
|  5   | 192.168.5.10 (50719) | 192.168.5.9 (42042)  |  D  |            |  12  |  1   |
|  6   | 192.168.5.9 (42042)  | 192.168.5.10 (50719) |  C  | ACK packet |  12  |  1   |\end{lstlisting}
\end{minipage}
\end{adjustbox}

The sending PC starts a file transmission with a file transfer start packet and
the first data packet. The next packet (3) is comming from the FPGA to the PC
and acknowledges the first data packet (sequence 0). To verify that all packets
are acknowledged, the sending program reports an acknowledge status after the
files has been sent. 

\begin{minipage}{\linewidth}
    \begin{lstlisting}[
        % basicstyle=\small, %or \small or \footnotesize etc.
        % columns=fullflexible,
        % frame=single,
        % breaklines=false,
        % style=CStyle, 
        % caption=ack buffer allocation, 
        % label=lst:ackbufalloc
        ]
$ ./sender 192.168.5.9 42042 payload/cat.jpg
UFT Sender demo
destination 192.168.5.9:42042
HURRAY! All 1036 packets have been acknowledged.
time elapsed: 1.18s Speed: 0.859 MB/s Size: 1.012 MB\end{lstlisting}
\end{minipage}

\vspace{1ex}
\textbf{User register:} To test wether the user registers can be written and are
output correctly by the communication core, a integrated logic analyzer (ILA)
was used. An ILA can be configured on the FPGA to record internal signals. The
results are transferred to the PC over USB and displayed in \gls{vivadohlx}. 


\subsubsection*{Controller}

% ==============================================================================
%
%                             Benchmark
%
% ==============================================================================
\section{Benchmark}\label{ch:benchmark}

