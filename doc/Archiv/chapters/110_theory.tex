% ==============================================================================
%
%                             Theory
%
% ==============================================================================
% -----------------------------------------------------------------------------
\chapter{Theoretical Background} \label{chapt:theoreticalback}
% -----------------------------------------------------------------------------
This chapter provides the basics of the project. On the one hand, it is about image and image processing as well as communication via Ethernet. How an image is recorded and what is necessary for filters and sensors is explained in the chapter \ref{chapt:imag}. Image processing follows in the chapter \ref{chapt:theroy_imageprocessing} and the communication in the chapter \ref{chapt:ethernet}

% ==============================================================================
%
%                                   Image
%
% ==============================================================================
\section{Image} \label{chapt:imag}
This section explains how an image is recorded by the camera. Also, how an image filter is built for a color image.

\subsection{Image Sensor}
An image sensor is a light sensor array that detects light intensity.
In this process light is converted into an electrical signal and saved. The most commonly used image sensors in the visible range are charge-coupled devices (CCD) and active pixel sensors (CMOS). In order not to distort the light intensities in the visible range, an IR filter is often placed in front of the sensor (see figure \ref{fig:image_sensor}).
\\

\textbf{CCD:} With the CCD sensor, each pixel is shifted to the output. At the output the signal is amplified by an amplifier. Manufacturing is therefore cheaper than the CMOS sensor and is used for cheap cameras \cite{ccd_cmos}.
\\

\textbf{CMOS:} CMOS sensors have an amplifier for each pixel. Thus CMOS sensors are faster than CCD sensors and are used in most cameras \cite{ccd_cmos}.

\subsection{Bayer Filter}
A bayer filter is a color filter array for arranging RGB color filters on a
square grid of photosensors (see figure \ref{fig:image_sensor}). There are twice as many green as red and blue
pixels in order to emulate the higher sensitivity of the human eye to green
light. Each pixel only measures one color due to the filter. The two missing
colors are interpolated in post processing from the color values of the eight adjacent pixels.
This technique is called bayer demosaicing.
\begin{figure}[tb!]
    \centering
    \includegraphics[width=0.7\textwidth]{images/theory/image_sensor.png}
    \caption{An image sensor with a bayer filter \cite{image_sensor}}
    \label{fig:image_sensor}
\end{figure}

% ==============================================================================
%
%                            Image Processing
%
% ==============================================================================
\section{Image Processing} \label{chapt:theroy_imageprocessing}
In technical terms, image processing is the processing of image data. This
includes image processing, image analysis and the output of image files 
\cite{image_processing}. Procedures that generate a new image can be
distinguished into point operations, neighborhood operations and global operations based on their input data.

\subsection{Point Operations}
The point operations use the color or intensity information at a given pixel in the image as input, calculates a new intensity value as the result and stores it to the same point in the target image (figure \ref{fig:image_operation}a). Typical applications of point operations are, for example, the correction of contrast and brightness, color correction by rotating the color space or the application of different threshold value methods.

Example for a point operation with $u$ and $v$ being pixel coordinates:
\begin{equation}
    I'(u, v) = f(I(u, v))
    \label{eq:point_operation}
\end{equation}

\subsection{Neighborhood Operations}
Neighborhood operations use a certain amount of neighboring pixels as input (figure \ref{fig:image_operation}b).
They calculate the result and stored it at the reference point in the target
image. Neighborhood operations are often used in convolution filters. These
filters can be used, for instance, to implement smoothing filters such as the
Gauss filter. Convolution filters can also be used to detect edges in an image. This is possible, for example, with the Sobel filter.

Example to calculate the x-derivative and y-derivative with the Sobel matrix \cite{sobel_matrix}:

\noindent\begin{minipage}{.5\linewidth}
\begin{equation}
    G_{x} = I * \begin{bmatrix}
                -1 & 0 & 1 \\ 
                -2 & 0 & 2 \\ 
                -1 & 0 & 1
                \end{bmatrix}
    \label{eq:neighborhood_operation}
\end{equation}

\end{minipage}%
\begin{minipage}{.5\linewidth}

\begin{equation}
    G_{y} = I * \begin{bmatrix}
                -1 & -2 & -1 \\ 
                0 & 0 & 0 \\ 
                1 & 2 & 1
                \end{bmatrix}
    \label{eq:neighborhood_operation}
\end{equation} 

\end{minipage}

\subsubsection*{Border Handling}
With the application of filters, the so-called border handling problem occurs in every image (see figure \ref{fig:image_handling}). Because the filter protrudes beyond the original image.

There are several possible solutions to this issue:
\begin{itemize}
\item The border pixels are not considered. However, the output image is two
pixels smaller in its height and width than the original image
\item If the filter exceeds the original image, the filter coefficients
at the outside of the image are set to zero
\item The pixels required outside the image are extrapolated according to the closest pixels
\item The image is continued periodically
\end{itemize}
    
\begin{figure}[b!]
    \centering
    \includegraphics[width=0.6\textwidth]{images/theory/border_handling.png}
    \caption{Border problem with filter operations \cite{border_handling}}
    \label{fig:image_handling}
\end{figure}

\subsection{Global Operations}
Image analysis often employs global image operations that uses the entire image as input data (figure \ref{fig:image_operation}c). It is often about finding regions or recognizing geometrical objects. A typical representative of this is the Hough transform and the Fourier transform.
\begin{figure}[tb!]
    \centering
    \includegraphics[width=\textwidth]{images/theory/image_operations.jpg}
    \caption{(a) point, (b) neighborhood and (c) global image processing
    operations \cite{image_operation}}
    \label{fig:image_operation}
\end{figure}

% ==============================================================================
%
%                             Ethernet
%
% ==============================================================================
\section{Ethernet Communication} \label{chapt:ethernet}
Exchanging data between two devices can be done using different approaches. The
following section contains an overview of the communication standards used in
local area networks (LAN).

\subsection{Open Systems Interconnection (OSI) Model}
Such a telecommunication system can be characterized by the Open Systems 
Interconnection Model. The OSI model is a stack of seven abstraction layers 
grouped into two groups: The host layers and the media layers (see figure \ref{fig:osi}). 
Each layer serves the layer above it and is served with data from the layer
beneath it. In the following chapters the OSI reference model is used to
characterize the Ethernet standard.

\begin{figure}[tb!]
    \centering
    \includegraphics[width=0.5\textwidth]{images/theory/osi.png}
    \caption{OSI Model \cite{osi}}
    \label{fig:osi}
\end{figure}

\subsection{Physical} \label{chapt:physical}
The first layer of the OSI model is the physical layer. It defines the 
electrical and physical specification of the connection. In the case of local
area networks the connection medium is usually copper. The circuitry required to
implement
the physical layer is done by the PHY-Chip. This integrated circuit provides 
digital access through a media independent interface (MII) to the analog physical
data link.

\subsection{Data Link} 
The data served from the physical layer is then passed to the data link layer.
Its main purpose is to ensure a reliable transfer of data frames between two
nodes connected by a physical layer. It may also provide means to detect errors
that may occur in the physical layer. Ethernet is the protocol used in the data
layer of local area networks and the layer is split into two sublayers, the logical
link control (LLC) and the medium access control (MAC). The LLC provides means
to allow multiple network protocols (OSI layer three) to be multiplexed onto
the same medium. The MAC encapsulates higher level frames into frames 
appropriate to be transmitted by the physical layer.
\\

Figure \ref{fig:eth} shows an Ethernet frame. The first seven bytes consist of a
fixed preamble. It allows devices on the network to easily synchronize their 
clocks for bit-level synchronization. It is followed by the start frame delimiter
(SFD) that marks the beginning of a frame. Sender and receiver MAC addresses 
ensure that the packet is received by the corresponding host and that it can 
reply to the sender. The type field indicates the protocol used on the next layer
(network layer). After the data payload a frame check sequence in form of a CRC
(cyclic redundancy check) is sent to provide error detection. The maximum data
payload size is limited to 1500 bytes.
\\

\clearpage
\begin{figure}[tb!]
    \centering
    \begin{adjustbox}{max width=\textwidth}
        \begin{tikzpicture}[
    rounded corners=0mm,
]
    %nodes
    \node[draw, minimum height=1.0cm] (pre) {Preamble};
    \node[draw, minimum height=1.0cm, right = 0cm of pre] (sfd) {SFD};
    \node[draw, minimum height=1.0cm, right = 0cm of sfd] (dst) {Destination MAC Adr.};
    \node[draw, minimum height=1.0cm, right = 0cm of dst] (src) {Source MAC Adr.};
    \node[draw, align = center, text width=1cm, minimum height=1.0cm, right = 0cm of src] (tp) {Type\\Field};
    \node[draw, minimum height=1.0cm, right = 0cm of tp] (dat) {Data (46 - 1500 Bytes)};
    \node[draw, minimum height=1.0cm, right = 0cm of dat] (pad) {PAD};
    \node[draw, minimum height=1.0cm, right = 0cm of pad] (crc) {CRC};

    \path[draw,-] ($(dst.180) + (0,0)$) -- ++(0,1.2) ;
    \path[draw,-] ($(crc.0) + (0,0)$) -- ++(0,1.2) ;
    \path[draw,{Latex[length=2.5mm]}-{Latex[length=2.5mm]}] ($(dst.180) + (0,1.0)$) -- ($(crc.0) + (0,1.0)$) node [midway, above] () {Basic MAC Frame} ;
\end{tikzpicture}
    \end{adjustbox}
    \caption{Ethernet Frame}
    % \includegraphics[width=\textwidth]{images/theory/ethernet.png}
    % \caption{Ethernet Frame \cite{ethernet}}
    \label{fig:eth}
\end{figure}
The medium access controller is implemented in hardware to ensure that every bit
is received and stored. The transmit and receive data is commonly stored in
FiFo (first in first out) buffers. This way the next layer in the OSI model is
not required to have low latency capability.

\subsection{Network} 
The data link layer provides means to send frames across nodes in the same
network. As soon as the destination node is in another network, a network layer
is required. Using logical device addresses, network packets can be routed
across different networks and on different media. This allows data to be sent
over long distances.
\\

The most commonly used network layer is the Internet Protocol version 4 (IPv4).
It consists of a 20 byte sized header that contains destination and source IP
addresses, total length, checksum and other fields. The protocol field indicates
what layer four protocol is used. Figure \ref{fig:ip} shows a complete IPv4
header.

\clearpage
\begin{figure}[tb!]
    \centering
    \includegraphics[width=\textwidth]{images/theory/ip.png}
    \caption{IPv4 header \cite{ip}}
    \label{fig:ip}
\end{figure}
\begin{figure}[tb!]
    \centering
    \includegraphics[width=\textwidth]{images/theory/tcp.png}
    \caption{TCP Header \cite{tcpudp}}
    \label{fig:tcp}
\end{figure}
\begin{figure}[tb!]
    \centering
    \includegraphics[width=\textwidth]{images/theory/udp.png}
    \caption{UDP Header \cite{tcpudp}}
    \label{fig:udp}
\end{figure}
\clearpage

\subsection{Transport} 
The transport layer is the first layer in the OSI model that is not required by
the network. Its main purpose is to control the communication of different
applications on two hosts. Therefore a port number is required to distinguish
between the different applications utilizing the same network connection. The
transport layer may also provide segmentation of the data, guarantee of delivery
and flow control to avoid network jam.
\\

The two most used transport layer protocols are the Transmission Control
Protocol (TCP) and the User Datagram Protocol (UDP). Table \ref{tab:tcpudp}
shows the main differences between these two protocols. The most important
aspect is the protocol connection setup. While UDP is connection less (data is
sent without setup), TCP establishes a connection between host and client
prior to data transmission. This ensures a reliable data delivery hence all
messages are acknowledged. UDP has its benefits in lower overhead and for that
reason has slightly higher transmission speed but the protocol does not
guarantee
that the message has been received by the client.

\begin{table}[h]
    \centering
    \begin{tabular}{ l  c  c }
        \toprule
         & \textbf{TCP} & \textbf{UDP} \\
        \midrule
        Connection oriented & Yes & No \\
        Header size & 20 Byte & 8 Byte \\
        Reliable transmission & Yes & No \\
        Acknowledge & Yes & No \\
        Segmentation & Yes & No \\
        Best for & reliable transfer & fast transfer  \\
        \bottomrule
    \end{tabular}
    \caption{TCP vs. UDP}
    \label{tab:tcpudp}
\end{table}

% ==============================================================================
%
%                             FPGA
%
% ==============================================================================
% \section{FPGA}


% \subsection{Available Resources}

% \subsection{Criteria}
% Area, Speed, latency, throughput


% ==============================================================================
%
%                             FPGA Bus
%
% ==============================================================================
% \section{FPGA Bus}
% \subsection{AXI4}


